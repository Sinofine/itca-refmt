\documentclass{book}
\usepackage{hyperref}
\usepackage{unicode-math}
\setmathfont{texgyrepagella-math.otf}[math-style=TeX]
\usepackage{fontspec}
\setmainfont{TeX Gyre Pagella}
\usepackage{standalone}
\usepackage{biblatex}
\addbibresource{intro.bib}
\usepackage{amsthm}
\newtheorem{theorem}{Theorem}[chapter]
\newtheorem{proposition}[theorem]{Proposition}
\newtheorem{lemma}[theorem]{lemma}
\newtheorem*{example}{Example}
\theoremstyle{definition}
\newtheorem{definition}[theorem]{Definition}
\theoremstyle{remark}
\newtheorem*{remark}{Remark}
\title{Introduction to Commutative Algebra}
\author{M. F. ATIYAH \and I. G. MACDONALD}
\begin{document}
\frontmatter
\maketitle
\tableofcontents
\begin{refsection}
\chapter{Introduction}
Commutative algebra is essentially the study of commutative rings. Roughly
speaking, it has developed from two sources: (1) algebraic geometry and (2)
algebraic number theory. In (1) the prototype of the rings studied is the ring
$k[x_1, \cdots , x_n]$ of polynomials in several variables over a field $k$;
in (2) it is the
ring $\mathbf{Z}$ of rational integers. Of these two the algebro-geometric case is the more
far-reaching and, in its modern development by Grothendieck, it embraces much
of algebraic number theory. Commutative algebra is now one of the foundation
stones of this new algebraic geometry. It provides the complete local tools
for the subject in much the same way as differential analysis provides the tools
for differential geometry.

This book grew out of a course of lectures given to third year under-
graduates at Oxford University and it has the modest aim of providing a rapid
introduction to the subject. It is designed to be read by students who have had a
first elementary course in general algebra. On the other hand, it is not intended
as a substitute for the more voluminous tracts on commutative algebra such as
Zariski-Samuel \cite{zariski1960commutative} or Bourbaki \cite{bourbaki1961algebre}. We have concentrated on certain central
topics, and large areas, such as field theory, are not touched. In content we
cover rather more ground than Northcott \cite{northcott_1953} and our treatment is substantially
different in that, following the modern trend, we put more emphasis on modules
and localization.

The central notion in commutative algebra is that of a prime ideal. This
provides a common generalization of the primes of arithmetic and the points of
geometry. The geometric notion of concentrating attention ``near a point''
has as its algebraic analogue the important process of localizing a ring at a prime
ideal. It is not surprising, therefore, that results about localization can usefully
be thought of in geometric terms. This is done methodically in Grothendieck's
theory of schemes and, partly as an introduction to Grothendieck's work \cite{grothendieck1960elements},
and partly because of the geometric insight it provides, we have added schematic
versions of many results in the form of exercises and remarks.

The lecture-note origin of this book accounts for the rather terse style~ with
little general padding, and for the condensed account of many proofs. We have
resisted the temptation to expand it in the hope that the brevity of our presenta-
tion will make clearer the mathematical structure of what is by now an elegant
and attractive theory. Our philosophy has been to build up to the main theorems
in a succession of simple steps and to omit routine verifications.

Anyone writing now on commutative algebra faces a dilemma in connection
with homological algebra, which plays such an important part in modern
developments. A proper treatment of homological algebra is impossible within
the confines of a small book: on the other hand, it is hardly sensible to ignore it
completely. The compromise we have adopted is to use elementary homological
methods-exact sequences, diagrams, etc.-but to stop short of any results
requiring a deep study of homology. In this way we hope to prepare the ground
for a systematic course on homological algebra which the reader should undertake
if he wishes to pursue algebraic geometry in any depth.

We have provided a substantial number of exercises at the end of each
chapter. Some of them are easy and some of them are hard. Usually we have
provided hints, and sometimes complete solutions, to the hard ones. We are
indebted to Mr. R. Y. Sharp, who worked through them all and saved us from
error more than once.

We have made no attempt to describe the contributions of the many
mathematicians who have helped to develop the theory as expounded in this
book. We would, however, like to put on record our indebtedness to J.-P. Serre
and J. Tate from whom we learnt the subject, and whose influence was the
determining factor in our choice of material and mode of presentation.
\printbibliography[heading=subbibliography]
\end{refsection}
\chapter{Notation and Terminology}
Rings and modules are denoted by capital italic letters, elements of them by
small italic letters. A field is often denoted by $k$. Ideals are denoted by small
German characters. $\mathbf{Z, Q, R, C}$ denote respectively the ring of rational integers,
the field of rational numbers, the field of real numbers and the field of complex
numbers.

Mappings are consistently written on the \textit{left}, thus the image of an element $x$
under a mapping $f$ is written $f(x)$ and not $(x)f$. The composition of mappings
$f: X\to Y, g: Y\to Z$ is therefore $g\circ f$, not $f\circ g$.
A mapping $f: X\to Y$is \textit{injective} if $f(x_1) = f(x_2)$ implies $x_1 =
x_2$; surjective if $f(X) = Y$; bijective if both injective and surjective.

The end of a proof (or absence of proof) is marked thus \qedsymbol·
\newline

Inclusion of sets is denoted by the sign $\subseteq$. We reserve the sign
$\subset$ for strict inclusion. Thus $A\subset B$ means that $A$ is contained in $B$ and is not equal to $B$.
\mainmatter
\chapter{Rings and Ideals}
\chapter{Modules}
\chapter{Rings and Modules of Fractions}
\chapter{Primary Decomposition}
\documentclass{standalone}
\usepackage{unicode-math}
\setmathfont{texgyrepagella-math.otf}[math-style=TeX]
\usepackage{fontspec}
\setmainfont{TeX Gyre Pagella}
\usepackage{amsthm}
\newtheorem{theorem}{Theorem}[chapter]
\newtheorem{proposition}[theorem]{Proposition}
\newtheorem{lemma}[theorem]{lemma}
\newtheorem*{example}{Example}
\theoremstyle{definition}
\newtheorem{definition}[theorem]{Definition}
\theoremstyle{remark}
\newtheorem*{remark}{Remark}
\begin{document}
The decomposition of an ideal into primary ideals is a traditional pillar of ideal
theory. It provides the algebraic foundation for decomposing an algebraic
variety into its irreducible components---although it is only fair to point out
that the algebraic picture is more complicated than na\"{i}ve geometry would
suggest. From another point of view primary decomposition provides a generalization
of the factorization of an integer as a product of prime-powers. In
the modern treatment, with its emphasis on localization, primary decomposition
is no longer such a central tool in the theory. It is still, however, of interest in
itself and in this chapter we establish the classical uniqueness theorems.

The prototypes of commutative rings are $\mathbf{Z}$ and the ring of polynomials
 $k[x_1,\cdots,x_n]$ where $k$ is a field; both these are unique factorization domains.
This is not true of arbitrary commutative rings, even if they are integral domains
(the classical example is the ring $\mathbf{Z}[\sqrt{-5}]$, in which the element 6 has two
essentially distinct factorizations, $2\cdot 3$ and $(1+\sqrt{-5})(1-\sqrt{-5})$. However,
there is a generalized form of ``unique factorization'' of \textit{ideals} (not of elements)
in a wide class of rings (the Noetherian rings).

A prime ideal in a ring $A$ is in some sense a generalization of a prime number.
The corresponding generalization of a power of a prime number is a
primary ideal. An ideal $\symfrak{q}$ in a ring $A$ is \textit{primary}
if $\symfrak{q}\neq A$ and if
\[
  xy\in\symfrak{q}\implies\text{either }x\in\symfrak{q}\text{ or
  }y^n\in\symfrak{q}\text{ for some }n>0.
\]
In other words,
\[
\symfrak{q}\text{ is primary}\iff A/\symfrak{q}\ne 0 \text{ and every
  zero-divisor in }A/\symfrak{q}\text{ is nilpotent}.
\]

Clearly every prime ideal is primary. Also the contraction of a primary
ideal is primary, for if $f:A\to B$ and if $\symfrak{q}$ is a primary ideal in
$B$
, then $A/\symfrak{q}^c$ is isomorphic to a subring of $B/\symfrak{q}$.
\begin{proposition}
  Let $\symfrak{q}$ be a primary ideal in a ring $A$. Then $r(\symfrak{q})$ is the smallest
  prime ideal containing $\symfrak{q}$.
\end{proposition}
\begin{proof}
  By (1.8) it is enough to show that $\symfrak{p}=r(\symfrak{q})$ is prime.
  Let $xy\in r(\symfrak{q})$, then $(xy)^m\in \symfrak{q}$ for some $m>0$, and
  therefore either $x^m\in \symfrak{q}$ or $y^{mn}\in \symfrak{q}$ for some
  $n>0$; i.e., either $x\in r(\symfrak{q})$ or $y \in r(\symfrak{q})$.
\end{proof}

If $\symfrak{p}=r(\symfrak{q})$, then $\symfrak{q}$ is said to be
$\symfrak{p}$-primary.
\begin{example}
\begin{enumerate}
\item The primary ideals in $\mathbf{Z}$ are $(0)$ and $(p^n)$, where $p$ is prime. For
  these are the only ideals in $\mathbf{Z}$ with prime radical, and it is immediately checked
  that they are primary.
\item Let $A=k[x,y]$, $\symfrak{q} = (x, y^2)$. Then $A/\symfrak{q} \simeq
  k[y]/(y^2)$,
  in which the zero-divisors are all the multiples of $y$, hence are nilpotent.
  Hence $\symfrak{q}$ is primary, and its radical $\symfrak{p}$ is $(x, y)$. We
  have $\symfrak{p}^2\subset \symfrak{q}\subset \symfrak{p}$ (strict inclusions),
  so that a primary ideal is not necessarily a prime-power.
\item Conversely, a prime power $\symfrak{p}^n$ is not necessarily primary, although its
  radical is the prime ideal $\symfrak{p}$. For example, let $A = k[x, y, z]/(xy
  - z^2)$
  and let $\bar{x},\bar{y}, \bar{z}$ denote the images of $x, y, z$ respectively
  in $A$. Then $\symfrak{p} = (\bar{x}, \bar{z})$ is prime
  (since $A/\symfrak{p}\simeq k[y]$, an integral domain);
  we have $\bar{x}\bar{y} = \bar{z}^2\in \symfrak{p}^2$ but $x \notin
  \symfrak{p}^2$ and $y \not\in r(\symfrak{p}^2 ) = \symfrak{p}$;
  hence $\symfrak{p}^2$ is not primary. However, there is the following result:
\end{enumerate}
\end{example}
\begin{proposition}
  If $r(\symfrak{a})$ is maximal, then $\symfrak{a}$ is primary. In particular, the
  powers of a maximal ideal $\symfrak{m}$ are $\symfrak{m}$-primary.
\end{proposition}
\begin{proof}
  Let $r(\symfrak{a}) = \symfrak{m}$. The image of $\symfrak{m}$ in
  $A/\symfrak{a}$ is the nilradical of $A/\symfrak{a}$, hence $A/\symfrak{a}$ has only one prime ideal, by (1.8). Hence every element of $A/\symfrak{a}$ is either a unit or nilpotent, and so every zero-divisor in $A/\symfrak{a}$ is nilpotent.
\end{proof}
We are going to study presentations of an ideal as an intersection of primary
ideals. First, a couple of lemmas:
\begin{lemma}
  If $\symfrak{q}_i, (1 \le i \le n)$ are $\symfrak{p}$-primary, then $\symfrak{q} = \bigcap_{i=1}^{n}\symfrak{q}_i$ is $\symfrak{p}$-primary.
\end{lemma}
\begin{proof}
  $r(\symfrak{q}) = r(\bigcap_{i=1}^{n} \symfrak{q}_i) = \cap r(\symfrak{q}_i) = \symfrak{p}$. Let $xy\in \symfrak{q}$, $y \not\in
  \symfrak{q}$. Then for some $i$ we have $xy \in \symfrak{q}_i$ and $y\not\in \symfrak{q}_i$ hence $x\in \symfrak{p}$. since $\symfrak{q}_i$ is primary.
\end{proof}
\begin{lemma}
  Let $q$ be a $\symfrak{p}$-primary ideal, $x$ an element of $A$. Then
\begin{enumerate}
\item if $x \in \symfrak{q}$ then $(\symfrak{q}:x) = (1)$;
\item if $x \not\in \symfrak{q}$ then $(\symfrak{q}:x)$ is $\symfrak{p}$-primary, and therefore $r(\symfrak{q}:x) = \symfrak{p}$;
\item if $x\not\in \symfrak{p}$ then $(\symfrak{q}:x) = \symfrak{q}$.
\end{enumerate}
\end{lemma}
\begin{proof}
  i) and iii) follow immediately from the definitions.
  ii): if $y \in (\symfrak{q}:x)$ then $xy \in \symfrak{q}$, hence (as $x \not\in \symfrak{q}$) we have $y\in
  \symfrak{p}$ Hence $\symfrak{q} \subseteq (\symfrak{q}:x)\subseteq \symfrak{p}$;taking radicals, we get $r(\symfrak{q}:x) = \symfrak{p}$.
  Let $yz\in(\symfrak{q}:x)$ with $y\not\in \symfrak{p}$; then
  $xyz \in \symfrak{q}$, hence $xz \in \symfrak{q}$, hence $z \in (\symfrak{q}:x)$.
\end{proof}
A primary decomposition of an ideal $a$ in $A$ is an expression of $a$ as a finite
intersection of primary ideals, say
\begin{equation}
  \label{eq:4.1}
  a=\bigcap_{i=1}^{n}\symfrak{q}_i
\end{equation}
(In general such a primary decomposition need not exist; in this chapter we shall
restrict our attention to ideals which have a primary decomposition.) If more

% over (i) the r(q,) are all distinct, and (ii) we have q,
% nh'l q, (1 ::;;; i ::;;; n) the
% primary decomposition (1) is said to be minimal (or irredundant, or reduced, or
% normal, ... ). By (4.3) we can achieve (i) and then we can omit any superfluous
% terms to achieve (ii); thus any primary decomposition can be reduced to a
% minimal one. We shall say that a is decomposable if it has a primary decomposi-
% tion.
% Theorem 4.5. (1st uniqueness theorem). Let a be a decomposable ideal and
% let a = nf= 1q, be a minimal primary decomposition of a. Let -13t = r(q,)
% (1 ::;;; i ::;;; n). Then the .);!, are precisely the prime ideals which occur in the set
% of ideals r(a:x) (x E A), and hence are independent of the particular de-
% composition of a.
% Proof. For any X E A we have (a:x) =
% q,:x) =
% (q,:x), hence r(a:x) =
% nf. 1r(q1:x) = nx~q1 .)31 by (4.4). Suppose r(a:x) is prime; then by (1.11) we
% have r(a:x) = .)31 for somej. Hence every prime ideal of the form r(a.:x) is one of
% the .);!,. Conversely, for each i there exists x, ¢
% x,
% q,, since the de-
% composition is minimal; and we have r(a:x,) = .);1 1• •
% en
% n
% qb En,,.,
% Remarks. I) The above proof, coupled with the last part of (4.4), shows that
% for each i there exists x, in A such that (a:x,) is .);!,-primary.
% 2) Considering A/a as an A-module, (4.5) is equivalent to saying that the.);!,
% are precisely the prime ideals which occur as radicals of annihilators of elements
% of A/a.
% Example. Let a = (x2, xy) in A = k[x, y]. Then a = .);1 1 n .);!~where .);1 1 = (x),
% The ideal .);!~ is primary by (4.2). So the prime ideals are -lJ1o .);1 2•
% In this example .);1 1 c: .);1 2; we have r(a) = -lJ1 n -lJ2 = -lJ1o but a is not a primary
% ideal.
% .);1 2 = (x, y).
% The prime ideals .);!, in (4. 5) are said to belong to a, or to be associated with a.
% The ideal a is primary if and only if it has only one associated prime ideal. The
% minimal elements of the set {-131> ••• , -lJJ are called the minimal or isolated prime
% ideals belonging to a. The others are called embedded prime ideals. In the
% example above, .);1 2 = (x, y) is.embedded.
% Proposition 4.6. Let a be a decomposable ideal. Then any prime ideal
% .);! 2 a contains a minimal prime ideal belonging to a, and thus the minimal
% prime ideals of a are precisely the minimal elements in the set of all prime
% ideals containing a.
% Proof. If -13 2 a = nf. 1q" then -13 = r(-13) 2 n r(q1) = n .);11. Hence by
% (1.11) we have.);! 2 .);11 for some i; hence.);! contains a minimal prime ideal of a. •
% Remarks. 1) The names isolated and embedded come from geometry. Thus if
% A = k[xh ... , Xnl where k is a field, the ideal a gives rise to a variety X s;;; kn
% (see Chapter 1, Exercise 25). The minimal primes .);!, correspond to the irre-
% ducible components of X, and the embedded primes correspond to subvarieties
\end{document}
\chapter{Integral Dependence and Valuations}
\chapter{Chain Conditions}
\chapter{Noetherian Rings}
\chapter{Artin Rings}
\chapter{Discrete Valuation Rings and Dedekind Domains}
\chapter{Completions}
\chapter{Dimension Theory}
\backmatter
\chapter{Index}
\end{document}
