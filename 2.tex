One of the things which distinguishes the modern approach to Commutative Algebra
is the greater emphasis on modules, rather than just on ideals. The extra
``elbow-room'' that this gives makes for greater clarity and simplicity. For
instance, an ideal $a$ and its quotient ring $A / \mathfrak{a}$ are both
examples of modules and so, to a certain extent, can be treated on an equal
footing. In this chapter we give the definition and elementary properties of
modules. We also give a brief treatment of tensor products, including a
discussion of how they behave for exact sequences.

\section{Modules and module homomorphisms}
Let $A$ be a ring (commutative, as always). An $A$-module is an abelian group
$M$ (written additively) on which $A$ acts linearly: more precisely, it is a
pair $(M, \mu)$, where $M$ is an abelian group and $\mu$ is a mapping of
$A \times M$ into $M$ such that, if we write $a x$ for
$\mu(a, x)(a \in A, x \in M)$, the following axioms are satisfied:

\begin{align*}
  a(x+y) & =a x+a y, \\
  (a+b) x & =a x+b x, \\
  (a b) x & =a(b x), \\
  1 x & =x \tag{$a,b\in A; x,y\in M$}
\end{align*}
(Equivalently, $M$ is an abelian group together with a ring homomorphism
$A \to E(M)$, where $E(M)$ is the ring of endomorphisms of the abelian group
$M$.)

The notion of a module is a common generalization of several familiar concepts,
as the following examples show:
\begin{example}
  \begin{enumerate}
    \item An ideal $\mathfrak{a}$ of $A$ is an $A$-module. In particular $A$
          itself is an $A$-module.
    \item If $A$ is a field $k$, then $A$-module $=k$-vector space.
    \item $A=\mathbf{Z}$, then $\mathbf{Z}$-module = abelian group (define $n x$
          to be $x+\cdots+x$ ).
    \item $A=k[x]$ where $k$ is a field; an $A$-module is a $k$-vector space
          with a linear transformation.
    \item $G=$ finite group, $A=k[G]=$ group-algebra of $G$ over the field $k$
          (thus $A$ is not commutative, unless $G$ is). Then $A$-module
          $=k$-representation of $G$.
  \end{enumerate}
\end{example}

Let $M, N$ be $A$-modules. A mapping $f: M \to N$ is an $A$\textit{-module
  homomorphism} (or is $A$\textit{-linear}) if
\begin{gather*}
  f(x+y)=f(x)+f(y) \\
  f(a x)=a \cdot f(x)
\end{gather*}
for all $a \in A$ and all $x, y \in M$. Thus $f$ is a homomorphism of abelian
groups which commutes with the action of each $a \in A$. If $A$ is a field, an
$A$-module homomorphism is the same thing as a linear transformation of vector
spaces.

The composition of $A$-module homomorphisms is again an $A$-module homomorphism.

The set of all $A$-module homomorphisms from $M$ to $N$ can be turned into an
$A$-module as follows: we define $f+g$ and $af$ by the rules
\begin{align*}
  (f+g)(x) & =f(x)+g(x), \\
  (a f)(x) & =a \cdot f(x)
\end{align*}
for all $x \in M$. It is a trivial matter to check that the axioms for an
$A$-module are satisfied. This $A$-module is denoted by $\Hom_{A}(M, N)$ (or
just $\Hom(M, N)$ if there is no ambiguity about the ring $A$ ).

Homomorphisms $u: M' \to M$ and $v: N \to N''$ induce mappings
$\bar{u}: \Hom(M, N) \to \Hom(M', N)$ and $\bar{v}: \Hom(M, N) \to \Hom(M, N'')$
defined as follows:
\[
  \bar{u}(f)=f \circ u, \quad \bar{v}(f)=v \circ f.
\]
These mappings are $A$-module homomorphisms.

For any module $M$ there is a natural isomorphism $\Hom(A, M) \cong M$: any
$A$-module homomorphism $f: A \to M$ is uniquely determined by $f(1)$, which can
be any element of $M$.

\section{Submodules and quotient modules}
A \textit{submodule} $M'$ of $M$ is a subgroup of $M$ which is closed under
multiplication by elements of $A$. The abelian group $M / M'$ then inherits an
$A$-module structure from $M$, defined by $a(x+M')=a x+M'$. The $A$-module
$M / M'$ is the \textit{quotient} of $M$ by $M'$. The natural map of $M$ onto
$M / M'$ is an $A$-module homomorphism. There is a one-to-one order-preserving
correspondence between submodules of $M$ which contain $M'$, and submodules of
$M/M'$ (just as for ideals; the statement for ideals is a special case).

If $f\colon M \to N$ is an $A$-module homomorphism, the \textit{kernel} of $f$ is the
set
\[
  \Ker(f)=\{x \in M: f(x)=0\}
\]
and is a submodule of $M$. The \textit{image} of $f$ is the set
\[
  \Img(f)=f(M)
\]
and is a submodule of $N$. The \textit{cokernel} of $f$ is
\[
  \Coker(f)=N / \Img(f)
\]
which is a quotient module of $N$.

If $M'$ is a submodule of $M$ such that $M' \subseteq \Ker(f)$, then $f$ gives
rise to a homomorphism $\bar{f}\colon M / M' \to N$, defined as follows: if
$\bar{x} \in M / M'$ is the image of $x \in M$; then $\bar{f}(\bar{x})=f(x)$.
The kernel of $\bar{f}$ is $\Ker(f) / M'$. The homomorphism $\bar{f}$ is said to
be \textit{induced} by $f$. In particular, taking $M'=\Ker(f)$, we have an
isomorphism of $A$-modules

\[
  M / \Ker(f) \cong \Img(f) .
\]

\section{Operations on submodules}
Most of the operations on ideals considered in Chapter 1 have their counterparts
for modules. Let $M$ be an $A$-module and let $(M_{i})_{t \in I}$ be
a family of submodules of $M$. Their \textit{sum} $\sum M_{i}$ is the set of all
(finite) sums $\sum x_{i}$, where $x_{i} \in M_{i}$ for all $i \in I$, and
almost all the $x_{i}$ (that is, all but a finite number) are zero. $\sum M_{i}$
is the smallest submodule of $M$ which contains all the $M_{i}$.

The intersection $\bigcap M_{i}$ is again a submodule of $M$. Thus the
submodules of $M$ form a complete lattice with respect to inclusion.
\begin{proposition}\label{prop:2.1}
  \begin{enumerate}[i)]
    \item If $L \supseteq M \supseteq N$ are $A$-modules, then
          \[
          (L / N) /(M / N) \cong L / M .
          \]
    \item If $M_{1}, M_{2}$ are submodules of $M$, then
          \[
          (M_{1}+M_{2}) / M_{1} \cong M_{2} /(M_{1} \cap M_{2}) .
          \]
  \end{enumerate}
\end{proposition}
\begin{proof}
  \begin{enumerate}[i)]
    \item Define $\theta: L / N \to L / M$ by $\theta(x+N)=x+M$. Then $\theta$
          is a well-defined $A$-module homomorphism of $L / N$ onto $L / M$, and
          its kernel is $M / N$; hence (i).
    \item The composite homomorphism
          $M_{2} \to M_{1}+M_{2} \to(M_{1}+M_{2}) / M_{1}$ is
          surjective, and its kernel is $M_{1} \cap M_{2}$; hence (ii).\qedhere
  \end{enumerate}
\end{proof}

We cannot in general define the \textit{product} of two submodules, but we can
define the product $\mathfrak{a} M$, where $\mathfrak{a}$ is an ideal and $M$ an
$A$-module; it is the set of all finite sums $\sum a_{i} x_{i}$ with
$a_{i} \in \mathfrak{a}, x_{i} \in M$, and is a submodule of $M$.

If $N, P$ are submodules of $M$, we define $(N: P)$ to be the set of all
$a \in A$ such that $a P \subseteq N$; it is an \textit{ideal} of $A$. In
particular, $(0: M)$ is the set of all $a \in A$ such that $a M=0$; this ideal
is called the \textit{annihilator} of $M$ and is also denoted by $\Ann(M)$. If
$\mathfrak{a} \subseteq \Ann(M)$, we may regard $M$ as an
$A / \mathfrak{a}$-module, as follows: if $\bar{x} \in A / \mathfrak{a}$ is
represented by $x \in A$, define $\bar{x} m$ to be $x m(m \in M)$: this is
independent of the choice of the representative $x$ of $\bar{x}$, since
$\mathfrak{a} M=0$.

An $A$-module is \textit{faithful} if $\Ann(M)=0$. If $\Ann(M)=\mathfrak{a}$,
then $M$ is faithful as an $A / \mathfrak{a}$-module.
\begin{exercise}\label{exc:2.2}
  \begin{enumerate}[i)]
    \item $\Ann(M+N)=\Ann(M) \cap \Ann(N)$.
    \item $(N: P)=\Ann\left((N+P) / N\right)$.
  \end{enumerate}
\end{exercise}

If $x$ is an element of $M$, the set of all multiples $a x(a \in A)$ is a
submodule of $M$, denoted by $A x$ or $(x)$. If $M=\sum_{i \in I} A x_{i}$, the
$x_{i}$ are said to be a \textit{set of generators} of $M$; this means that
every element of $M$ can be expressed (not necessarily uniquely) as a finite
linear combination of the $x_{i}$ with coefficients in $A$. An $A$-module $M$ is
said to be \textit{finitely generated} if it has a finite set of generators.

\section{Direct sum and product}
If $M, N$ are $A$-modules, their \textit{direct sum} $M \oplus N$ is the set of
all pairs $(x, y)$ with $x \in M, y \in N$. This is an $A$-module if we define
addition and scalar multiplication in the obvious way:
\begin{align*}
  (x_{1}, y_{1})+(x_{2}, y_{2}) & =(x_{1}+x_{2}, y_{1}+y_{2}) \\
  a(x, y) & =(a x, a y) .
\end{align*}

More generally, if $(M_{i})_{i \in I}$ is any family of $A$-modules,
we can define their {\itshape direct sum} $\bigoplus_{i \in I} M_{i}$; its elements are
families $(x_{i})_{i \in I}$ such that $x_{i} \in M_{i}$ for each
$i \in I$ and almost all $x_{i}$ are 0. If we drop the restriction on the
number of non-zero $x$'s we have the {\itshape direct product} $\prod_{i \in I} M_{i}$.
Direct sum and direct product are therefore the same if the index set $I$ is
finite, but not otherwise, in general.

Suppose that the ring $A$ is a direct product $\prod_{i=1}^{n} A_{i}$ (Chapter
1). Then the set of all elements of $A$ of the form
\[
  (0, \ldots, 0, a_{i}, 0, \ldots, 0)
\]
with $a_{i} \in A_{i}$ is an {\itshape ideal} $\mathfrak{a}_{i}$ of $A$ (it is \textit{not} a subring of
$A$---except in trivial cases---because it does not contain the identity element of
$A$ ). The ring $A$, considered as an $A$-module, is the direct sum of the
ideals $\mathfrak{a}_{1}, \ldots, \mathfrak{a}_{n}$. Conversely, given a module
decomposition
\[
  A=\mathfrak{a}_{1} \oplus \cdots \oplus \mathfrak{a}_{n}
\]
of $A$ as a direct sum of ideals, we have
\[
  A \cong \prod_{i=1}^{n}(A / \mathfrak{b}_{i})
\]
where $\mathfrak{b}_{i}=\bigoplus_{j \neq i} \mathfrak{a}_{j}$. Each ideal
$\mathfrak{a}_{i}$ is a ring (isomorphic to $A / \mathfrak{b}_{i}$ ). The
identity element $e_{i}$ of $\mathfrak{a}_{i}$ is an
idempotent in $A$, and
$\mathfrak{a}_{i}=(e_{i})$.

\section{Finitely generated modules}
A \textit{free} $A$-module is one which is isomorphic to an $A$-module of the form
$\bigoplus_{i \in I} M_{i}$, where each $M_{i} \cong A$ (as an $A$-module). The
notation $A^{(I)}$ is sometimes used. A finitely generated free $A$-module is
therefore isomorphic to $A \oplus \cdots \oplus A$ ( $n$ summands), which is
denoted by $A^{n}$. (Conventionally, $A^{0}$ is the zero module, denoted by 0.)
\begin{proposition}\label{prop:2.3}
  $M$ is a finitely generated $A$-module $\iff M$ is
isomorphic to a quotient of $A^{n}$ for some integer $n>0$.
\end{proposition}
\begin{proof}
  $\implies$: Let $x_{1}, \ldots, x_{n}$ generate $M$. Define
$\phi: A^{n} \to M$ by $\phi(a_{1}, \ldots, a_{n})=$
$a_{1} x_{1}+\cdots+a_{n} x_{n}$. Then $\phi$ is an $A$-module homomorphism onto
$M$, and therefore $M \cong A^{n} / \Ker(\phi)$.

$\impliedby$: We have an $A$-module homomorphism $\phi$ of $A^{n}$ onto $M$. If
$e_{i}=$ $(0, \ldots, 0,1,0, \ldots, 0)$ (the 1 being in the $i$th place), then
the $e_{i}(1 \leq i \leq n)$ generate $A^{n}$, hence the
$\phi(e_{i})$ generate $M$.
\end{proof}
\begin{proposition}\label{prop:2.4}
Let $M$ be a finitely generated $A$-module, let $\mathfrak{a}$ be an ideal
of $A$, and let $\phi$ be an $A$-module endomorphism of $M$ such that
$\phi(M) \subseteq \mathfrak{a} M$. Then $\phi$ satisfies an equation of the
form
\[
  \phi^{n}+a_{1} \phi^{n-1}+\cdots+a_{n}=0
\]
where the $a_{i}$ are in $a$.
\end{proposition}
\begin{proof}
Let $x_{1}, \ldots, x_{n}$ be a set of generators of $M$. Then each
$\phi(x_{i}) \in \mathfrak{a} M$, so that we have say
$\phi(x_{i})=\sum_{j=1}^{n} a_{i j} x_{j}(1 \le i \le n ; a_{i j} \in \mathfrak{a})$,
i.e.,
\[
  \sum_{j=1}^{n}(\delta_{i j} \phi-a_{i j}) x_{j}=0
\]
where $\delta_{i j}$ is the Kronecker delta. By multiplying on the left by the
adjoint of the matrix $(\delta_{i j} \phi-a_{i j})$ it follows that
$\Det(\delta_{i j} \phi-a_{i j})$ annihilates each $x_{i}$, hence is
the zero endomorphism of $M$. Expanding out the determinant, we have an equation
of the required form.
\end{proof}
\begin{corollary}\label{cor:2.5}
Let $M$ be a finitely generated A-module and let $\mathfrak{a}$
be an ideal of $A$ such that $\mathfrak{a} M=M$. Then there exists $x
\equiv 1\pmod{\mathfrak{a}}$ such that $x M=0$.
\end{corollary}
\begin{proof}
  Take $\phi=$ identity, $x=1+a_{1}+\cdots+a_{n}$ in (\ref{prop:2.4}).
\end{proof}
\begin{proposition}
[Nakayama's Lemma]\label{prop:2.6} Let $M$ be a finitely generated $A$-module
and $\mathfrak{a}$ an ideal of $A$ contained in the Jacobson radical
$\mathfrak{R}$ of $A$. Then $\mathfrak{a} M=M$ implies $M=0$.
\end{proposition}
\begin{proof}[First proof]
By (\ref{cor:2.5}) we have $x M=0$ for some $x \equiv 1\pmod{\Re}$. By (\ref{prop:1.9})
$x$ is a unit in $A$, hence $M=x^{-1} x M=0$.
\end{proof}
\begin{proof}[Second proof]
Suppose
$M \neq 0$, and let $u_{1}, \ldots, u_{n}$ be a minimal set of generators of
$M$. Then $u_{n} \in \mathfrak{a} M$, hence we have an equation of the form
$u_{n}=a_{1} u_{1}+\cdots+a_{n} u_{n}$, with the $a_{i} \in \mathfrak{a}$.
Hence
\[
  (1-a_{n}) u_{n}=a_{1} u_{1}+\cdots+a_{n-1} u_{n-1}
\]
since $a_{n} \in \Re$, it follows from (\ref{prop:1.9}) that $1-a_{n}$ is a unit in $A$.
Hence $u_{n}$ belongs to the submodule of $M$ generated by
$u_{1}, \ldots, u_{n-1}$: contradiction.
\end{proof}
\begin{corollary}\label{cor:2.7}
Let $M$ be a finitely generated $A$-module, $N$ a submodule of
$M$, $\mathfrak{a} \subseteq \Re$ an ideal. Then
$M=\mathfrak{a} M+N \implies M=N$.
\end{corollary}
\begin{proof}
  Apply (\ref{prop:2.6}) to $M / N$, observing that
$\mathfrak{a}(M / N)=(\mathfrak{a} M+N) / N$.
\end{proof}

Let $A$ be a local ring, $\mathfrak{m}$ its maximal ideal, $k=A / \mathfrak{m}$
its residue field. Let $M$ be a finitely generated $A$-module.
$M / \mathfrak{m} M$ is annihilated by $\mathfrak{m}$, hence is naturally an
$A / \mathfrak{m}$-module, i.e., a $k$-vector space, and as such is
finite-dimensional.
\begin{proposition}\label{prop:2.8}
Let $x_{i}(1 \le i \le n)$ be elements of $M$ whose
images in $M / \mathfrak{m} M$ form a basis of this vector space. Then the
$x_{i}$ generate $M$.
\end{proposition}
\begin{proof}
Let $N$ be the submodule of $M$ generated by the $x_{i}$. Then the
composite map $N \to M \to M / \mathfrak{m} M$ maps $N$ onto
$M / \mathfrak{m} M$, hence $N+\mathfrak{m} M=M$, hence $N=M$ by (\ref{cor:2.7}).
\end{proof}
\section{Exact sequences}
A sequence of $A$-modules and $A$-homomorphisms

\[
  \cdots \longrightarrow M_{i-1} \stackrel{f_{i}}{\longrightarrow}
  M_{i} \stackrel{f_{i+1}}{\longrightarrow} M_{i+1} \longrightarrow
  \cdots\tag{0}\label{es0}
\]

is said to be {\itshape exact} at $M_{i}$ if
$\Img(f_{i})=\Ker(f_{i+1})$. The sequence is {\itshape exact} if it
is exact at each $M_{i}$. In particular:
\begin{align}
&0 \to M' \stackrel{f}{\to} M\text{ is exact }\Leftrightarrow f\text{ is injective;}\tag{1}  \label{es1}\\
&M \stackrel{g}{\to} M'' \to 0\text{ is exact }\Leftrightarrow g\text{
  is surjective;}\tag{2}\label{es2}\\
    \begin{split}
      &0 \to M' \stackrel{f}{\to} M \stackrel{g}{\to} M'' \to 0\text{ is exact
    $\Leftrightarrow f$ is injective, $g$ is surjective}\\&\text{ and $g$ induces an
    isomorphism of Coker $(f)=M / f(M')$ onto $M''$.}
  \end{split}\tag{3}\label{es3}
\end{align}

A sequence of type (\ref{es3}) is called a {\itshape short exact
  sequence}. Any long exact sequence (\ref{es0}) can be split up into
short exact sequences: if 
$N_{i}=\Img(f_{i})=\Ker(f_{i+1})$, we have short exact
sequences $0 \to N_{i} \to M_{i} \to N_{i+1} \to 0$ for each $i$.
\begin{proposition}
  \label{prop:2.9}
  \begin{enumerate}[i)]
  \item Let
    \begin{equation}
      \label{es4}
  M' \stackrel{u}{\to} M \stackrel{v}{\to} M'' \to 0\tag{4}
    \end{equation}
be a sequence of A-modules and homomorphisms. Then the sequence (\ref{es4}) is exact
$\iff$ for all $A$-modules $N$, the sequence
\begin{equation}
  \label{es4a}
  0 \to \Hom(M'', N) \stackrel{\bar{v}}{\to} \Hom(M, N) \stackrel{\bar{u}}{\to} \Hom(M', N)
  \tag{4'}
\end{equation}
is exact.
\item Let
  \begin{equation}
    \label{es5}
    0 \to N' \stackrel{u}{\to} N \stackrel{v}{\to} N''
    \tag{5}
  \end{equation}
be a sequence of A-modules and homomorphisms. Then the sequence (\ref{es5}) is exact
$\iff$ for all $A$-modules $M$, the sequence
\begin{equation}
  \label{es5a}
  0 \to \Hom(M, N') \stackrel{\bar{u}}{\to} \Hom(M, N) \stackrel{\bar{v}}{\to} \Hom(M, N'')
\tag{5'}
\end{equation}
is exact.
  \end{enumerate}
\end{proposition}
All four parts of this proposition are easy exercises. For example,
suppose that (\ref{es4a}) is exact for all $N$. First of all, since $\bar{v}$ is
injective for all $N$ it follows that $v$ is surjective. Next, we have
$\bar{u} \circ \bar{v}=0$, that is $v \circ u \circ f=0$ for all $f\colon M'' \to N$.
Taking $N$ to be $M''$ and $f$ to be the identity mapping, it follows that
$v \circ u=0$, hence $\Img(u) \subseteq \Ker(v)$. Next take $N=M / \Img(u)$ and
let $\phi\colon M \to N$ be the projection. Then $\phi \in \Ker(\bar{u})$, hence
there exists $\psi\colon M'' \to N$ such that $\phi=\psi \circ v$. Consequently
$\Img(u)=\Ker(\phi) \supseteq \Ker(v)$.\qedsymbol
\begin{proposition}
  \label{prop:2.10}
  Let
  \begin{center}
      \begin{tikzcd}
0 \arrow[r] & M' \arrow[r, "u"] \arrow[d, "f'"] & M \arrow[r, "v"] \arrow[d, "f"] & M'' \arrow[r] \arrow[d, "f''"] & 0 \\
0 \arrow[r] & N' \arrow[r]                      & N \arrow[r]                     & N'' \arrow[r]                  & 0
\end{tikzcd}
\end{center}
be a commutative diagram of A-modules and homomorphisms, with the rows exact.
Then there exists an exact sequence
\begin{align*}
  \label{es6}
   0 \to \Ker(f') \stackrel{\bar{u}}{\to} \Ker(f) \stackrel{\bar{v}}{\to} &\Ker(f'') \stackrel{d}{\to} \\
   &\Coker(f') \stackrel{\bar{u}'}{\to} \Coker(f) \stackrel{\bar{v}'}{\to} \Coker(f'') \to 0\tag{6}
\end{align*}
in which $\bar{u}, \bar{v}$ are restrictions of $u, v$, and $\bar{u}', \bar{v}'$
are induced by $u', v'$.
\end{proposition}
The {\itshape boundary homomorphism} $d$ is defined as follows: if
$x'' \in \Ker(f'')$, we have $x''=v(x)$ for some $x \in M$, and
$v'(f(x))=f''(v(x))=0$, hence $f(x) \in \Ker(v')=\Img(u')$, so that
$f(x)=u'(y')$ for some $y' \in N'$. Then $d(x'')$ is defined to be the
image of $y'$ in $\Coker(f')$. The verification that $d$ is well-defined, and that the
sequence (\ref{es6}) is exact, is a straightforward exercise in diagram-chasing which we
leave to the reader.\qedsymbol
\begin{remark}
(\ref{prop:2.10}) is a special case of the exact homology sequence of homological
algebra.
\end{remark}
Let $C$ be a class of $A$-modules and let $\lambda$ be a function on $C$ with
values in $\mathbf{Z}$ (or, more generally, with values in an abelian group $G$
). The function $\lambda$ is additive if, for each short exact sequence (3) in
which all the terms belong to $C$, we have
$\lambda(M')-\lambda(M)+\lambda(M'')=0$.
\begin{example}
  Let $A$ be a field $k$, and let $C$ be the class of all
finite-dimensional $k$-vector spaces $V$. Then $V \mapsto \dim V$
is an additive function on $C$.
\end{example}
\begin{proposition}
  \label{prop:2.11}
  Let
$0 \to M_{0} \to M_{1} \to \cdots \to M_{n} \to 0$ be an exact sequence of
$A$-modules in which all the modules $M_{i}$ and the kernels of all the
homomorphisms belong to $C$. Then for any additive function $\lambda$ on $C$ we
have
\[
  \sum_{i=0}^{n}(-1)^{i} \lambda(M_{i})=0 .
\]
\end{proposition}
\begin{proof}
  Split up the sequence into short exact sequences
\[
  0 \to N_{i} \to M_{i} \to N_{i+1} \to 0
\]
$(N_{0}=N_{n+1}=0)$. Then we have
$\lambda(M_{i})=\lambda(N_{i})+\lambda(N_{i+1})$.
Now take the alternating sum of the $\lambda(M_{i})$, and everything
cancels out.
\end{proof}
\section{Tensor product of modules}
Let $M, N, P$ be three $A$-modules. A mapping $f\colon M \times N \to P$ is said to
be {\itshape $A$-bilinear} if for each $x \in M$ the mapping $y \mapsto f(x, y)$ of $N$
into $P$ is $A$-linear, and for each $y \in N$ the mapping $x \mapsto f(x, y)$
of $M$ into $P$ is $A$-linear.

We shall construct an $A$-module $T$, called the {\itshape tensor
  product} of $M$ and $N$, with the property that the $A$-bilinear
mappings $M \times N \to P$ are in a natural one-to-one correspondence
with the $A$-linear mappings $T \to P$, for all $A$ modules $P$. More
precisely:
\begin{proposition}
  \label{prop:2.12}
  Let $M, N$ be A-modules. Then there exists a pair $(T, g)$
consisting of an A-module $T$ and an A-bilinear mapping $g: M \times N \to T$,
with the following property:

Given any A-module $P$ and any A-bilinear mapping $f: M \times N \to P$, there
exists a unique $A$-linear mapping $f'\colon T \to P$ such that $f=f' \circ g$ (in
other words, every bilinear function on $M \times N$ factors through $T$ ).

Moreover, if $(T, g)$ and $(T', g')$ are two pairs with this
property, then there exists a unique isomorphism $j\colon T \to T'$ such that
$j \circ g=g'$.
\end{proposition}
\begin{proof}
  \begin{enumerate}[i)]
  \item Uniqueness. Replacing $(P, f)$ by $(T', g')$ we get a
unique $j\colon T \to T'$ such that $g'=j \circ g$. Interchanging the roles of $T$
and $T'$, we get $j'\colon T' \to T$ such that $g=j' \circ g'$. Each of the
compositions $j \circ j', j' \circ j$ must be the identity, and therefore $j$ is
an isomorphism.
\item Existence. Let $C$ denote the free $A$-module $A^{(M \times N)}$. The
elements of $C$ are formal linear combinations of elements of $M \times N$ with
coefficients in $A$, i.e. they are expressions of the form
$\sum_{i=1}^{n} a_{i} \cdot(x_{i}, y_{i})$($a_{i} \in A, x_{i} \in M$, $y_{i} \in N$).

Let $D$ be the submodule of $C$ generated by all elements of $C$ of the
following types:
\[
  \begin{gathered}
    (x+x', y)-(x, y)-(x', y) \\
    (x, y+y')-(x, y)-(x, y') \\
    (a x, y)-a \cdot(x, y) \\
    (x, a y)-a \cdot(x, y) .
  \end{gathered}
\]
Let $T=C / D$. For each basis element $(x, y)$ of $C$, let $x \otimes y$ denote
its image in $T$. Then $T$ is generated by the elements of the form
$x \otimes y$, and from our definitions we have
\[
  \begin{gathered}
    (x+x') \otimes y=x \otimes y+x' \otimes y, x \otimes(y+y')=x \otimes y+x \otimes y', \\
    (a x) \otimes y=x \otimes(a y)=a(x \otimes y)
  \end{gathered}
\]
Equivalently, the mapping $g\colon M \times N \to T$ defined by $g(x, y)=x \otimes y$
is A-bilinear.

Any map $f$ of $M \times N$ into an $A$-module $P$ extends by linearity to an
$A$ module homomorphism $\bar{f}\colon C \to P$. Suppose in particular that $f$ is
$A$-bilinear. Then, from the definitions, $\bar{f}$ vanishes on all the generators of
$D$, hence on the whole of $D$, and therefore induces a well-defined
$A$-homomorphism $f'$ of $T=C / D$ into $P$ such that $f'(x \otimes y)=f(x, y)$.
The mapping $f'$ is uniquely defined by this condition, and therefore the pair
$(T, g)$ satisfy the conditions of the proposition.
\end{enumerate}
\end{proof}
\begin{remark}
\begin{enumerate}[i)]
\item The module $T$ constructed above is called the {\itshape tensor product} of $M$
and $N$, and is denoted by $M \otimes_{A} N$, or just $M \otimes N$ if there is
no ambiguity about the ring $A$. It is generated as an $A$-module by the
``products'' $x \otimes y$. If
$(x_{i})_{i \in I},(y_{j})_{j \in J}$ are families of
generators of $M, N$ respectively, then the elements $x_{i} \otimes y_{j}$
generate $M \otimes N$. In particular, if $M$ and $N$ are finitely generated, so
is $M \otimes N$.
  \item The notation $x \otimes y$ is inherently ambiguous unless we specify the
tensor product to which it belongs. Let $M', N'$ be submodules of $M, N$
respectively, and let $x \in M'$ and $y \in N'$. Then it can happen that
$x \otimes y$ as an element of $M \otimes N$ is zero whilst $x \otimes y$ as an
element of $M' \otimes N'$ is non-zero. For example, take
$A=\mathbf{Z}, M=\mathbf{Z}, N=\mathbf{Z} / 2 \mathbf{Z}$, and let $M'$ be the submodule
$2 \mathbf{Z}$ of $\mathbf{Z}$, whilst $N'=N$. Let $x$ be the non-zero element of $N$ and
consider $2 \otimes x$. As an element of $M \otimes N$, it is zero because
$2 \otimes x=1 \otimes 2 x=1 \otimes 0=0$. But as an element of $M' \otimes N'$
it is non-zero. See the example after (\ref{prop:2.18}).

However, there is the following result:
\begin{corollary}
  \label{coro:2.13}
  Let $x_{i} \in M, y_{i} \in N$ be such that
$\sum x_{i} \otimes y_{i}=0$ in $M \otimes N$. Then there exist finitely
generated submodules $M_{0}$ of $M$ and $N_{0}$ of $N$ such that
$\sum x_{i} \otimes y_{i}=0$ in $M_{0} \otimes N_{0}$.
\end{corollary}
\begin{proof}
  If $\sum x_{i} \otimes y_{i}=0$ in $M \otimes N$, then in the notation of
the proof of (\ref{prop:2.12}) we have $\sum(x_{i}, y_{i}) \in D$, and therefore
$\sum(x_{i}, y_{i})$ is a finite sum of generators of $D$. Let
$M_{0}$ be the submodule of $M$ generated by the $x_{i}$ and all the elements of
$M$ which occur as first coordinates in these generators of $D$, and define
$N_{0}$ similarly. Then $\sum x_{i} \otimes y_{i}=0$ as an element of
$M_{0} \otimes N_{0}$.
\end{proof}
\item We shall never again need to use the construction of the tensor product
given in (\ref{prop:2.12}), and the reader may safely forget it if he prefers. What is
essential to keep in mind is the defining property of the tensor product. 
\item Instead of starting with bilinear mappings we could have started with
multilinear mappings $f: M_{1} \times \cdots \times M_{r} \to P$ defined in the
same way (i.e., linear in each variable). Following through the proof
of (\ref{prop:2.12}) 
we should end up with a ``multi-tensor product''
$T=M_{1} \otimes \cdots \otimes M_{r}$, generated by all products
$x_{1} \otimes \cdots \otimes x_{r}(x_{i} \in M_{i}, 1 \leq i \leq r)$.
The details may safely be left to the reader; the result corresponding
to (\ref{prop:2.12}) is
\begin{thprop}{2.12*}\label{prop:2.12a}
Let $M_{1}, \ldots, M_{r}$ be $A$-modules. Then there
exists a pair $(T, g)$ consisting of an $A$-module $T$ and an $A$-multilinear
mapping $g: M_{1} \times \cdots\times M_{r} \to T$ with the following
property:

Given any $A$-module $P$ and any $A$-multilinear mapping $f: M_{1} \times
\cdots\times M_{r} \to T$, there exists a unique $A$-homomorphism $f': T
\to P$ such that $f' \circ g=f$.

Moreover, if $(T, g)$ and $(T', g')$ are two pairs with this
property, then there exists a unique isomorphism $j: T \to T'$ such that
$j \circ g=g'$.  
\end{thprop}
\end{enumerate}
\end{remark}
There are various so-called ``canonical isomorphisms'', some of which we state
here:
\begin{proposition}
  \label{prop:2.14}
  Let $M, N, P$ be $A$-modules. Then there exist unique isomorphisms
  \begin{enumerate}[i)]
  \item $M \otimes N \to N \otimes M$
  \item $(M \otimes N) \otimes P \to M \otimes(N \otimes P) \to M \otimes N \otimes P$
  \item $(M \oplus N) \otimes P \to(M \otimes P) \oplus(N \otimes P)$
  \item $A \otimes M \to M$
  \end{enumerate}
  such that, respectively,
  \begin{enumerate}[a)]
  \item $x \otimes y \mapsto y \otimes x$
  \item $(x \otimes y) \otimes z \mapsto x \otimes(y \otimes z) \mapsto x \otimes y \otimes z$
  \item $(x, y) \otimes z \mapsto(x \otimes z, y \otimes z)$
  \item $a \otimes x \mapsto a x$.
  \end{enumerate}
\end{proposition}
\begin{proof}
  In each case the point is to show that the mappings so described are well
defined. The technique is to construct suitable bilinear or multilinear
mappings, and use the defining property (\ref{prop:2.12}) or (\ref{prop:2.12a}) to
infer the existence of homomorphisms of tensor products. We shall prove half of
ii) as an example of the method, and leave the rest to the reader.

We shall construct homomorphisms
\[
  (M \otimes N) \otimes P \stackrel{f}{\to} M \otimes N \otimes P \stackrel{g}{\to}(M \otimes N) \otimes P
\]
such that $f((x \otimes y) \otimes z)=x \otimes y \otimes z$ and
$g(x \otimes y \otimes z)=(x \otimes y) \otimes z$ for all
$x \in M, y \in N, z \in P$.

To construct $f$, fix the element $z \in P$. The mapping
$(x, y) \mapsto x \otimes y \otimes z$ $(x \in M, y \in N)$ is bilinear in $x$
and $y$ and therefore induces a homomorphism
$f_{z}\colon M \otimes N \to M \otimes N \otimes P$ such that
$f_{z}(x \otimes y)=x \otimes y \otimes z$. Next, consider the mapping
$(t, z) \mapsto f_{z}(t)$ of $(M \otimes N) \times P$ into
$M \otimes N \otimes P$. This is bilinear in $t$ and $z$ and therefore induces a
homomorphism
\[
  f\colon(M \otimes N) \otimes P \to M \otimes N \otimes P
\]
such that $f((x \otimes y) \otimes z)=x \otimes y \otimes z$.

To construct $g$, consider the mapping
$(x, y, z) \mapsto(x \otimes y) \otimes z$ of $M \times N$
$\times \boldsymbol{P}$ into
$(\boldsymbol{M} \otimes N) \otimes \boldsymbol{P}$. This is linear in each
variable and therefore induces a homomorphism
\[
  g\colon M \otimes N \otimes P \to(M \otimes N) \otimes P
\]
such that $g(x \otimes y \otimes z)=(x \otimes y) \otimes z$.

Clearly $f \circ g$ and $g \circ f$ are identity maps, hence $f$ and $g$ are
isomorphisms.
\end{proof}
\begin{exercise}\label{exc:2.15}
  Let $A, B$ be rings, let $M$ be an $A$-module, $P$ a $B$-module
and $N$ an $(A, B)$-bimodule (that is, $N$ is simultaneously an $A$-module and
a $B$-module and the two structures are compatible in the sense that
$a(x b)=(a x) b$ for all $a \in A$, $b \in B, x \in N)$. Then $M \otimes_{A} N$
is naturally a $B$-module, $N \otimes_{B} P$ an A-module, and we have
\[
  (M \otimes_{A} N) \otimes_{B} P \cong M \otimes_{A}(N \otimes_{B} P).
\]
\end{exercise}
Let $f\colon M \to M', g\colon N \to N'$ be homomorphisms of $A$-modules. Define
$h\colon M \times N \to M' \otimes N'$ by $h(x, y)=f(x) \otimes g(y)$. It is easily
checked that $h$ is $A$-bilinear and therefore induces an $A$-module
homomorphism
\[
  f \otimes g: M \otimes N \to M' \otimes N'
\]
such that
\[
  (f \otimes g)(x \otimes y)=f(x) \otimes g(y) \quad(x \in M, y \in N) .
\]

Let $f': M' \to M''$ and $g': N' \to N''$ be homomorphisms of $A$-modules. Then
clearly the homomorphisms
$(f' \circ f) \otimes(g' \circ g)$ and
$(f' \otimes g') \circ(f \otimes g)$ agree on all elements of the
form $x \otimes y$ in $M \otimes N$. Since these elements generate
$M \otimes N$, it follows that
\[
  (f' \circ f) \otimes(g' \circ g)=(f' \otimes g') \circ(f \otimes g) .
\]
\section{Restriction and extension of scalars}
Let $f\colon A \to B$ be a homomorphism of rings and let $N$ be a $B$-module. Then
$N$ has an $A$-module structure defined as follows: if $a \in A$ and $x \in N$,
then $a x$ is defined to be $f(a) x$. This $A$-module is said to be obtained
from $N$ by {\itshape restriction of scalars}. In particular, $f$ defines in this way an
$A$-module structure on $B$.
\begin{proposition}
  \label{prop:2.16}
  Suppose $N$ is finitely generated
as a $B$-module and that $B$ is finitely generated as an
$A$-module. Then $N$ is finitely generated as an 
$A$-module. 
\end{proposition}
\begin{proof}
  Let $y_{1}, \ldots, y_{n}$ generate $N$ over $B$, and let
$x_{1}, \ldots, x_{m}$ generate $B$ as an $A$-module. Then the $m n$ products
$x_{i} y_{j}$ generate $N$ over $A$.
\end{proof}
Now let $M$ be an $A$-module. Since, as we have just seen, $B$ can be regarded
as an $A$-module, we can form the $A$-module $M_{B}=B \otimes_{A} M$. In fact
$M_{B}$ carries a $B$-module structure such that
$b(b' \otimes x)=b b' \otimes x$ for all $b, b' \in B$ and all
$x \in M$. The $B$-module $M_{B}$ is said to be obtained from $M$ by
{\itshape extension of scalars}.
\begin{proposition}
  \label{prop:2.17}
  If $M$ is finitely generated as an $A$-module, then $M_{B}$ is
finitely generated as a $B$-module.
\end{proposition}
\begin{proof}
  If $x_{1}, \ldots, x_{m}$ generate $M$ over $A$, then the
$1 \otimes x_{i}$ generate $M_{B}$ over $B$.
\end{proof}
\section{Exactness properties of the tensor product}
Let $f\colon M \times N \to P$ be an $A$-bilinear mapping. For each $x \in M$ the
mapping $y \mapsto f(x, y)$ of $N$ into $P$ is $A$-linear, hence $f$ gives rise
to a mapping $M \to\Hom(N, P)$ which is $A$-linear because $f$ is linear in
the variable $x$. Conversely any $A$-homomorphism $\phi\colon M \to \Hom_{A}(N, P)$
defines a bilinear map, namely $(x, y) \mapsto \phi(x)(y)$. Hence the set $S$ of
$A$-bilinear mappings $M \times N \to P$ is in natural one-to-one correspondence
with $\Hom(M, \Hom(N, P))$. On the other hand $S$ is in one-to-one
correspondence with $\Hom(M \otimes N, P)$, by the defining property of the
tensor product. Hence we have a canonical isomorphism
\begin{equation}
  \label{exp1}
  \Hom(M \otimes N, P) \cong \Hom(M, \Hom(N, P)).\tag{1}
\end{equation}
\begin{proposition}
  \label{prop:2.18}
  Let
  \begin{equation}
    \label{exp2}
  M' \stackrel{f}{\to} M \stackrel{g}{\to} M'' \to 0\tag{2}
  \end{equation}
be an exact sequence of $A$-modules and homomorphisms, and let $N$ be any
$A$-module. Then the sequence
\begin{equation}
  \label{exp3}
  M' \otimes N \stackrel{f \otimes 1}{\longrightarrow} M \otimes N \stackrel{g \otimes 1}{\longrightarrow} M'' \otimes N \to 0
\end{equation}
(where 1 denotes the identity mapping on $N$ ) is exact.
\end{proposition}
\begin{proof}
  Let $E$ denote the sequence (\ref{exp2}), and let $E \otimes N$ denote the
sequence (\ref{exp3}). Let $P$ be any $A$-module. Since (\ref{exp2}) is exact, the sequence
$\Hom(E$, Hom $(N, P))$ is exact by (\ref{prop:2.9}); hence by (\ref{exp1}) the sequence
$\Hom(E \otimes N, P)$ is exact. By (\ref{prop:2.9}) again, it follows
that $E \otimes N$ is exact.
\end{proof}
\begin{remark}
\begin{enumerate}[i)]
\item Let $T(M)=M \otimes N$ and let $U(P)=\Hom(N, P)$. Then (\ref{exp1}) takes the
form Hom $(T(M), P)=\Hom(M, U(P))$ for all $A$-modules $M$ and $P$. In the
language of abstract nonsense, the functor $T$ is the left adjoint of $U$, and
$U$ is the right adjoint of $T$. The proof of (\ref{prop:2.18}) shows that any functor
which is a left adjoint is right exact. Likewise any functor which is a right
adjoint is left exact.
\item It is not in general true that, if $M' \to M \to M''$
is an exact sequence of $A$-modules and homomorphisms, the sequence
$M' \otimes N \to M \otimes N \to M'' \otimes N$ obtained by tensoring with an
arbitrary $A$-module $N$ is exact.
\end{enumerate}
\end{remark}
\begin{example}
  Take $A=\mathbf{Z}$ and consider the exact sequence
$0 \to \mathbf{Z} \stackrel{f}{\to} \mathbf{Z}$, where $(x)=2 x$ for all
$x \in \mathbf{Z}$. If we tensor with $N=\mathbf{Z} / 2 \mathbf{Z}$, the
sequence $0 \to \mathbf{Z} \otimes N\stackrel{f \otimes
  1}{\longrightarrow} \mathbf{Z} \otimes N$ is {\itshape not} exact, 
because for any $x \otimes y \in \mathbf{Z} \otimes N$ we have
\[
  (f \otimes 1)(x \otimes y)=2 x \otimes y=x \otimes 2 y=x \otimes 0=0,
\]
so that $f \otimes 1$ is the zero mapping, whereas
$\mathbf{Z} \otimes N \neq 0$.
\end{example}
The functor $T_{N}\colon M \mapsto M \otimes_{A} N$ on the category of $A$-modules
and homomorphisms is therefore not in general exact. If $T_{N}$ is exact, that
is to say if tensoring with $N$ transforms all exact sequences into exact
sequences, then $N$ is said to be a {\itshape flat} $A$-module.
\begin{proposition}\label{prop:2.19}
  The following are equivalent, for an $A$-module $N$ :
\begin{enumerate}[i)]
\item $N$ is flat.
\item If $0 \to M' \to M \to M'' \to 0$ is any exact sequence of $A$-modules, the
tensored sequence $0 \to M' \otimes N \to M \otimes N \to M'' \otimes N \to 0$
is exact.
\item If $f: M' \to M$ is injective, then
$f \otimes 1: M' \otimes N \to M \otimes N$ is injective.
\item If $f: M' \to M$ is injective and $M, M'$ are finitely generated, then
$f \otimes 1: M' \otimes N \to M \otimes N$ is injective.
\end{enumerate}
\end{proposition}
\begin{proof}
  \phantom{}
  \begin{enumerate}
  \item[i) $\Longleftrightarrow$ ii)]by splitting up a long exact sequence into short
exact sequences.
  \item[ii) $\Longleftrightarrow$ iii)]by (\ref{prop:2.18}).
  \item[iii) $\implies$ iv)]clear.
    \item[iv) $\implies$ iii)] Let $f: M' \to M$ be injective and let
$u=\sum x_{i} \otimes y_{i} \in \Ker(f \otimes 1)$, so that
$\sum f(x_{i}') \otimes y_{t}=0$ in $M \otimes N$. Let $M_{0}'$ be
the submodule of $M'$ generated by the $x_{i}'$ and let $u_{0}$ denote
$\sum x_{i}' \otimes y_{i}$ as an element of $M_{0}' \otimes N$. By \ref{prop:2.14} there
exists a finitely generated submodule $M_{0}$ of $M$ containing
$f(M_{0}')$ and such that $\sum f(x_{i}') \otimes y_{i}=0$
as an element of $M_{0} \otimes N$. If $f_{0}: M_{0}' \to M_{0}$ is the
restriction of $f$, this means that
$(f_{0} \otimes 1)(u_{0})=0$. Since $M_{0}$ and $M_{0}'$
are finitely generated, $f_{0} \otimes 1$ is injective and therefore $u_{0}=0$,
hence $u=0$.
  \end{enumerate}
\end{proof}
\begin{exercise}
If $f\colon A \to B$ is a ring homomorphism and $M$ is a flat
$A$-module, then $M_{B}=B \otimes_{A} M$ is a flat $B$-module. (Use the
canonical isomorphisms (\ref{prop:2.14}), (\ref{exc:2.15}).)
\end{exercise}
\section{Algebras}
Let $f\colon A \to B$ be a ring homomorphism. If $a \in A$ and $b \in B$, define a
product
\[
  a b=f(a) b .
\]
This definition of scalar multiplication makes the ring $B$ into an $A$-module
(it is a particular example of restriction of scalars). Thus $B$ has an
$A$-module structure as well as a ring structure, and these two structures are
compatible in a sense which the reader will be able to formulate for himself.
The ring $B$, equipped with this $A$-module structure, is said to be an
{\itshape $A$-algebra}. Thus an $A$-algebra is, by definition, a ring
$B$ together with a ring homomorphism $f\colon A \to B$.
\begin{remark}
\begin{enumerate}[i)]
\item In particular, if $A$ is a field $K$ (and $B \neq 0$ ) then $f$ is
injective by (\ref{prop:1.2}) and therefore $K$ can be canonically identified with its
image in $B$. Thus a $K$-algebra ( $K$ a field) is effectively a ring containing
$K$ as a subring.
  \item  Let $A$ be any ring. Since $A$ has an identity element there is a unique
homomorphism of the ring of integers $\mathbf{Z}$ into $A$, namely
$n \mapsto n .1$. Thus every ring is automatically a $\mathbf{Z}$-algebra.
\end{enumerate}
\end{remark}
Let $f\colon A \to B, g\colon A \to C$ be two ring homomorphisms. An
{\itshape $A$-algebra
homomorphism} $h\colon B \to C$ is a ring homomorphism which is also an $A$-module
homomorphism. The reader should verify that $h$ is an $A$-algebra homomorphism
if and only if $h \circ f=g$.

A ring homomorphism $f\colon A \to B$ is finite, and $B$ is a finite $A$-algebra, if
$B$ is finitely generated as an $A$-module. The homomorphism $f$ is
{\itshape of finite type}, and $B$ is a {\itshape finitely-generated
  $A$-algebra}, if there exists a finite set of elements $x_{1},
\ldots x_{n}$ in $B$ such that every element of $B$ can be 
written as a polynomial in $x_{1}, \ldots, x_{n}$ with coefficients in $f(A)$;
or equivalently if there is an $A$-algebra homomorphism from a polynomial ring
$A\left[t_{1}, \ldots, t_{n}\right]$ onto $B$.

A ring $A$ is said to be {\itshape finitely generated} if it is finitely generated as a
$\mathbf{Z}$ algebra. This means that there exist finitely many elements
$x_{1}, \ldots, x_{n}$ in $A$ such that every element of $A$ can be written as a
polynomial in the $x_{i}$ with rational integer coefficients.

\section{Tensor product of algebras}
Let $B, C$ be two $A$-algebras, $f\colon A \to B, g\colon A \to C$ the corresponding
homomorphisms. Since $B$ and $C$ are $A$-modules we may form their tensor
product $D=B \otimes_{A} C$, which is an $A$-module. We shall now define a
multiplication on $D$.

Consider the mapping $B \times C \times B \times C \to D$ defined by
\[
  (b, c, b', c') \mapsto b b' \otimes c c' .
\]
This is $A$-linear in each factor and therefore, by (\ref{prop:2.12a}),
induces an $A$-module homomorphism
\[
  B \otimes C \otimes B \otimes C \to D,
\]
hence by (\ref{prop:2.14}) an $A$-module homomorphism
\[
  D \otimes D \to D
\]
and this in turn by (\ref{prop:2.11}) corresponds to an $A$-bilinear mapping
\[
  \mu: D \times D \to D
\]
which is such that
\[
  \mu(b \otimes c, b' \otimes c')=b b' \otimes c c' .
\]
Of course, we could have written down this formula directly, but without some
such argument as we have given there would be no guarantee that $\mu$ was
well-defined.

We have therefore defined a multiplication on the tensor product $D=B
\otimes_{A} C$ : for elements of the form $b \otimes c$ it
is given by
\[
  (b \otimes c)(b' \otimes c')=b b' \otimes c c',
\]
and in general by
\[
  \left(\sum_{i}(b_{i} \otimes c_{i})\right)\left(\sum_{j}(b_{j}' \otimes c_{j}')\right)=\sum_{i, j}(b_{i} b_{j}' \otimes c_{i} c_{j}') .
\]
The reader should check that with this multiplication $D$ is a commutative ring,
with identity element $1 \otimes 1$. Furthermore, $D$ is an $A$-algebra: the
mapping $a \mapsto f(a) \otimes g(a)$ is a ring homomorphism $A \to D$.

In fact there is a commutative diagram of ring homomorphisms

\begin{center}
  \begin{tikzcd}
    & B \arrow[rd, "u"]  &   \\
    A \arrow[ru, "f"] \arrow[rd, "g"'] &                    & D \\
    & C \arrow[ru, "v"'] &
  \end{tikzcd}
\end{center}
in which $u$, for example, is defined by $u(b)=b \otimes 1$.
\section{Exercises}
\begin{enumerate}[series=exc2]
  \item Show that $(\mathbf{Z} / m \mathbf{Z}) \otimes_{\mathbf{Z}}(\mathbf{Z} / n \mathbf{Z})=0$ if $m, n$ are coprime.

  \item Let $A$ be a ring, $\mathfrak{a}$ an ideal, $M$ an $A$-module. Show that
        $(A / \mathfrak{a}) \otimes_{A} M$ is isomorphic to
        $M / \mathfrak{a} M$.
        
[Tensor the exact sequence $0 \to \mathfrak{a} \to A \to A / \mathfrak{a} \to 0$
with $M$.]
  \item Let $A$ be a local ring, $M$ and $N$ finitely generated $A$-modules.
    Prove that if $M \otimes N=0$, then $M=0$ or $N=0$.
    
[Let $\mathfrak{m}$ be the maximal ideal, $k=A / \mathfrak{m}$ the residue
field. Let $M_{k}=k \otimes_{A} M \cong$ $M / \mathfrak{m} M$ by Exercise 2. By
Nakayama's lemma, $M_{k}=0 \implies M=0$. But
$M \otimes_{A} N=0 \to(M \otimes_{A} N)_{k}=0 \to M_{k} \otimes_{k} N_{k}=0 \to M_{k}=0$
or $N_{k}=0$, since $M_{k}, N_{k}$ are vector spaces over a field.]

  \item Let $M_{i}(i \in I)$ be any family of $A$-modules, and let $M$ be their
        direct sum. Prove that $M$ is flat $\Leftrightarrow$ each $M_{i}$ is
        flat.
  \item Let $A[x]$ be the ring of polynomials in one indeterminate over
    a ring $A$. Prove that $A[x]$ is a flat $A$-algebra.
    [Use Exercise 4.]
  \item For any $A$-module, let $M[x]$ denote the set of all polynomials in $x$
        with coefficients in $M$, that is to say expressions of the form
\[
  m_{0}+m_{1} x+\cdots+m_{r} x^{r} \quad(m_{i} \in M).
\]
Defining the product of an element of $A[x]$ and an element of $M[x]$ in the
obvious way, show that $M[x]$ is an $A[x]$-module.

Show that $M[x] \cong A[x] \otimes_{A} M$.
  \item Let $\mathfrak{p}$ be a prime ideal in $A$. Show that $\mathfrak{p}[x]$
        is a prime ideal in $A[x]$. If $\mathfrak{m}$ is a maximal ideal in $A$,
        is $\mathfrak{m}[x]$ a maximal ideal in $A[x]$ ?

  \item \begin{enumerate}[i)]
    \item If $M$ and $N$ are flat $A$-modules, then so is $M
      \otimes_{A} N$.
      \item If $B$ is a flat $A$-algebra and $N$ is a flat $B$-module, then $N$ is flat
        as an $A$-module.
      \end{enumerate}
      
  \item Let $0 \to M' \to M \to M'' \to 0$ be an exact sequence of $A$-modules.
      If $M'$ and $M''$ are finitely generated, then so is $M$.

  \item Let $A$ be a ring, $\mathfrak{a}$ an ideal contained in the Jacobson
        radical of $A$; let $M$ be an $A$-module and $N$ a finitely generated
        $A$-module, and let $u: M \to N$ be a homomorphism. If the induced
        homomorphism $M / \mathfrak{a} M \to N / \mathfrak{a} N$ is surjective,
        then $u$ is surjective.

  \item Let $A$ be a rings $\neq 0$. Show that
        $A^{m} \cong A^{n} \to \hat{m}=n$.

        [Let $\mathfrak{m}$ be a maximal ideal of $A$ and let $\phi: A^{m} \to A^{n}$
be an isomorphism. Then $1 \otimes \phi\colon (A / \mathfrak{m}) \otimes
A^{m} \to(A / \mathfrak{m}) \otimes A^{n}$ is an isomorphism between
vector spaces of dimensions $m$ and $n$ over the field $k=A /
\mathfrak{m}$. Hence $m=n$.] (Cf. Chapter 3, Exercise 15.)

If $\phi: A^{m} \to A^{n}$ is surjective, then $m \geqslant n$.

If $\phi: A^{m} \to A^{n}$ is injective, is it always the case that
$m \leqslant n$ ?
  \item Let $M$ be a finitely generated $A$-module and $\phi: M \to A^{n}$ a
        surjective homomorphism. Show that $\Ker(\phi)$ is finitely generated.

        [Let $e_{1}, \ldots, e_{n}$ be a basis of $A^{n}$ and choose $u_{i} \in M$ such
that $\phi(u_{i})=e_{i}$ $(1 \leqslant i \leqslant n)$. Show that $M$
is the direct sum of $\Ker(\phi)$ and the submodule generated by
$u_{1}, \ldots, u_{n}$.]

  \item Let $f\colon A \to B$ be a ring homomorphism, and let $N$ be a $B$-module.
        Regarding $N$ as an $A$-module by restriction of scalars, form the
        $B$-module $N_{B}=B \otimes_{A} N$. Show that the homomorphism
        $g: N \to N_{B}$ which maps $y$ to $1 \otimes y$ is injective and that
        $g(N)$ is a direct summand of $N_{B}$.

[Define $p: N_{B} \to N$ by $p(b \otimes y)=b y$, and show that
$N_{B}=\Img(g) \oplus \Ker(p)$.]
\end{enumerate}
\subsection*{Direct limits}
\begin{enumerate}[resume*=exc2]
  \item A partially ordered set $I$ is said to be a {\itshape directed set} if for each
        pair $i, j$ in $I$ there exists $k \in I$. such that $i \leq k$ and
        $j \leq k$.

Let $A$ be a ring, let $I$ be a directed set and let
$(M_{i})_{i \in I}$ be a family of $A$-modules indexed by $I$. For
each pair $i, j$ in $I$ such that $i \leq j$, let
$\mu_{i j}: M_{i} \to M_{j}$ be an $A$-homomorphism, and suppose that the
following axioms are satisfied:
\begin{enumerate}
\item $\mu_{ii}$ is the identity mapping of
$M_{i}$, for all $i \in I$;
\item $\mu_{i k}=\mu_{j k} \circ \mu_{i j}$ whenever $i \leq j \leq
  k$.
\end{enumerate}
Then the modules $M_{i}$ and homomorphisms $\mu_{i j}$ are said to
form a {\itshape direct
system} $\mathbf{M}=(M_{i}, \mu_{i j})$ over the directed
set $I$.

We shall construct an $A$-module $M$ called the {\itshape direct limit} of the direct
system $\mathbf{M}$. Let $C$ be the direct sum of the $M_{i}$, and identify each module
$M_{i}$ with its canonical image in $C$. Let $D$ be the submodule of $C$
generated by all elements of the form $x_{i}-\mu_{i j}(x_{i})$ where
$i \leq j$ and $x_{i} \in M_{i}$. Let $M=C / D$, let $\mu: C \to M$ be the
projection and let $\mu_{i}$ be the restriction of $\mu$ to $M_{i}$.

The module $M$, or more correctly the pair consisting of $M$ and the family of
homomorphisms $\mu_{i}: M_{i} \to M$, is called the {\itshape direct
  limit} of the direct 
system $\mathbf{M}$, and is written $\varinjlim M_{i}$. From the construction
it is clear that $\mu_{i}=\mu_{f} \circ \mu_{i j}$ whenever $i \leq j$.

  \item In the situation of Exercise 14, show that every element of $M$ can be
        written in the form $\mu_{i}(x_{i})$ for some $i \in I$ and
        some $x_{i} \in M_{i}$.
        
Show also that if $\mu_{i}(x_{i})=0$ then there exists
$j \geq i$ such that $\mu_{i j}(x_{i})=0$ in
$M_{j}$.

  \item Show that the direct limit is characterized (up to isomorphism) by the
        following property. Let $N$ be an $A$-module and for each $i \in I$ let
        $\alpha_{i}: M_{i} \to N$ be an $A$ module homomorphism such that
        $\alpha_{i}=\alpha_{j} \circ \mu_{ij}$ whenever $i \leq j$. Then
        there exists a unique homomorphism $\alpha: M \to N$ such that
        $\alpha_{i}=\alpha \circ \mu_{i}$ for all $i \in I$.

  \item Let $(M_{i})_{i \in I}$ be a family of submodules of an
        $A$-module, such that for each pair of indices $i, j$ in $I$ there
        exists $k \in I$ such that $M_{i}+M_{j} \subseteq M_{k}$. Define
        $i \leq j$ to mean $M_{i} \subseteq M_{i}$ and let
        $\mu_{i j}: M_{i} \to M_{j}$ be the embedding of $M_{i}$ in $M_{j}$.
        Show that
\[
  \varinjlim M_{i}=\sum M_{i}=\bigcup M_{i}.
\]
In particular, any $A$-module is the direct limit of its finitely generated
submoduies.
  \item Let
        $\mathbf{M}=(M_{i}, \mu_{i j}), \mathbf{N}=(N_{i}, v_{i j})$
        be direct systems of $A$-modules over the same directed set. Let $M, N$
        be the direct limits and $\mu_{i}: M_{i} \to M, \nu_{i}: N_{i} \to N$
        the associated homomorphisms.
        
A {\itshape homomorphism} $\symbf{\Phi}: \mathbf{M} \to \mathbf{N}$ is by definition a
family of $A$-module homomorphisms $\phi_{i}: M_{i} \to N_{i}$ such that
$\phi_{j} \circ \mu_{i j}=\nu_{i j} \circ \phi_{i}$ whenever $i \leq j$. Show
that $\symbf{\Phi}$ defines a unique homomorphism
$\phi=\varinjlim \phi_{i}: M \to N$ such that $\phi \circ
\mu_{i}=\nu_{i} \circ \phi_{i}$ for all $i \in I$. 
  \item A sequence of direct systems and homomorphisms
\[
  \mathbf{M} \to \mathbf{N} \to \mathbf{P}
\]
is {\itshape exact} if the corresponding sequence of modules and module homomorphisms is
exact for each $i \in I$. Show that the sequence $M \to N \to P$ of direct
limits is then exact. [Use Exercise 15.]
\end{enumerate}
\subsection*{Tensor products commute with direct limits}
\begin{enumerate}[resume*=exc2]
  \item Keeping the same notation as in Exercise 14 , let $N$ be any $A$-module.
        Then $(M_{i} \otimes N, \mu_{i j} \otimes 1)$ is a direct
        system; let $P=\varinjlim(M_{i} \otimes N)$ be its
        direct limit. For each $i \in I$ we have a homomorphism
        $\mu_{i} \otimes 1\colon M_{i} \otimes N \to M \otimes N$, hence by Exercise
        16 a homomorphism $\psi\colon P \to M \otimes N$. Show that $\psi$ is an
        isomorphism, so that
\[
  \varinjlim(M_{i} \otimes N) \cong(\varinjlim M_{i}) \otimes N .
\]
[For each $i \in I$, let $g_{i}: M_{i} \times N \to M_{i} \otimes N$ be the
canonical bilinear mapping. Passing to the limit we obtain a mapping
$g: M \times N \to P$. Show that $g$ is $A$-bilinear and hence define a
homomorphism $\phi: M \otimes N \to P$. Verify that $\phi \circ \psi$ and
$\psi \circ \phi$ are identity mappings.]
  \item Let $(A_{i})_{i\in I}$ be a family of rings indexed by
        a directed set $I$, and for each pair $i \le j$ in $I$ let
        $\alpha_{i j}: A_{i} \to A_{j}$ be a ring homomorphism, satisfying
        conditions (1) and (2) of Exercise 14. Regarding each $A_{i}$ as a
        $\mathbf{Z}$-module we can then form the direct limit of
        $A=\varinjlim A_{i}$. Show that $A$ inherits a ring
        structure from the $A_{i}$ so that the mappings
        $A_{i} \to A$ are ring homomorphisms. The ring $A$ is
        the {\itshape direct limit} of the system $(A_{i},
        \alpha_{i})$.
        
If $A=0$ prove that $A_{i}=0$ for some $i \in I$. [Remember that all rings have
identity elements!]
  \item Let $(A_{i}, \alpha_{i j})$ be a direct system of rings and
        let $\mathfrak{N}_{i}$ be the nilradical of $A_{i}$. Show that
        $\varinjlim\mathfrak{N}_{i}$ is the nilradical
        of $\varinjlim A_{i}$.
        
If each $A_{i}$ is an integral domain, then
$\varinjlim A_{i}$ is an integral domain.
  \item Let $(B_{\lambda})_{\lambda \leq \Lambda}$ be a family of
        $A$-algebras. For each finite subset of $\Lambda$ let $B_{J}$ denote the
        tensor product (over $A$ ) of the $B_{\lambda}$ for $\lambda \in J$. If
        $J'$ is another finite subset of $\Lambda$ and $J \subseteq J'$, there
        is a canonical $A$-algebra homomorphism $B_{J} \to B_{J'}$. Let $B$
        denote the direct limit of the rings $B_{J}$ as $J$ runs through all
        finite subsets of $\Lambda$. The ring $B$ has a natural $A$-algebra
        structure for which the homomorphisms $B_{J} \to B$ are $A$-algebra
        homomorphisms. The $A$-algebra $B$ is the {\itshape tensor
          product} of the family $(B_{\lambda})_{\lambda \in
          \Lambda}$. 
\end{enumerate}
\subsection*{Flatness and Tor}
In these Exercises it will be assumed that the reader is familiar with the
definition and basic properties of the $\Tor$ functor.
\begin{enumerate}[resume*=exc2]
  \item If $M$ is an $A$-module, the following are equivalent:
\begin{enumerate}
\item $M$ is flat;
\item $\Tor_{n}^{A}(M, N)=0$ for all $n>0$ and all $A$-modules $N$;
\item $\Tor_{1}^{A}(M, N)=0$ for all $A$-modules $N$.
\end{enumerate}
[To show that (i) $\implies$ (ii), take a free resolution of $N$ and tensor it with
$M$. Since $M$ is flat, the resulting sequence is exact and therefore its
homology groups, which are the $\Tor_{n}^{A}(M, N)$, are
zero for $n>0$. To show that (iii) $\to$ (i), let $0 \to N' \to N \to N'' \to 0$
be an exact sequence. Then, from the Tor exact sequence,
\[
  \Tor_{1}(M, N') \to M \otimes N' \to M \otimes N \to M \otimes N'' \to 0
\]
is exact. Since Tor $_{1}(M, N'')=0$ it follows that $M$ is flat.]
  \item Let $0 \to N' \to N \to N'' \to 0$ be an exact sequence, with $N''$
        flat. Then $N'$ is flat $\iff N$ is flat. [Use Exercise 24
        and the Tor exact sequence.]
  \item Let $N$ be an $A$-module. Then $N$ is
        flat $\iff\Tor_{1}(A / \mathfrak{a}, N)=0$ for all finitely
        generated ideals $\mathfrak{a}$ in $A$.

[Show first that $N$ is flat if $\Tor_{1}(M, N)=0$ for all
{\itshape finitely generated} $A$-modules $M$, by using
(\ref{prop:2.19}). If $M$ is finitely 
generated, let $x_{1}, \ldots, x_{n}$ be a set of generators of $M$, and let
$M_{i}$ be the submodule generated by $x_{1}, \ldots, x_{i}$. By considering the
successive quotients $M_{i} / M_{i-1}$ and using Exercise
25, deduce that $N$ is flat if $\Tor_{1}(M, N)=0$ for all {\itshape cyclic}
$A$-modules $M$, i.e., all $M$ generated by a single element, and therefore of
the form $A / \mathfrak{a}$ for some ideal $\mathfrak{a}$. Finally use
(\ref{prop:2.19}) again to reduce to the case where $\mathfrak{a}$ is a finitely
generated ideal.]

  \item A ring $A$ is absolutely flat if every $A$-module is flat. Prove that
    the following are equivalent:
    \begin{enumerate}[i)]
    \item $\boldsymbol{A}$ is absolutely flat.
    \item Every principal ideal is idempotent.
    \item Every finitely generated ideal is a direct summand of $A$.
    \end{enumerate}
[i) $\to$ ii). Let $x \in A$. Then $A /(x)$ is a flat $A$-module, hence in the
diagram
\begin{center}
  \begin{tikzcd}
    (x)\otimes A \arrow[d] \arrow[r, "\beta"] & (x)\otimes A/(x) \arrow[d, "\alpha"] \\
    A \arrow[r] & A/(x)
  \end{tikzcd}
\end{center}
the mapping $\alpha$ is injective. Hence $\Img(\beta)=0$, hence
$(x)=(x^{2})$. ii) $\to$ iii). Let $x \in A$. Then $x=a x^{2}$ for
some $a \in A$, hence $e=a x$ is idempotent and we have $(e)=(x)$. Now if $e, f$
are idempotents, then $(e, f)=(e+f-e f)$. Hence every finitely generated ideal
is principal, and generated by an idempotent $e$, hence is a direct summand
because $A=(e) \oplus(1-e)$. iii) $\to$ i). Use the criterion of Exercise 26.]

  \item A Boolean ring is absolutely flat. The ring of Chapter 1, Exercise 7 is
        absolutely fiat. Every homomorphic image of an absolutely flat ring is
        absolutely flat. If a local ring is absolutely flat, then it is a field.

        If $A$ is absolutely flat, every non-unit in $A$ is a zero-divisor.
      \end{enumerate}
