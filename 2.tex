\documentclass{standalone}
\usepackage{unicode-math}
\setmathfont{texgyrepagella-math.otf}[math-style=TeX]
\usepackage{fontspec}
\setmainfont{TeX Gyre Pagella}
\usepackage{amsthm}
\newtheorem{theorem}{Theorem}[chapter]
\newtheorem{proposition}[theorem]{Proposition}
\newtheorem{lemma}[theorem]{lemma}
\newtheorem*{example}{Example}
\theoremstyle{definition}
\newtheorem{definition}[theorem]{Definition}
\theoremstyle{remark}
\newtheorem*{remark}{Remark}
\usepackage[export]{adjustbox}\graphicspath{ {./images/} }
\begin{document}
One of the things which distinguishes the modern approach to Commutative Algebra is the greater emphasis on modules, rather than just on ideals. The extra "elbow-room" that this gives makes for greater clarity and simplicity. For instance, an ideal $a$ and its quotient ring $A / \mathfrak{a}$ are both examples of modules and so, to a certain extent, can be treated on an equal footing. In this chapter we give the definition and elementary properties of modules. We also give a brief treatment of tensor products, including a discussion of how they behave for exact sequences.

\section{MODULES AND MODULE HOMOMORPHISMS}
Let $A$ be a ring (commutative, as always). An $A$-module is an abelian group $M$ (written additively) on which $A$ acts linearly: more precisely, it is a pair $(M, \mu)$, where $M$ is an abelian group and $\mu$ is a mapping of $A \times M$ into $M$ such that, if we write $a x$ for $\mu(a, x)(a \in A, x \in M)$, the following axioms are satisfied:

\[
\begin{aligned}
a(x+y) & =a x+a y, \\
(a+b) x & =a x+b x, \\
(a b) x & =a(b x), \\
1 x & =x
\end{aligned}
\][

(a, b \textbackslash in A ; \textbackslash quad x, y \textbackslash in M) \text{.}
]

(Equivalently, $M$ is an abelian group together with a ring homomorphism $A \rightarrow E(M)$, where $E(M)$ is the ring of endomorphisms of the abelian group $M$.)

The notion of a module is a common generalization of several familiar concepts, as the following examples show:

Examples. 1) An ideal $a$ of $A$ is an $A$-module. In particular $A$ itself is an $A$-module.

\begin{enumerate}
  \setcounter{enumi}{1}
  \item If $A$ is a field $k$, then $A$-module $=k$-vector space.

  \item $A=\mathbf{Z}$, then Z-module = abelian group (define $n x$ to be $x+\cdots+x$ ).

  \item $A=k[x]$ where $k$ is a field; an $A$-module is a $k$-vector space with a linear transformation.

  \item $G=$ finite group, $A=k[G]=$ group-algebra of $G$ over the field $k$ (thus $A$ is not commutative, unless $G$ is). Then $A$-module $=k$-representation of $G$. Let $M, N$ be $A$-modules. A mapping $f: M \rightarrow N$ is an $A$-module homomorphism (or is A-linear) if

\end{enumerate}

\[
\begin{gathered}
f(x+y)=f(x)+f(y) \\
f(a x)=a \cdot f(x)
\end{gathered}
\]

for all $a \in A$ and all $x, y \in M$. Thus $f$ is a homomorphism of abelian groups which commutes with the action of each $a \in A$. If $A$ is a field, an $A$-module homomorphism is the same thing as a linear transformation of vector spaces.

The composition of $A$-module homomorphisms is again an $A$-module homomorphism.

The set of all $A$-module homomorphisms from $M$ to $N$ can be turned into an $A$-module as follows: we define $f+g$ and af by the rules

\[
\begin{aligned}
(f+g)(x) & =f(x)+g(x), \\
(a f)(x) & =a \cdot f(x)
\end{aligned}
\]

for all $x \in M$. It is a trivial matter to check that the axioms for an $A$-module are satisfied. This $A$-module is denoted by $\operatorname{Hom}_{A}(M, N)$ (or just $\operatorname{Hom}(M, N)$ if there is no ambiguity about the ring $A$ ).

Homomorphisms $u: M^{\prime} \rightarrow M$ and $v: N \rightarrow N^{\prime \prime}$ induce mappings

$\bar{u}: \operatorname{Hom}(M, N) \rightarrow \operatorname{Hom}\left(M^{\prime}, N\right)$ and $\bar{v}: \operatorname{Hom}(M, N) \rightarrow \operatorname{Hom}\left(M, N^{\prime \prime}\right)$

defined as follows:

\[
\bar{u}(f)=f \circ u, \quad v(f)=v \circ f .
\]

These mappings are $A$-module homomorphisms.

For any module $M$ there is a natural isomorphism $\operatorname{Hom}(A, M) \cong M$ : any $A$-module homomorphism $f: A \rightarrow M$ is uniquely determined by $f(i)$, which can be any element of $M$.

\section{SUBMODULES AND QUOTIENT MODULES}
A submodule $M^{\prime}$ of $M$ is a subgroup of $M$ which is closed under multiplication by elements of $A$. The abelian group $M / M^{\prime}$ then inherits an $A$-module structure from $M$, defined by $a\left(x+M^{\prime}\right)=a x+M^{\prime}$. The $A$-module $M / M^{\prime}$ is the quotient of $M$ by $M^{\prime}$. The natural map of $M$ onto $M / M^{\prime}$ is an $A$-module homomorphism. There is a one-to-one order-preserving correspondence between submodules of $M$ which contain $M^{\prime}$, and submodules of $M^{\prime \prime}$ (just as for ideals; the statement for ideals is a special case).

If $f: M \rightarrow N$ is an $A$-module homomorphism, the kernel of $f$ is the set

\[
\operatorname{Ker}(f)=\{x \in M: f(x)=0\}
\]

and is a submodule of $M$. The image of $f$ is the set

\[
\operatorname{Im}(f)=f(M)
\]

and is a submodule of $N$. The cokernel of $f$ is

\[
\text { Coker }(f)=N / \operatorname{Im}(f)
\]

which is a quotient module of $N$.

If $M^{\prime}$ is a submodule of $M$ such that $M^{\prime} \subseteq \operatorname{Ker}(f)$, then $f$ gives rise to a homomorphism $\bar{f}: M / M^{\prime} \rightarrow N$, defined as follows: if $\bar{x} \in M / M^{\prime}$ is the image of $x \in M$; then $\bar{f}(\bar{x})=f(x)$. The kernel of $\bar{f}$ is $\operatorname{Ker}(f) / M^{\prime}$. The homomorphism $f$ is said to be induced by $f$. In particular, taking $M^{\prime}=\operatorname{Ker}(f)$, we have an isomorphism of $A$-modules

\[
M / \operatorname{Ker}(f) \cong \operatorname{Im}(f) .
\]

\section{OPERATIONS ON SUBMODULES}
Most of the operations on ideals considered in Chapter 1 have their counterparts for modules. Let $M$ be an $A$-module and let $\left(M_{i}\right)_{t \in I}$ be a family of submodules of $M$. Their sum $\sum M_{i}$ is the set of all (finite) sums $\sum x_{i}$, where $x_{i} \in M_{i}$ for all $i \in I$, and almost all the $x_{i}$ (that is, all but a finite number) are zero. $\sum M_{i}$ is the smallest submodule of $M$ which contains all the $M_{i}$.

The intersection $\cap M_{i}$ is again a submodule of $M$. Thus the submodules of $M$ form a complete lattice with respect to inclusion.

Proposition 2.1. i) If $L \supseteq M \supseteq N$ are $A$-modules, then

\[
(L / N) /(M / N) \cong L / M .
\]

ii) If $M_{1}, M_{2}$ are submodules of $M$, then

\[
\left(M_{1}+M_{2}\right) / M_{1} \cong M_{2} /\left(M_{1} \cap M_{2}\right) .
\]

Proof. i) Define $\theta: L / N \rightarrow L / M$ by $\theta(x+N)=x+M$. Then $\theta$ is a welldefined $A$-module homomorphism of $L / N$ onto $L / M$, and its kernel is $M / N$; hence (i).

ii) The composite homomorphism $M_{2} \rightarrow M_{1}+M_{2} \rightarrow\left(M_{1}+M_{2}\right) / M_{1}$ is surjective, and its kernel is $M_{\mathrm{I}} \cap M_{2}$; hence (ii).

We cannot in general define the product of two submodules, but we can define the product $\mathfrak{a} M$, where $\mathfrak{a}$ is an ideal and $M$ an $A$-module; it is the set of all finite sums $\sum a_{i} x_{i}$ with $a_{i} \in \mathfrak{a}, x_{i} \in M$, and is a submodule of $M$.

If $N, P$ are submodules of $M$, we define $(N: P)$ to be the set of all $a \in A$ such that $a P \subseteq N$; it is an ideal of $A$. In particular, $(0: M)$ is the set of all $a \in A$ such that $a M=0$; this ideal is called the annihilator of $M$ and is also denoted by Ann $(M)$. If $\mathfrak{a} \subseteq \operatorname{Ann}(M)$, we may regard $M$ as an $A / \mathfrak{a}$-module, as follows: if $\bar{x} \in A / \mathfrak{a}$ is represented by $x \in A$, define $\bar{x} m$ to be $x m(m \in M)$ : this is independent of the choice of the representative $x$ of $\bar{x}$, since $\mathfrak{a} M=0$. An $A$-module is faithful if $\operatorname{Ann}(M)=0$. If $\operatorname{Ann}(M)=\mathfrak{a}$, then $M$ is faithful as an $A / \mathfrak{a}$-module.

Exercise 2.2. i) Ann $(M+N)=\operatorname{Ann}(M) \cap \operatorname{Ann}(N)$.

ii) $(N: P)=\operatorname{Ann}((N+P) / N)$.

If $x$ is an element of $M$, the set of all multiples $a x(a \in A)$ is a submodule of $M$, denoted by $A x$ or $(x)$. If $M=\sum_{i \in I} A x_{i}$, the $x_{i}$ are said to be a set of generators of $M$; this means that every element of $M$ can be expressed (not necessarily uniquely) as a finite linear combination of the $x_{t}$ with coefficients in $A$. An $A$-module $M$ is said to be finitely generated if it has a finite set of generators.

\section{DIRECT SUM AND PRODUCT}
If $M, N$ are $A$-modules, their direct sum $M \oplus N$ is the set of all pairs $(x, y)$ with $x \in M, y \in N$. This is an $A$-module if we define addition and scalar multiplication in the obvious way:

\[
\begin{aligned}
\left(x_{1}, y_{1}\right)+\left(x_{2}, y_{2}\right) & =\left(x_{1}+x_{2}, y_{1}+y_{2}\right) \\
a(x, y) & =(a x, a y) .
\end{aligned}
\]

More generally, if $\left(M_{i}\right)_{i \in l}$ is any family of $A$-modules, we can define their direct sum $\oplus_{i \in I} M_{i}$; its elements are families $\left(x_{i}\right)_{i \in I}$ such that $x_{i} \in M_{i}$ for each $i \in I$ and almost all $x_{i}$ are 0 . If we drop the restriction on the number of non-zero $x$ 's we have the direct product $\prod_{i \in I} M_{i}$. Direct sum and direct product are therefore the same if the index set $I$ is finite, but not otherwise, in general.

Suppose that the ring $A$ is'a direct product $\prod_{i=1}^{n} A_{t}$ (Chapter 1). Then the set of all elements of $A$ of the form

\[
\left(0, \ldots, 0, a_{i}, 0, \ldots, 0\right)
\]

with $a_{i} \in A_{i}$ is an ideal $a_{i}$ of $A$ (it is not a subring of $A$-except in trivial casesbecause it does not contain the identity element of $A$ ). The ring $A$, considered as an $A$-module, is the direct sum of the ideals $\mathfrak{a}_{1}, \ldots, \mathfrak{a}_{n}$. Conversely, given a module decomposition

\[
A=\mathfrak{a}_{1} \oplus \cdots \oplus \mathfrak{a}_{n}
\]

of $A$ as a direct sum of ideals, we have

\[
A \cong \prod_{i=1}^{n}\left(A / \mathfrak{b}_{i}\right)
\]

where $\mathfrak{b}_{i}=\bigoplus_{j \neq 1} \mathfrak{a}_{j}$. Each ideal $\mathfrak{a}_{i}$ is a ring (isomorphic to $A / \mathfrak{b}_{i}$ ). The identity element $e_{\mathfrak{i}}$ of $\mathfrak{a}_{\mathfrak{i}}$ is an idempotent in $A$, and $\mathfrak{a}_{\mathfrak{i}}=\left(e_{\mathfrak{i}}\right)$.

\section{FINITELY GENERATED MODULES}
A free $A$-module is one which is isomorphic to an $A$-module of the form $\bigoplus_{i \in I} M_{i}$, where each $M_{i} \cong A$ (as an $A$-module). The notation $A^{(n)}$ is sometimes used. A finitely generated free $A$-module is therefore isomorphic to $A \oplus \cdots \oplus A$ ( $n$ summands), which is denoted by $A^{n}$. (Conventionally, $A^{0}$ is the zero module, denoted by 0 .)

Proposition 2.3. $M$ is a finitely generated $A$-module $\Leftrightarrow M$ is isomorphic to a quotient of $A^{n}$ for some integer $n>0$.

Proof. $\Rightarrow$ : Let $x_{1}, \ldots, x_{n}$ generate $M$. Define $\phi: A^{n} \rightarrow M$ by $\phi\left(a_{1}, \ldots, a_{n}\right)=$ $a_{1} x_{1}+\cdots+a_{n} x_{n}$. Then $\phi$ is an $A$-module homomorphism onto $M$, and therefore $M \cong A^{n} / \operatorname{Ker}(\phi)$.

$\Leftarrow$ : We have an $A$-module homomorphism $\phi$ of $A^{n}$ onto $M$. If $e_{i}=$ $(0, \ldots, 0,1,0, \ldots, 0)$ (the 1 being in the $i$ th place), then the $e_{i}(1 \leqslant i \leqslant n)$ generate $A^{n}$, hence the $\phi\left(e_{i}\right)$ generate $M$.

Proposition 2.4. Let $M$ be a finitely generated A-module, let $a$ be an ideal of $A$, and let $\phi$ be an $A$-module endomorphism of $M$ such that $\phi(M) \subseteq \mathfrak{a} M$. Then $\phi$ satisfies an equation of the form

\[
\phi^{n}+a_{1} \phi^{n-1}+\cdots+a_{n}=0
\]

where the $a_{i}$ are in $a$.

Proof. Let $x_{1}, \ldots, x_{n}$ be a set of generators of $M$. Then each $\phi\left(x_{i}\right) \in \mathfrak{a} M$, so that we have say $\phi\left(x_{i}\right)=\sum_{j=1}^{n} a_{i j} x_{j}\left(1 \leqslant i \leqslant n ; a_{i j} \in \mathfrak{a}\right)$, i.e.,

\[
\sum_{j=1}^{n}\left(\delta_{i j} \phi-a_{i j}\right) x_{j}=0
\]

where $\delta_{i j}$ is the Kronecker delta. By multiplying on the left by the adjoint of the matrix $\left(\delta_{i j} \phi-a_{i j}\right)$ it follows that det $\left(\delta_{i j} \phi-a_{i j}\right)$ annihilates each $x_{i}$, hence is the zero endomorphism of $M$. Expanding out the determinant, we have an equation of the required form.

Corollary 2.5. Let $M$ be a finitely generated A-module and let $\mathfrak{a}$ be an ideal of $A$ such that $a M=M$. Then there exists $x \equiv 1(\bmod a)$ such that $x M=0$.

Proof. Take $\phi=$ identity, $x=1+a_{1}+\cdots+a_{n}$ in (2.4).

Proposition 2.6. (Nakayama's lemma). Let $M$ be a finitely generated $A$-module and $a$ an ideal of $A$ contained in the Jacobson radical $\Re$ of $A$. Then $\mathfrak{a} M=M$ implies $M=0$.

First Proof. By $(2.5)$ we have $x M=0$ for some $x \equiv 1(\bmod \Re)$. By $(1.9) x$ is a unit in $A$, hence $M=x^{-1} \times M=0$. Second Proof. Suppose $M \neq 0$, and let $u_{1}, \ldots, u_{n}$ be a minimal set of gener ators of $M$. Then $u_{n} \in \mathfrak{a} M$, hence we have an equation of the form $u_{n}=a_{1} u_{1}+$ $\cdots+a_{n} u_{n}$, with the $a_{i} \in \mathfrak{a}$. Hence

\[
\left(1-a_{n}\right) u_{n}=a_{1} u_{1}+\cdots+a_{n-1} u_{n-1}
\]

since $a_{n} \in \Re$, it follows from (1.9) that $1-a_{n}$ is a unit in $A$. Hence $u_{n}$ belongs tc the submodule of $M$ generated by $u_{1}, \ldots, u_{n-1}$ : contradiction.

Corollary 2.7. Let $M$ be a finitely generated $A$-module, $N$ a submodule of $M$ $\mathfrak{a} \subseteq \mathfrak{R}$ an ideal. Then $M=\mathfrak{a} M+N \Rightarrow M=N$.

Proof. Apply (2.6) to $M / N$, observing that $\mathfrak{a}(M / N)=(\mathfrak{a} M+N) / N$.

Let $A$ be a local ring, $\mathfrak{m}$ its maximal ideal, $k=A / \mathfrak{m}$ its residue field. Let $M$ be a finitely generated $A$-module. $M / \mathfrak{m} M$ is annihilated by $\mathfrak{m}$, hence is naturally an $A / \mathfrak{m}$-module, i.e., a $k$-vector space, and as such is finite-dimensional.

Proposition 2.8. Let $x_{1}(1 \leqslant i \leqslant n)$ be elements of $M$ whose images in $M / \mathfrak{m} M$ form a basis of this vector space. Then the $x_{i}$ generate $M$.

Proof. Let $N$ be the submodule of $M$ generated by the $x_{i}$. Then the composite map $N \rightarrow M \rightarrow M / \mathfrak{m} M$ maps $N$ onto $M / \mathfrak{m} M$, hence $N+\mathfrak{m} M=M$, hence $N=M$ by (2.7).

\section{EXACT SEQUENCES}
A sequence of $A$-modules and $A$-homomorphisms

\[
\cdots \longrightarrow M_{i-1} \stackrel{f_{i}}{\longrightarrow} M_{t} \stackrel{f_{i+1}}{\longrightarrow} M_{i+1} \longrightarrow \cdots
\]

is said to be exact at $M_{i}$ if $\operatorname{Im}\left(f_{i}\right)=\operatorname{Ker}\left(f_{i+1}\right)$. The sequence is exact if it is exact at each $M_{i}$. In particular:

$0 \rightarrow M^{\prime} \stackrel{f}{\rightarrow} M$ is exact $\Leftrightarrow f$ is injective;

$M \stackrel{g}{\rightarrow} M^{\prime \prime} \rightarrow 0$ is exact $\Leftrightarrow g$ is surjective;

$0 \rightarrow M^{\prime} \stackrel{f}{\rightarrow} M \stackrel{g}{\rightarrow} M^{\prime \prime} \rightarrow 0$ is exact $\Leftrightarrow f$ is injective, $g$ is surjective and $g$ induces an isomorphism of Coker $(f)=M / f\left(M^{\prime}\right)$ onto $M^{\prime \prime}$.

A sequence of type (3) is called a short exact sequence. Any long exact sequence $(0)$ can be split up into short exact sequences: if $N_{i}=\operatorname{Im}\left(f_{i}\right)=$ $\operatorname{Ker}\left(f_{i+1}\right)$, we have short exact sequences $0 \rightarrow N_{i} \rightarrow M_{i} \rightarrow N_{i+1} \rightarrow 0$ for each $i$.

Proposition 2.9. i) Let

\[
M^{\prime} \stackrel{\text { ㄴ }}{\rightarrow} M \stackrel{v}{\rightarrow} M^{\prime \prime} \rightarrow 0
\]

be a sequence of A-modules and homomorphisms. Then the sequence (4) is exact $\Leftrightarrow$ for all $A$-modules $N$, the sequence

\[
0 \rightarrow \operatorname{Hom}\left(M^{\prime \prime}, N\right) \stackrel{b}{\rightarrow} \operatorname{Hom}(M, N) \stackrel{a}{\rightarrow} \operatorname{Hom}\left(M^{\prime}, N\right)
\]

is exact.

\section{ii) Let}
\[
0 \rightarrow N^{\prime} \stackrel{u}{\rightarrow} N \stackrel{\bullet}{\rightarrow} N^{\prime \prime}
\]

be a sequence of A-modules and homomorphisms. Then the sequence (5) is exact $\Leftrightarrow$ for all $A$-modules $M$, the sequence

\[
0 \rightarrow \operatorname{Hom}\left(M, N^{\prime}\right) \stackrel{a}{\rightarrow} \operatorname{Hom}(M, N) \stackrel{b}{\rightarrow} \operatorname{Hom}\left(M, N^{\prime \prime}\right)
\]

is exact.

All four parts of this proposition are easy exercises. For example, suppose that ( $\left.4^{\prime}\right)$ is exact for all $N$. First of all, since $\bar{v}$ is injective for all $N$ it follows that $v$ is surjective. Next, we have $\bar{u} \circ \bar{v}=0$, that is $v \circ u \circ f=0$ for all $f: M^{\prime \prime} \rightarrow N$. Taking $N$ to be $M^{\prime \prime}$ and $f$ to be the identity mapping, it follows that $v \circ u=0$, hence $\operatorname{Im}(u) \subseteq \operatorname{Ker}(v)$. Next take $N=M / \operatorname{Im}(u)$ and let $\phi: M \rightarrow N$ be the projection. Then $\phi \in \operatorname{Ker}(\bar{u})$, hence there exists $\psi: M^{\prime \prime} \rightarrow N$ such that $\phi=\psi \circ v$. Consequently $\operatorname{Im}(u)=\operatorname{Ker}(\phi) \supseteq \operatorname{Ker}(v)$.

Proposition 2.10. Let

\[
\begin{aligned}
& 0 \rightarrow M^{\prime} \stackrel{u}{\rightarrow} M \stackrel{v}{\rightarrow} M^{\prime \prime} \rightarrow 0 \\
& \text { " } \downarrow \quad \downarrow \downarrow \quad \downarrow r^{\prime \prime} \\
& 0 \rightarrow N^{\prime} \underset{u^{\prime}}{\rightarrow} N_{\vec{v}} \rightarrow N^{\prime \prime} \rightarrow 0
\end{aligned}
\]

be a commutative diagram of A-modules and homomorphisms, with the rows exact. Then there exists an exact sequence

\[
\begin{aligned}
& 0 \rightarrow \operatorname{Ker}\left(f^{\prime}\right) \stackrel{a}{\rightarrow} \operatorname{Ker}(f) \stackrel{b}{\rightarrow} \operatorname{Ker}\left(f^{\prime \prime}\right) \stackrel{d}{\rightarrow} \\
& \text { Coker }\left(f^{\prime}\right) \stackrel{a^{\prime}}{\rightarrow} \text { Coker }(f) \stackrel{v p}{\rightarrow} \text { Coker }\left(f^{\prime \prime}\right) \rightarrow 0
\end{aligned}
\]

in which $\bar{u}, \bar{v}$ are restrictions of $u, v$, and $\bar{u}^{\prime}, \bar{v}^{\prime}$ are induced by $u^{\prime}, v^{\prime}$.

The boundary homomorphism $d$ is defined as follows: if $x^{*} \in \operatorname{Ker}\left(f^{\prime \prime}\right)$, we have $x^{\prime \prime}=v(x)$ for some $x \in M$, and $v^{\prime}(f(x))=f^{\prime \prime}(v(x))=0$, hence $f(x) \in \operatorname{Ker}\left(v^{\prime}\right)=$ Im $\left(u^{\prime}\right)$, so that $f(x)=u^{\prime}\left(y^{\prime}\right)$ for some $y^{\prime} \in N^{\prime}$. Then $d\left(x^{\prime \prime}\right)$ is defined to be the image of $y^{\prime}$ in Coker $\left(f^{\prime}\right)$. The verification that $d$ is well-defined, and that the sequence (6) is exact, is a straightforward exercise in diagram-chasing which we leave to the reader.

Remark. (2.10) is a special case of the exact homology sequence of homological algebra.

Let $C$ be a class of $A$-modules and let $\lambda$ be a function on $C$ with values in $\mathbf{Z}$ (or, more generally, with values in an abelian group $G$ ). The function $\lambda$ is additive if, for each short exact sequence (3) in which all the terms belong to $C$, we have $\lambda\left(\boldsymbol{M}^{\prime}\right)-\lambda(\boldsymbol{M})+\lambda\left(\boldsymbol{M}^{\prime \prime}\right)=0$.

Example. Let $A$ be a field $k$, and let $C$ be the class of all finite-dimensional $k$-vector spaces $V$. Then $V \mapsto \operatorname{dim} V$ is an additive function on $C$.

$2+$ I.C.A. Proposition 2.11. Let $0 \rightarrow M_{0} \rightarrow M_{1} \rightarrow \cdots \rightarrow M_{n} \rightarrow 0$ be an exact sequence of $A$-modules in which all the modules $M_{1}$ and the kernels of all the homomorphisms belong to $C$. Then for any additive function $\lambda$ on $C$ we have

\[
\sum_{i=0}^{n}(-1)^{i} \lambda\left(M_{i}\right)=0 .
\]

Proof. Split up the sequence into short exact sequences

\[
0 \rightarrow N_{i} \rightarrow M_{i} \rightarrow N_{i+1} \rightarrow 0
\]

$\left(N_{0}=N_{n+1}=0\right)$. Then we have $\lambda\left(M_{i}\right)=\lambda\left(N_{i}\right)+\lambda\left(N_{i+1}\right)$. Now take the alternating sum of the $\lambda\left(M_{i}\right)$, and everything cancels out.

\section{TENSOR PRODUCT OF MODULES}
Let $M, N, P$ be three $A$-modules. A mapping $f: M \times N \rightarrow P$ is said to be $A$-bilinear if for each $x \in M$ the mapping $y \mapsto f(x, y)$ of $N$ into $P$ is $A$-linear, and for each $y \in N$ the mapping $x \mapsto f(x, y)$ of $M$ into $P$ is $A$-linear.

We shall construct an $A$-module $T$, called the tensor product of $M$ and $N$, with the property that the $A$-bilinear mappings $M \times N \rightarrow P$ are in a natural one-to-one correspondence with the $A$-linear mappings $T \rightarrow P$, for all $A$ modules $P$. More precisely:

Proposition 2.12. Let $M, N$ be A-modules. Then there exists a pair $(T, g)$ consisting of an A-module $T$ and an A-bilinear mapping $g: M \times N \rightarrow T$, with the following property:

Given any A-module $P$ and any A-bilinear mapping $f: M \times N \rightarrow P$, there exists a unique $A$-linear mapping $f^{\prime}: T \rightarrow P$ such that $f=f^{\prime} \circ g$ (in other words, every bilinear function on $M \times N$ factors through $T$ ).

Moreover, if $(T, g)$ and $\left(T^{\prime}, g^{\prime}\right)$ are two pairs with this property, then there exists a unique isomorphism $j: T \rightarrow T^{\prime}$ such that $j \circ g=g^{\prime}$.

Proof. i) Uniqueness. Replacing $(P, f)$ by $\left(T^{\prime}, g^{\prime}\right)$ we get a unique $j: T \rightarrow T^{\prime}$ such that $g^{\prime}=j \circ g$. Interchanging the roles of $T$ and $T^{\prime}$, we get $j^{\prime}: T^{\prime} \rightarrow T$ such that $g=j^{\prime} \circ g^{\prime}$. Each of the compositions $j \circ j^{\prime}, j^{\prime} \circ j$ must be the identity, and therefore $j$ is an isomorphism.

ii) Existence. Let $C$ denote the free $A$-module $A^{(M \times N)}$. The elements of $C$ are formal linear combinations of elements of $M \times N$ with coefficients in $A$, i.e. they are expressions of the form $\sum_{i=1}^{n} a_{i} \cdot\left(x_{i}, y_{i}\right)\left(a_{i} \in A, x_{i} \in M, y_{i} \in N\right)$.

Let $D$ be the submodule of $C$ generated by all elements of $C$ of the following types:

\[
\begin{gathered}
\left(x+x^{\prime}, y\right)-(x, y)-\left(x^{\prime}, y\right) \\
\left(x, y+y^{\prime}\right)-(x, y)-\left(x, y^{\prime}\right) \\
(a x, y)-a \cdot(x, y) \\
(x, a y)-a \cdot(x, y) .
\end{gathered}
\]

Let $T=C / D$. For each basis element $(x, y)$ of $C$, let $x \otimes y$ denote its image in $T$. Then $T$ is generated by the elements of the form $x \otimes y$, and from our definitions we have

\[
\begin{gathered}
\left(x+x^{\prime}\right) \otimes y=x \otimes y+x^{\prime} \otimes y, x \otimes\left(y+y^{\prime}\right)=x \otimes y+x \otimes y^{\prime}, \\
(a x) \otimes y=x \otimes(a y)=a(x \otimes y)
\end{gathered}
\]

Equivalently, the mapping $g: M \times N \rightarrow T$ defined by $g(x, y)=x \otimes y$ is A-bilinear.

Any map $f$ of $M \times N$ into an $A$-module $P$ extends by linearity to an $A$ module homomorphism $\bar{f}: C \rightarrow P$. Suppose in particular that $f$ is $A$-bilinear. Then, from the definitions, $f$ vanishes on all the generators of $D$, hence on the whole of $D$, and therefore induces a well-defined $A$-homomorphism $f^{\prime}$ of $T=C / D$ into $P$ such that $f^{\prime}(x \otimes y)=f(x, y)$. The mapping $f^{\prime}$ is uniquely defined by this condition, and therefore the pair $(T, g)$ satisfy the conditions of the proposition.

Remarks. i) The module $T$ constructed above is called the tensor product of $M$ and $N$, and is denoted by $M \otimes_{A} N$, or just $M \otimes N$ if there is no ambiguity about the ring $A$. It is generated as an $A$-module by the "products" $x \otimes y$. If $\left(x_{i}\right)_{i \in I},\left(y_{j}\right)_{j \in J}$ are families of generators of $M, N$ respectively, then the elements $x_{i} \otimes y_{j}$ generate $M \otimes N$. In particular, if $M$ and $N$ are finitely generated, so is $M \otimes N$.

ii) The notation $x \otimes y$ is inherently ambiguous unless we specify the tensor product to which it belongs. Let $M^{\prime}, N^{\prime}$ be submodules of $M, N$ respectively, and let $x \in M^{\prime}$ and $y \in N^{\prime}$. Then it can happen that $x \otimes y$ as an element of $M \otimes N$ is zero whilst $x \otimes y$ as an element of $M^{\prime} \otimes N^{\prime}$ is non-zero. For example, take $A=\mathbf{Z}, M=\mathbf{Z}, N=\mathbf{Z} / 2 Z$, and let $M^{\prime}$ be the submodule $2 Z$ of $\mathbf{Z}$, whilst $N^{\prime}=N$. Let $x$ be the non-zero element of $N$ and consider $2 \otimes x$. As an element of $M \otimes N$, it is zero because $2 \otimes x=1 \otimes 2 x=1 \otimes 0=0$. But as an element of $M^{\prime} \otimes N^{\prime}$ it is non-zero. See the example after (2.18).

However, there is the following result:

Corollary 2.13. Let $x_{i} \in M, y_{i} \in N$ be such that $\sum x_{i} \otimes y_{i}=0$ in $M \otimes N$. Then there exist finitely generated submodules $M_{0}$ of $M$ and $N_{0}$ of $N$ such that $\sum x_{i} \otimes y_{i}=0$ in $M_{0} \otimes N_{0}$.

Proof. If $\sum x_{i} \otimes y_{i}=0$ in $M \otimes N$, then in the notation of the proof of (2.12) we have $\sum\left(x_{i}, y_{i}\right) \in D$, and therefore $\sum\left(x_{i}, y_{i}\right)$ is a finite sum of generators of $D$. Let $M_{0}$ be the submodule of $M$ generated by the $x_{i}$ and all the elements of $M$ which occur as first coordinates in these generators of $D$, and define $N_{0}$ similarly. Then $\sum x_{i} \otimes y_{i}=0$ as an element of $M_{0} \otimes N_{0}$.

iii) We shall never again need to use the construction of the tensor product given in (2.12), and the reader may safely forget it if he prefers. What is essential to keep in mind is the defining property of the tensor product. iv) Instead of starting with bilinear mappings we could have started with multilinear mappings $f: M_{1} \times \cdots \times M_{r} \rightarrow P$ defined in the same way (i.e., linear in each variable). Following through the proof of (2.12) we should end up with a "multi-tensor product" $T=M_{1} \otimes \cdots \otimes M_{r}$, generated by all products $x_{1} \otimes \cdots \otimes x_{r}\left(x_{i} \in M_{i}, 1 \leqslant i \leqslant r\right)$. The details may safely be left to the reader; the result corresponding to (2.12) is

Proposition 2.12*. Let $M_{1}, \ldots, M_{\mathrm{r}}$ be A-modules. Then there exists a pair $(T, g)$ consisting of an A-module $T$ and an A-multilinear mapping $g: M_{1} \times \cdots$ $\times M_{r} \rightarrow T$ with the following property:

Given any A-module $P$ and any A-multilinear mapping $f: M_{1} \times \cdots$ $\times M_{r} \rightarrow T$, there exists a unique A-homomorphism $f^{\prime}: T \rightarrow P$ such that $f^{\prime} \circ g=f$.

Moreover, if $(T, g)$ and $\left(T^{\prime}, g^{\prime}\right)$ are two pairs with this property, then there exists a unique isomorphism $j: T \rightarrow T^{\prime}$ such that $j \circ g=g^{\prime}$.

There are various so-called "canonical isomorphisms", some of which we state here:

Proposition 2.14. Let $M, N, P$ be A-modules. Then there exist unique isomorphisms

i) $M \otimes N \rightarrow N \otimes M$

ii) $(M \otimes N) \otimes P \rightarrow M \otimes(N \otimes P) \rightarrow M \otimes N \otimes P$

iii) $(M \oplus N) \otimes P \rightarrow(M \otimes P) \oplus(N \otimes P)$

iv) $A \otimes M \rightarrow M$

such that, respectively,

a) $x \otimes y \mapsto y \otimes x$

b) $(x \otimes y) \otimes z \mapsto x \otimes(y \otimes z) \mapsto x \otimes y \otimes z$

c) $(x, y) \otimes z \mapsto(x \otimes z, y \otimes z)$

d) $a \otimes x \mapsto a x$.

Proof. In each case the point is to show that the mappings so described are well defined. The technique is to construct suitable bilinear or multilinear mappings, and use the defining property (2.12) or $\left(2.12^{*}\right)$ to infer the existence of homomorphisms of tensor products. We shall prove half of ii) as an example of the method, and leave the rest to the reader.

We shall construct homomorphisms

\[
(M \otimes N) \otimes P \stackrel{f}{\rightarrow} M \otimes N \otimes P \stackrel{g}{\rightarrow}(M \otimes N) \otimes P
\]

such that $f((x \otimes y) \otimes z)=x \otimes y \otimes z$ and $g(x \otimes y \otimes z)=(x \otimes y) \otimes z$ for all $x \in M, y \in N, z \in P$.

To construct $f$, fix the element $z \in P$. The mapping $(x, y) \mapsto x \otimes y \otimes z$ $(x \in M, y \in N)$ is bilinear in $x$ and $y$ and therefore induces a homomorphism $f_{z}: M \otimes N \rightarrow M \otimes N \otimes P$ such that $f_{2}(x \otimes y)=x \otimes y \otimes z$. Next, consider the mapping $(t, z) \mapsto f_{z}(t)$ of $(M \otimes N) \times P$ into $M \otimes N \otimes P$. This is bilinear in $t$ and $z$ and therefore induces a homomorphism

\[
f:(M \otimes N) \otimes P \rightarrow M \otimes N \otimes P
\]

such that $f((x \otimes y) \otimes z)=x \otimes y \otimes z$.

To construct $g$, consider the mapping $(x, y, z) \mapsto(x \otimes y) \otimes z$ of $M \times N$ $\times \boldsymbol{P}$ into $(\boldsymbol{M} \otimes N) \otimes \boldsymbol{P}$. This is linear in each variable and therefore induces a homomorphism

\[
g: M \otimes N \otimes P \rightarrow(M \otimes N) \otimes P
\]

such that $g(x \otimes y \otimes z)=(x \otimes y) \otimes z$.

Clearly $f \circ g$ and $g \circ f$ are identity maps, hence $f$ and $g$ are isomorphisms.

Exercise 2.15. Let $A, B$ be rings, let $M$ be an. $A$-module, $P$ a $B$-module and $N$ an $(A, B)$-bimodule (that is, $N$ is simultaneously an $A$-module and $a$ B-module and the two structures are compatible in the sense that $a(x b)=(a x) b$ for all $a \in A$, $b \in B, x \in N)$. Then $M \otimes_{A} N$ is naturally a $B$-module, $N \otimes_{B} P$ an A-module, and we have

\[
\left(M \otimes_{A} N\right) \otimes_{B} P \cong M \otimes_{A}\left(N \otimes_{B} P\right) .
\]

Let $f: M \rightarrow M^{\prime}, g: N \rightarrow N^{\prime}$ be homomorphisms of $A$-modules. Define $h: M \times N \rightarrow M^{\prime} \otimes N^{\prime}$ by $h(x, y)=f(x) \otimes g(y)$. It is easily checked that $h$ is $A$-bilinear and therefore induces an $A$-module homomorphism

\[
f \otimes g: M \otimes N \rightarrow M^{\prime} \otimes N^{\prime}
\]

such that

\[
(f \otimes g)(x \otimes y)=f(x) \otimes g(y) \quad(x \in M, \quad y \in N) .
\]

Let $f^{\prime}: M^{\prime} \rightarrow M^{\prime \prime}$ and $g^{\prime}: N^{\prime} \rightarrow N^{\prime \prime}$ be homomorphisms of $A$-modules. Then clearly the homomorphisms $\left(f^{\prime} \circ f\right) \otimes\left(g^{\prime} \circ g\right)$ and $\left(f^{\prime} \otimes g^{\prime}\right) \circ(f \otimes g)$ agree on all elements of the form $x \otimes y$ in $M \otimes N$. Since these elements generate $M \otimes N$, it follows that

\[
\left(f^{\prime} \circ f\right) \otimes\left(g^{\prime} \circ g\right)=\left(f^{\prime} \otimes g^{\prime}\right) \circ(f \otimes g) .
\]

\section{RESTRICTION AND EXTENSION OF SCALARS}
Let $f: A \rightarrow B$ be a homomorphism of rings and let $N$ be a $B$-module. Then $N$ has an $A$-module structure defined as follows: if $a \in A$ and $x \in N$, then $a x$ is defined to be $f(a) x$. This $A$-module is said to be obtained from $N$ by restriction of scalars. In particular, $f$ defines in this way an $A$-module structure on $B$. Proposition 2.16. Suppose $N$ is finitely generated as a $B$-module and that $B$ is

finitely generated as an $A$-module. Then $N$ is finitely generated as an $A$-module. Proof. Let $y_{1}, \ldots, y_{n}$ generate $N$ over $B$, and let $x_{1}, \ldots, x_{m}$ generate $B$ as an $A$-module. Then the $m n$ products $x_{i} y_{j}$ generate $N$ over $A$.

Now let $M$ be an $A$-module. Since, as we have just seen, $B$ can be regarded as an $A$-module, we can form the $A$-module $M_{B}=B \otimes_{A} M$. In fact $M_{B}$ carries a $B$-module structure such that $b\left(b^{\prime} \otimes x\right)=b b^{\prime} \otimes x$ for all $b, b^{\prime} \in B$ and all $x \in M$. The $B$-module $M_{B}$ is said to be obtained from $M$ by extension of scalars.

Proposition 2.17. If $M$ is finitely generated as an A-module, then $M_{B}$ is finitely generated as a $B$-module.

Proof. If $x_{1}, \ldots, x_{m}$ generate $M$ over $A$, then the $1 \otimes x_{i}$ generate $M_{B}$ over $B$.

\section{EXACTNESS PROPERTIES OF THE TENSOR PRODUCT}
Let $f: M \times N \rightarrow P$ be an $A$-bilinear mapping. For each $x \in M$ the mapping $y \mapsto f(x, y)$ of $N$ into $P$ is $A$-linear, hence $f$ gives rise to a mapping $M \rightarrow$ Hom $(N, P)$ which is $A$-linear because $f$ is linear in the variable $x$. Conversely any $A$-homomorphism $\phi: M \rightarrow \operatorname{Hom}_{A}(N, P)$ defines a bilinear map, namely $(x, y) \mapsto \phi(x)(y)$. Hence the set $S$ of $A$-bilinear mappings $M \times N \rightarrow P$ is in natural one-to-one correspondence with $\operatorname{Hom}(M, \operatorname{Hom}(N, P))$. On the other hand $S$ is in one-to-one correspondence with $\operatorname{Hom}(M \otimes N, P)$, by the defining property of the tensor product. Hence we have a canonical isomorphism

\[
\operatorname{Hom}(M \otimes N, P) \cong \operatorname{Hom}(M, \operatorname{Hom}(N, P)) \text {. }
\]

Proposition 2.18. Let

\[
M^{\prime} \stackrel{\rho}{\rightarrow} M \stackrel{g}{\rightarrow} M^{\prime \prime} \rightarrow 0
\]

be an exact sequence of A-modules and homomorphisms, and let $N$ be any $A$-module. Then the sequence

\[
M^{\prime} \otimes N \stackrel{f \otimes 1}{\longrightarrow} M \otimes N \stackrel{g \otimes 1}{\longrightarrow} M^{\prime \prime} \otimes N \rightarrow 0
\]

(where 1 denotes the identity mapping on $N$ ) is exact.

Proof. Let $E$ denote the sequence (2), and let $E \otimes N$ denote the sequence (3). Let $P$ be any $A$-module. Since (2) is exact, the sequence $\operatorname{Hom}(E$, Hom $(N, P))$ is exact by (2.9); hence by (1) the sequence $\operatorname{Hom}(E \otimes N, P)$ is exact. By (2.9) again, it follows that $E \otimes N$ is exact.

Remarks. i) Let $T(M)=M \otimes N$ and let $U(P)=\operatorname{Hom}(N, P)$. Then (1) takes the form Hom $(T(M), P)=\operatorname{Hom}(M, U(P))$ for all $A$-modules $M$ and $P$. In the language of abstract nonsense, the functor $T$ is the left adjoint of $U$, and $U$ is the right adjoint of $T$. The proof of (2.18) shows that any functor which is a left adjoint is right exact. Likewise any functor which is a right adjoint is left exact. ii) It is not in general true that, if $M^{\prime} \rightarrow M \rightarrow M^{\prime \prime}$ is an exact sequence of $A$-modules and homomorphisms, the sequence $M^{\prime} \otimes N \rightarrow M \otimes N \rightarrow M^{\prime \prime} \otimes N$ obtained by tensoring with an arbitrary $A$-module $N$ is exact.

Example. Take $A=\mathbf{Z}$ and consider the exact sequence $0 \rightarrow \mathbf{Z} \stackrel{b}{\rightarrow} \mathbf{Z}$, where $\dot{-}(x)=2 x$ for all $x \in \mathbf{Z}$. If we tensor with $N=\mathbf{Z} / 2 \mathbf{Z}$, the sequence $0 \rightarrow \mathbf{Z} \otimes N$ $\stackrel{r \otimes 1}{\longrightarrow} \mathbf{Z} \otimes N$ is not exact, because for any $x \otimes y \in \mathbf{Z} \otimes N$ we have

\[
(f \otimes 1)(x \otimes y)=2 x \otimes y=x \otimes 2 y=x \otimes 0=0,
\]

so that $f \otimes 1$ is the zero mapping, whereas $\mathrm{Z} \otimes N \neq 0$.

The functor $T_{N}: M \mapsto M \otimes_{A} N$ on the category of $A$-modules and homomorphisms is therefore not in general exact. If $T_{N}$ is exact, that is to say if tensoring with $N$ transforms all exact sequences into exact sequences, then $N$ is said to be a flat $A$-module.

Proposition 2.19. The following are equivalent, for an A-module $N$ :

i) $N$ is flat.

ii) If $0 \rightarrow M^{\prime} \rightarrow M \rightarrow M^{\prime \prime} \rightarrow 0$ is any exact sequence of $A$-modules, the tensored sequence $0 \rightarrow M^{\prime} \otimes N \rightarrow M \otimes N \rightarrow M^{\prime \prime} \otimes N \rightarrow 0$ is exact.

iii) If $f: M^{\prime} \rightarrow M$ is injective, then $f \otimes 1: M^{\prime} \otimes N \rightarrow M \otimes N$ is injective.

iv) If $f: M^{\prime} \rightarrow M$ is injective and $M, M^{\prime}$ are finitely generated, then

$f \otimes 1: M^{\prime} \otimes N \rightarrow M \otimes N$ is injective.

Proof. i) $\Leftrightarrow$ ii) by splitting up a long exact sequence into short exact sequences.

ii) $\Leftrightarrow$ iii) by $(2.18)$.

iii) $\Rightarrow$ iv): clear.

iv) $\Rightarrow$ iii). Let $f: M^{\prime} \rightarrow M$ be injective and let $u=\sum x_{i} \otimes y_{i} \in \operatorname{Ker}(f \otimes 1)$, so that $\sum f\left(x_{i}^{\prime}\right) \otimes y_{t}=0$ in $M \otimes N$. Let $M_{0}^{\prime}$ be the submodule of $M^{\prime}$ generated by the $x_{i}^{\prime}$ and let $u_{0}$ denote $\sum x_{i}^{\prime} \otimes y_{i}$ as an element of $M_{0}^{\prime} \otimes N$. By (2.14) there exists a finitely generated submodule $M_{0}$ of $M$ containing $f\left(M_{0}^{\prime}\right)$ and such that $\sum f\left(x_{i}^{\prime}\right) \otimes y_{i}=0$ as an element of $M_{0} \otimes N$. If $f_{0}: M_{0}^{\prime} \rightarrow M_{0}$ is the restriction of $f$, this means that $\left(f_{0} \otimes 1\right)\left(u_{0}\right)=0$. Since $M_{0}$ and $M_{0}^{\prime}$ are finitely generated, $f_{0} \otimes 1$ is injective and therefore $u_{0}=0$, hence $u=0$.

Exercise 2.20. If $f: A \rightarrow B$ is a ring homomorphism and $M$ is a flat $A$-module, then $M_{B}=B \otimes_{A} M$ is a flat $B$-module. (Use the canonical isomorphisms (2.14), (2.15).)

\section{ALGEBRAS}
Let $f: A \rightarrow B$ be a ring homomorphism. If $a \in A$ and $b \in B$, define a product

\[
a b=f(a) b .
\]

This definition of scalar multiplication makes the ring $B$ into an $A$-module (it is a particular example of restriction of scalars). Thus $B$ has an $A$-module structure as well as a ring structure, and these two structures are compatible in a sense which the reader will be able to formulate for himself. The ring $B$, equipped with this $A$-module structure, is said to be an $A$-algebra. Thus an $A$-algebra is, by definition, a ring $B$ together with a ring homomorphism $f: A \rightarrow B$.

Remarks. i) In particular, if $A$ is a field $K$ (and $B \neq 0$ ) then $f$ is injective by (1.2) and therefore $K$ can be canonically identified with its image in $B$. Thus a $K$-algebra ( $K$ a field) is effectively a ring containing $K$ as a subring.

ii) Let $A$ be any ring. Since $A$ has an identity element there is a unique homomorphism of the ring of integers $\mathrm{Z}$ into $A$, namely $n \mapsto n .1$. Thus every ring is automatically a $\mathbf{Z}$-algebra.

Let $f: A \rightarrow B, g: A \rightarrow C$ be two ring homomorphisms. An $A$-algebra homomorphism $h: B \rightarrow C$ is a ring homomorphism which is also an $A$-module homomorphism. The reader should verify that $h$ is an $A$-algebra homomorphism if and only if $h \circ f=g$.

A ring homomorphism $f: A \rightarrow B$ is finite, and $B$ is a finite $A$-algebra, if $B$ is finitely generated as an $A$-module. The homomorphism $f$ is of finite type, and $B$ is a finitely-generated $A$-algebra, if there exists a finite set of elements $x_{1}, \ldots x_{n}$ in $B$ such that every element of $B$ can be written as a polynomial in $x_{1}, \ldots, x_{n}$ with coefficients in $f(A)$; or equivalently if there is an $A$-algebra homomorphism from a polynomial ring $A\left[t_{1}, \ldots, t_{n}\right]$ onto $B$.

A ring $A$ is said to be finitely generated if it is finitely generated as a $\mathbf{Z}$ algebra. This means that there exist finitely many elements $x_{1}, \ldots, x_{n}$ in $A$ such that every element of $A$ can be written as a polynomial in the $x_{i}$ with rational integer coefficients.

\section{TENSOR PRODUCT OF ALGEBRAS}
Let $B, C$ be two $A$-algebras, $f: A \rightarrow B, g: A \rightarrow C$ the corresponding homomorphisms. Since $B$ and $C$ are $A$-modules we may form their tensor product $D=B \otimes_{A} C$, which is an $A$-module. We shall now define a multiplication on $D$.

Consider the mapping $B \times C \times B \times C \rightarrow D$ defined by

\[
\left(b, c, b^{\prime}, c^{\prime}\right) \mapsto b b^{\prime} \otimes c c^{\prime} .
\]

This is $A$-linear in each factor and therefore, by $\left(2.12^{*}\right)$, induces an $A$-module homomorphism

\[
B \otimes C \otimes B \otimes C \rightarrow D,
\]

hence by (2.14) an $A$-module homomorphism

\[
D \otimes D \rightarrow D
\]

and this in turn by (2.11) corresponds to an $A$-bilinear mapping

\[
\mu: D \times D \rightarrow D
\]

which is such that

\[
\mu\left(b \otimes c, b^{\prime} \otimes c^{\prime}\right)=b b^{\prime} \otimes c c^{\prime} .
\]

Of course, we could have written down this formula directly, but without some such argument as we have given there would be no guarantee that $\mu$ was welldefined.

We have therefore defined a multiplication on the tensor product $D=$ $B \otimes_{A} C$ : for elements of the form $b \otimes{ }^{\circ} c$ it is given by

\[
(b \otimes c)\left(b^{\prime} \otimes c^{\prime}\right)=b b^{\prime} \otimes c c^{\prime},
\]

and in general by

\[
\left(\sum_{i}\left(b_{i} \otimes c_{i}\right)\right)\left(\sum_{j}\left(b_{j}^{\prime} \otimes c_{j}^{\prime}\right)\right)=\sum_{i, j}\left(b_{i} b_{j}^{\prime} \otimes c_{i} c_{j}^{\prime}\right) .
\]

The reader should check that with this multiplication $D$ is a commutative ring, with identity element $1 \otimes 1$. Furthermore, $D$ is an $A$-algebra: the mapping $a \mapsto f(a) \otimes g(a)$ is a ring homomorphism $A \rightarrow D$.

In fact there is a commutative diagram of ring homomorphisms

\begin{center}
\begin{tikzcd}
                                   & B \arrow[rd, "u"]  &   \\
A \arrow[ru, "f"] \arrow[rd, "g"'] &                    & D \\
                                   & C \arrow[ru, "v"'] &
\end{tikzcd}
\end{center}

in which $u$, for example, is defined by $u(b)=b \otimes 1$.

\section{EXERCISES}
\begin{enumerate}
  \item Show that $(Z / m Z) \otimes_{\mathbf{Z}}(Z / n Z)=0$ if $m, n$ are coprime.

  \item Let $A$ be a ring, $\mathfrak{a}$ an ideal, $M$ an $A$-module. Show that $(A / \mathfrak{a}) \otimes_{A} M$ is isomorphic to $M / \mathfrak{a} M$.

\end{enumerate}

[Tensor the exact sequence $0 \rightarrow \mathfrak{a} \rightarrow A \rightarrow A / \mathfrak{a} \rightarrow 0$ with $M$.]

\begin{enumerate}
  \setcounter{enumi}{2}
  \item Let $A$ be a local ring, $M$ and $N$ finitely generated $A$-modules. Prove that if $M \otimes N=0$, then $M=0$ or $N=0$.
\end{enumerate}

[Let $\mathfrak{m}$ be the maximal ideal, $k=A / \mathfrak{m}$ the residue field. Let $M_{k}=k \otimes_{A} M \cong$ $M / \mathfrak{m} M$ by Exercise 2. By Nakayama's lemma, $M_{k}=0 \Rightarrow M=0$. But $M \otimes_{A} N=0 \Rightarrow\left(M \otimes_{A} N\right)_{k}=0 \Rightarrow M_{k} \otimes_{k} N_{k}=0 \Rightarrow M_{k}=0$ or $N_{k}=0$, since $M_{k}, N_{k}$ are vector spaces over a field.]

\begin{enumerate}
  \setcounter{enumi}{3}
  \item Let $M_{i}(i \in I)$ be any family of $A$-modules, and let $M$ be their direct sum. Prove that $M$ is flat $\Leftrightarrow$ each $M_{i}$ is flat. 5. Let $A[x]$ be the ring of polynomials in one indeterminate over a ring $A$. Prove that $A[x]$ is a flat $A$-algebra. [Use Exercise 4.]

  \item For any $A$-module, let $M[x]$ denote the set of all polynomials in $x$ with coefficients in $M$, that is to say expressions of the form

\end{enumerate}

\[
m_{0}+m_{1} x+\cdots+m_{r} x^{r} \quad\left(m_{i} \in M\right) \text {. }
\]

Defining the product of an element of $A[x]$ and an element of $M[x]$ in the obvious way, show that $M[x]$ is an $A[x]$-module.

Show that $M[x] \cong A[x] \otimes_{A} M$.

\begin{enumerate}
  \setcounter{enumi}{6}
  \item Let $\mathfrak{p}$ be a prime ideal in $A$. Show that $\mathfrak{p}[x]$ is a prime ideal in $A[x]$. If $\mathfrak{m}$ is a maximal ideal in $A$, is $\mathfrak{m}[x]$ a maximal ideal in $A[x]$ ?

  \item i) If $M$ and $N$ are flat $A$-modules, then so is $M \otimes_{A} N$.

\end{enumerate}

ii) If $B$ is a flat $A$-algebra and $N$ is a flat $B$-module, then $N$ is flat as an $A$-module.

\begin{enumerate}
  \setcounter{enumi}{8}
  \item Let $0 \rightarrow M^{\prime} \rightarrow M \rightarrow M^{\prime \prime} \rightarrow 0$ be an exact sequence of $A$-modules. If $M^{\prime}$ and $M^{\prime \prime}$ are finitely generated, then so is $M$.

  \item Let $A$ be a ring, $\mathfrak{a}$ an ideal contained in the Jacobson radical of $A$; let $M$ be an $A$-module and $N$ a finitely generated $A$-module, and let $u: M \rightarrow N$ be a homomorphism. If the induced homomorphism $M / \mathfrak{a} M \rightarrow N / \mathfrak{a} N$ is surjective, then $u$ is surjective.

  \item Lêt $A$ be à rings $\neq 0$. Shōw thât $A^{m} \cong A^{n} \Rightarrow \hat{m}=n$.

\end{enumerate}

[Let $\mathfrak{t t}$ be a maximal ideal of $A$ and let $\phi: A^{m} \rightarrow A^{n}$ be an isomorphism. Then $1 \otimes \phi:(A / \mathfrak{m}) \otimes A^{m} \rightarrow(A / \mathfrak{m}) \otimes A^{n}$ is an isomorphism between vector spaces of dimensions $m$ and $n$ over the field $k=A / m$. Hence $m=n$.] (Cf. Chapter 3, Exercise 15.)

If $\phi: A^{m} \rightarrow A^{n}$ is surjective, then $m \geqslant n$.

If $\phi: A^{m} \rightarrow A^{n}$ is injective, is it always the case that $m \leqslant n$ ?

\begin{enumerate}
  \setcounter{enumi}{11}
  \item Let $M$ be a finitely generated $A$-module and $\phi: M \rightarrow A^{n}$ a surjective homomorphism. Show that $\operatorname{Ker}(\phi)$ is finitely generated.
\end{enumerate}

[Let $e_{1}, \ldots, e_{n}$ be a basis of $A^{n}$ and choose $u_{i} \in M$ such that $\phi\left(u_{i}\right)=e_{i}$ $(1 \leqslant i \leqslant n)$. Show that $M$ is the direct sum of $\operatorname{Ker}(\phi)$ and the submodule generated by $u_{1}, \ldots, u_{n}$.]

\begin{enumerate}
  \setcounter{enumi}{12}
  \item Let $f: A \rightarrow B$ be a ring homomorphism, and let $N$ be a $B$-module. Regarding $N$ as an $A$-module by restriction of scalars, form the $B$-module $N_{B}=B \otimes_{A} N$. Show that the homomorphism $g: N \rightarrow N_{B}$ which maps $y$ to $1 \otimes y$ is injective and that $g(N)$ is a direct summand of $N_{B}$.
\end{enumerate}

[Define $p: N_{B} \rightarrow N$ by $p(b \otimes y)=b y$, and show that $N_{B}=\operatorname{Im}(g) \oplus \operatorname{Ker}(p)$.]

\section{Direct limits}
\begin{enumerate}
  \setcounter{enumi}{13}
  \item A partially ordered set $I$ is said to be a directed set if for each pair $i, j$ in $I$ there exists $k \in I$. such that $i \leqslant k$ and $j \leqslant k$.
\end{enumerate}

Let $A$ be a ring, let $I$ be a directed set and let $\left(M_{i}\right)_{i \in I}$ be a family of $A$-modules indexed by $I$. For each pair $i, j$ in $I$ such that $i \leqslant j$, let $\mu_{i j}: M_{i} \rightarrow M_{j}$ be an $A$-homomorphism, and suppose that the following axioms are satisfied: (1) $\mu_{13}$ is the identity mapping of $M_{i}$, for all $i \in I$;

(2) $\mu_{i k}=\mu_{j k} \circ \mu_{i j}$ whenever $i \leqslant j \leqslant k$.

Then the modules $M_{i}$ and homomorphisms $\mu_{i j}$ are said to form a direct system $\mathbf{M}=\left(\boldsymbol{M}_{i}, \mu_{i j}\right)$ over the directed set $I$.

We shall construct an $A$-module $M$ called the direct limit of the direct system $M$. Let $C$ be the direct sum of the $M_{i}$, and identify each module $M_{i}$ with its canonical image in $C$. Let $D$ be the submodule of $C$ generated by all elements of the form $x_{i}-\mu_{i j}\left(x_{i}\right)$ where $i \leqslant j$ and $x_{i} \in M_{i}$. Let $M=C / D$, let $\mu: C \rightarrow M$ be the projection and let $\mu_{i}$ be the restriction of $\mu$ to $M_{i}$.

The module $M$, or more correctly the pair consisting of $M$ and the family of homomorphisms $\mu_{i}: M_{i} \rightarrow M$, is called the direct limit of the direct system $\mathbf{M}$, and is written $\lim _{\rightarrow} M_{i}$. From the construction it is clear that $\mu_{i}=\mu_{f} \circ \mu_{i g}$ whenever $i \leqslant j$.

\begin{enumerate}
  \setcounter{enumi}{14}
  \item In the situation of Exercise 14, show that every element of $M$ can be written in the form $\mu_{i}\left(x_{i}\right)$ for some $i \in I$ and some $x_{i} \in M_{i}$.
\end{enumerate}

Show also that if $\mu_{i}\left(x_{i}\right)=0$ then there exists $j \geqslant i$ such that $\mu_{i j}\left(x_{i}\right)=0$ in $\boldsymbol{M}_{j}$.

\begin{enumerate}
  \setcounter{enumi}{15}
  \item Show that the direct limit is characterized (up to isomorphism) by the following property. Let $N$ be an $A$-module and for each $i \in I$ let $\alpha_{i}: M_{i} \rightarrow N$ be an $A$ module homomorphism such that $\alpha_{i}=\alpha_{j} \circ \mu_{i}$ whenever $i \leqslant j$. Then there exists a unique homomorphism $\alpha: M \rightarrow N$ such that $\alpha_{i}=\alpha \circ \mu_{i}$ for all $i \in I$.

  \item Let $\left(M_{1}\right)_{t \in l}$ be a family of submodules of an $A$-module, such that for each pair of indices $i, j$ in $I$ there exists $k \in I$ such that $M_{i}+M_{1} \subseteq M_{k}$. Define $i \leqslant j$ to mean $M_{i} \subseteq M_{i}$ and let $\mu_{i j}: M_{i} \rightarrow M_{j}$ be the embedding of $M_{i}$ in $M_{j}$. Show that

\end{enumerate}

\[
\stackrel{\lim }{\rightarrow} M_{\mathbf{i}}=\sum M_{\mathrm{i}}=\bigcup M_{\mathrm{i}} \text {. }
\]

In particular, any $A$-module is the direct limit of its finitely generated submoduies.

\begin{enumerate}
  \setcounter{enumi}{17}
  \item Let $\mathbf{M}=\left(M_{i}, \mu_{i j}\right), \mathbf{N}=\left(N_{i}, v_{i j}\right)$ be direct systems of $A$-modules over the same directed set. Let $M, N$ be the direct limits and $\mu_{i}: M_{i} \rightarrow M, \nu_{t}: N_{i} \rightarrow N$ the associated homomorphisms.
\end{enumerate}

A homomorphism $\boldsymbol{\phi}: \mathbf{M} \rightarrow \mathbf{N}$ is by definition a family of $A$-module homomorphisms $\phi_{i}: M_{i} \rightarrow N_{i}$ such that $\phi_{1} \circ \mu_{i j}=v_{i j} \circ \phi_{1}$ whenever $i \leqslant j$. Show that $\phi$ defines a unique homomorphism $\phi=\underline{\lim } \phi_{i}: M \rightarrow N$ such that $\phi \circ \mu_{i}=$ $\nu_{i} \circ \phi_{i}$ for all $i \in I$.

\begin{enumerate}
  \setcounter{enumi}{18}
  \item A sequence of direct systems and homomorphisms
\end{enumerate}

\[
\mathbf{M} \rightarrow \mathbf{N} \rightarrow \mathbf{P}
\]

is exact if the corresponding sequence of modules and module homomorphisms is exact for each $i \in I$. Show that the sequence $M \quad N \rightarrow P$ of direct limits is then exact. [Use Exercise 15.]

Tensor products commute with direct limits

\begin{enumerate}
  \setcounter{enumi}{19}
  \item Keeping the same notation as in Exercise 14 , let $N$ be any $A$-module. Then $\left(M_{i} \otimes N, \mu_{i j} \otimes 1\right)$ is a direct system; let $P=\underline{\lim }\left(M_{i} \otimes N\right)$ be its direct limit. For each $i \in I$ we have a homomorphism $\mu_{i} \otimes 1: M_{1} \otimes N \rightarrow M \otimes N$, hence by Exercise 16 a homomorphism $\psi: P \rightarrow M \otimes N$. Show that $\psi$ is an isomorphism, so that
\end{enumerate}

\[
\stackrel{\lim }{\longrightarrow}\left(M_{i} \otimes N\right) \cong\left(\lim _{\longrightarrow} M_{i}\right) \otimes N .
\]

[For each $i \in I$, let $g_{i}: M_{i} \times N \rightarrow M_{i} \otimes N$ be the canonical bilinear mapping. Passing to the limit we obtain a mapping $g: M \times N \rightarrow P$. Show that $g$ is $A$-bilinear and hence define a homomorphism $\phi: M \otimes N \rightarrow P$. Verify that $\phi \circ \psi$ and $\psi \circ \phi$ are identity mappings.]

\begin{enumerate}
  \setcounter{enumi}{20}
  \item Let $\left(A_{i}\right)_{\text {iel }}$ be a family of rings indexed by a directed set $I$, and for each pair $i \leqslant j$ in $I$ let $\alpha_{i j}: A_{t} \rightarrow A_{j}$ be a ring homomorphism, satisfying conditions (1) and (2) of Exercise 14. Regarding each $A_{1}$ as a $Z$-module we can then form the direct $\operatorname{limit} A=\lim A_{i}$. Show that $A$ inherits a ring structure from the $A_{1}$ so that the mappings $\overrightarrow{A_{i}} \rightarrow A$ are ring homomorphisms. The ring $A$ is the direct limit of the system $\left(A_{i}, \alpha_{i}\right)$.
\end{enumerate}

If $A=0$ prove that $A_{i}=0$ for some $i \in I$. [Remember that all rings have identity elements!]

\begin{enumerate}
  \setcounter{enumi}{21}
  \item Let $\left(A_{i}, \alpha_{i j}\right)$ be a direct system of rings and let $\mathfrak{R}_{i}$ be the nilradical of $A_{i}$. Show that $\underset{\lim }{\longrightarrow} \mathfrak{N}_{t}$ is the nilradical of $\lim _{\longrightarrow} A_{4}$.
\end{enumerate}

If each $A_{\mathfrak{i}}$ is an integral domain, then $\lim _{\longrightarrow} A_{\mathfrak{i}}$ is an integral domain.

\begin{enumerate}
  \setcounter{enumi}{22}
  \item Let $\left(B_{\lambda}\right)_{\lambda \leq \Lambda}$ be a family of $A$-algebras. For each finite subset of $\Lambda$ let $B_{y}$ denote the tensor product (over $A$ ) of the $B_{\lambda}$ for $\lambda \in J$. If $J^{\prime}$ is another finite subset of $\Lambda$ and $J \subseteq J^{\prime}$, there is a canonical $A$-algebra homomorphism $B_{J} \rightarrow B_{J}$. Let $B$ denote the direct limit of the rings $B_{j}$ as $J$ runs through all finite subsets of $\Lambda$. The ring $B$ has a natural $A$-algebra structure for which the homomorphisms $B_{j} \rightarrow B$ are $A$-algebra homomorphisms. The $A$-algebra $B$ is the tensor product of the family $\left(B_{\lambda}\right)_{\lambda \in \Lambda}$.
\end{enumerate}

\section{Flatness and Tor}
In these Exercises it will be assumed that the reader is familiar with the definition and basic properties of the Tor functor.

\begin{enumerate}
  \setcounter{enumi}{23}
  \item If $M$ is an $A$-module, the following are equivalent:
\end{enumerate}

i) $M$ is flat;

ii) $\operatorname{Tor}_{n}^{A}(M, N)=0$ for all $n>0$ and all $A$-modules $N$;

iii) $\operatorname{Tor}_{1}^{A}(M, N)=0$ for all $A$-modules $N$.

[To show that (i) $\Rightarrow$ (ii), take a free resolution of $N$ and tensor it with $M$. Since $M$ is flat, the resulting sequence is exact and therefore its homology groups, which are the $\operatorname{Tor}_{n}^{\Lambda}(M, N)$, are zero for $n>0$. To show that (iii) $\Rightarrow$ (i), let $0 \rightarrow N^{\prime} \rightarrow N \rightarrow N^{\prime \prime} \rightarrow 0$ be an exact sequence. Then, from the Tor exact sequence,

\[
\operatorname{Tor}_{1}\left(M, N^{\prime}\right) \rightarrow M \otimes N^{\prime} \rightarrow M \otimes N \rightarrow M \otimes N^{\prime \prime} \rightarrow 0
\]

is exact. Since Tor $_{1}\left(M, N^{\prime}\right)=0$ it follows that $M$ is flat.]

\begin{enumerate}
  \setcounter{enumi}{24}
  \item Let $0 \rightarrow N^{\prime} \rightarrow N \rightarrow N^{\prime \prime} \rightarrow 0$ be an exact sequence, with $N^{\prime \prime}$ flat. Then $N^{\prime}$ is flat $\Leftrightarrow N$ is flat. [Use Exercise 24 and the Tor exact sequence.] 26. Let $N$ be an $A$-module. Then $N$ is flat $-\operatorname{Tor}_{1}(A / \mathfrak{a}, N)=0$ for all finitely generated ideals $\mathfrak{a}$ in $A$.
\end{enumerate}

[Show first that $N$ is flat if $\operatorname{Tor}_{1}(M, N)=0$ for all finitely generated $A$-modules $M$, by using (2.19). If $M$ is finitely generated, let $x_{1}, \ldots, x_{n}$ be a set of generators of $M$, and let $M_{1}$ be the submodule generated by $x_{1}, \ldots, x_{1}$. By considering the successive quotients $M_{\mathfrak{t}} / M_{\mathfrak{i}-1}$ and using Exercise 25, deduce that $N$ is fiat if $\operatorname{Tor}_{1}(M, N)=0$ for all cyclic $A$-modules $M$, i.e., all $M$ generated by a single element, and therefore of the form $A / \mathfrak{a}$ for some ideal $a$. Finally use (2.19) again to reduce to the case where $a$ is a finitely generated ideal.]

\begin{enumerate}
  \setcounter{enumi}{26}
  \item A ring $A$ is absolutely flat if every $A$-module is flat. Prove that the following are equivalent:
\end{enumerate}

i) $\boldsymbol{A}$ is absolutely flat.

ii) Every principal ideal is idempotent.

iii) Every finitely generated ideal is a direct summand of $A$.

[i) $\Rightarrow$ ii). Let $x \in A$. Then $A /(x)$ is a flat $A$-module, hence in the diagram

\begin{center}
  \begin{tikzcd}
(x)\otimes A \arrow[d] \arrow[r, "\beta"] & (x)\otimes A/(x) \arrow[d, "\alpha"] \\
A \arrow[r]                               & A/(x)
\end{tikzcd}
\end{center}

the mapping $\alpha$ is injective. Hence $\operatorname{Im}(\beta)=0$, hence $(x)=\left(x^{2}\right)$. ii) $\Rightarrow$ iii). Let $x \in A$. Then $x=a x^{2}$ for some $a \in A$, hence $e=a x$ is idempotent and we have $(e)=(x)$. Now if $e, f$ are idempotents, then $(e, f)=(e+f-e f)$. Hence every finitely generated ideal is principal, and generated by an idempotent $e$, hence is a direct summand because $A=(e) \oplus(1-e)$. iii) $\Rightarrow$ i). Use the criterion of Exercise 26.]

\begin{enumerate}
  \setcounter{enumi}{27}
  \item A Boolean ring is absolutely flat. The ring of Chapter 1, Exercise 7 is absolutely fiat. Every homomorphic image of an absolutely flat ring is absolutely flat. If a local ring is absolutely flat, then it is a field.
\end{enumerate}

If $A$ is absolutely flat, every non-unit in $A$ is a zero-divisor.
\end{document}
