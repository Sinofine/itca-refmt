\documentclass{standalone}
\usepackage{unicode-math}
\setmathfont{texgyrepagella-math.otf}[math-style=TeX]
\usepackage{fontspec}
\setmainfont{TeX Gyre Pagella}
\usepackage{amsthm}
\newtheorem{theorem}{Theorem}[chapter]
\newtheorem{proposition}[theorem]{Proposition}
\newtheorem{lemma}[theorem]{lemma}
\newtheorem*{example}{Example}
\theoremstyle{definition}
\newtheorem{definition}[theorem]{Definition}
\theoremstyle{remark}
\newtheorem*{remark}{Remark}
\usepackage[export]{adjustbox}\graphicspath{ {./images/} }
\begin{document}
An Artin ring is one which satisfies the d.c.c. (or equivalently the minimal condition) on ideals.

The apparent symmetry with Noetherian rings is however misleading. In fact we will show that an Artin ring is necessarily Noetherian and of a very special kind. In a sense an Artin ring is the simplest kind of ring after a field, and we study them not because of their generality but because of their simplicity.

Proposition 8.1. In an Artin ring A etery prime ideal is maximal.

Proof. Let $\mathfrak{p}$ be a prime ideal of $A$. Then $B=A / \mathfrak{p}$ is an Artinian integral domain. Let $x \in B, x \neq 0$. By the d.c.c. we have $\left(x^{n}\right)=\left(x^{n+1}\right)$ for some $n$, hence $x^{n}=x^{n+1} y$ for some $y \in B$. Since $B$ is an integral domain and $x \neq 0$, it follows that we may cancel $x^{n}$, hence $x y=1$. Hence $x$ has an inverse in $B$, and therefore $B$ is a field, so that $\mathfrak{p}$ is a maximal ideal.

Corollary 8.2. In an Artin ring the nilradical is equal to the Jacobson radical.

Proposition 8.3. An Artin ring has only a finite number of maximal ideals. Proof. Consider the set of all finite intersections $\mathfrak{m}_{1} \cap \cdots \cap \mathfrak{m}_{r}$, where the $\mathfrak{m}_{i}$ are maximal ideals. This set has a minimal element, say $m_{1} \cap \cdots \cap m_{n}$; hence for any maximal ideal $\mathfrak{m}$ we have $\mathfrak{m} \cap \mathfrak{m}_{1} \cap \ldots \cap \mathfrak{m}_{n}=\mathfrak{m}_{1} \cap \cdots \cap \mathfrak{m}_{n}$, and therefore $\mathfrak{m} \supseteq \mathfrak{m}_{1} \cap \cdots \cap \mathfrak{m}_{n}$. By (1.11) $\mathfrak{m} \supseteq \mathfrak{m}_{i}$ for some $i$, hence $\mathfrak{m}=\mathfrak{m}_{i}$ since $\mathfrak{m}_{i}$ is maximal.

Proposition 8.4. In an Artin ring the nilradical $\mathfrak{R}$ is nilpotent.

Proof. By d.c.c. we have $\mathfrak{\Re}^{k}=\mathfrak{R}^{k+1}=\cdots=\mathfrak{a}$ say, for some $k>0$. Suppose $\mathfrak{a} \neq 0$, and let $\Sigma$ denote the set of all ideals $\mathfrak{b}$ such that $\mathfrak{a} \mathfrak{b} \neq 0$. Then $\Sigma$ is not empty, since $\mathfrak{a} \in \Sigma$. Let $\mathfrak{c}$ be a minimal element of $\Sigma$; then there exists $x \in \mathfrak{c}$ such that $x \mathfrak{a} \neq 0$; we have $(x) \subseteq \mathfrak{c}$, hence $(x)=\mathfrak{c}$ by the minimality of $\mathfrak{c}$. But $(x \mathfrak{a}) \mathfrak{a}=x \mathfrak{a}^{2}=x \mathfrak{a} \neq 0$, and $x \mathfrak{a} \subseteq(x)$, hence $x \mathfrak{a}=(x)$ (again by minimality). Hence $x=x y$ for some $y \in \mathfrak{a}$, and therefore $x=x y=x y^{2}=\cdots=x y^{n}=\cdots$. But $y \in \mathfrak{a}=\mathfrak{R}^{k} \supseteq \mathfrak{R}$, hence $y$ is nilpotent and therefore $x=x y^{n}=0$. This contradicts the choice of $x$, therefore $\mathfrak{a}=0$.

By a chain of prime ideals of a ring $A$ we mean a finite strictly increasing sequence $\mathfrak{p}_{0} \subset \mathfrak{p}_{1} \subset \ldots \subset \mathfrak{p}_{n}$; the length of the chain is $n$. We define the dimension of $A$ to be the supremum of the lengths of all chains of prime ideals in $A$ : it is an integer $\geqslant 0$, or $+\infty$ (assuming $A \neq 0$ ). A field has dimension 0 ; the ring $\mathbf{Z}$ has dimension 1 .

Theorem 8.5. A ring $A$ is Artin $\Leftrightarrow A$ is Noetherian and $\operatorname{dim} A=0$. Proof. $\Rightarrow:$ By (8.1) we have $\operatorname{dim} A=0$. Let $\mathfrak{n t}_{i}(1 \leqslant i \leqslant n)$ be the distinct maximal ideals of $A(8.3)$. Then $\prod_{i=1}^{n} \mathfrak{m}_{i}^{k} \subseteq\left(\bigcap_{i=1}^{n} \mathfrak{m}_{i}\right)^{k}=\mathfrak{R}^{k}=0$. Hence by (6.11) $A$ is Noetherian.

$\epsilon$ : Since the zero ideal has a primary decomposition (7.13), $A$ has only a finite number of minimal prime ideals, and these are all maximal since $\operatorname{din} A=0$. Hence $\mathfrak{R}=\bigcap_{i=1}^{n} \mathfrak{m}_{i}$ say; we have $\mathfrak{R}^{k}=0$ by $(7.15)$, hence $\prod_{i=1}^{n} \mathfrak{m}_{i}^{k}=0$ as in the previous part of the proof. Hence by (6.11) $A$ is an Artin ring.

If $A$ is an Artin local ring with maximal ideal $m$, then $m$ is the only prime ideal of $A$ and therefore $m$ is the nilradical of $A$. Hence every element of $m$ is nilpotent, and $\mathfrak{m}$ itself is nilpotent. Every element of $A$ is either a unit or is nilpotent. An example of such a ring is $Z /\left(p^{n}\right)$, where $p$ is prime and $n \geqslant 1$.

Proposition 8.6. Let $A$ be a Noetherian local ring, $m$ its maximal ideal. Then exactly one of the following two statements is true:

i) $\mathfrak{m}^{n} \neq \mathfrak{m}^{n+1}$ for all $n$;

ii) $\mathrm{m}^{\mathrm{n}}=0$ for some $n$, in which case $A$ is an Artin local ring.

Proof. Suppose $\mathfrak{m}^{n}=\mathfrak{m}^{n+1}$ for some $n$. By Nakayama's lemma (2.6) we have $\mathfrak{m}^{n}=0$. Let $\mathfrak{p}$ be any prime ideal of $A$. Then $\mathfrak{m}^{n} \subseteq \mathfrak{p}$, hence (taking radicals) $\mathfrak{m}=\mathfrak{p}$. Hence $\mathfrak{m}$ is the only prime ideal of $A$ and therefore $A$ is Artinian.

Theorem 8.7. (structure theorem for Artin rings). An Artin ring $A$ is uniquely (up to isomorphism) a finite direct product of Artin local rings.

Proof. Let $\mathfrak{m}_{i}(1 \leqslant i \leqslant n)$ be the distinct maximal ideals of $A$. From the proof of (8.5) we have $\prod_{i=1}^{n} \mathfrak{m}_{i}^{k}=0$ for some $k>0$. By (1.16) the ideals $\mathfrak{m}_{i}^{k}$ are coprime in pairs, hence $\bigcap \mathrm{m}_{i}^{k}=\Pi \mathrm{m}_{i}^{k}$ by (1.10). Consequently by (1.10) again the natural mapping $A \rightarrow \prod_{i=1}^{n}\left(A / \mathfrak{m}_{i}^{k}\right)$ is an isomorphism. Each $A / \mathfrak{m}_{i}^{k}$ is an Artin local ring, hence $A$ is a direct product of Artin local rings.

Conversely, suppose $A \cong \prod_{i=1}^{m} A_{i}$, where the $A_{i}$ are Artin local rings. Then for each $i$ we have a natural surjective homomorphism (projection on the $i$ th factor) $\phi_{i}: A \rightarrow A_{i}$. Let $\mathfrak{a}_{i}=\operatorname{Ker}\left(\phi_{i}\right)$. By $(1.10)$ the $a_{i}$ are pairwise coprime, and $\cap a_{i}=0$. Let $\mathfrak{q}_{i}$ be the unique prime ideal of $A_{i}$, and let $\mathfrak{p}_{i}$ be its contraction $\phi_{i}^{-1}\left(\mathfrak{q}_{i}\right)$. The ideal $\mathfrak{p}_{i}$ is prime and therefore maximal by (8.1). Since $\mathfrak{q}_{i}$ is nilpotent it follows that $\mathfrak{a}_{\mathfrak{i}}$ is $\mathfrak{p}_{\mathfrak{i}}$-primary, and hence $\cap \mathfrak{a}_{\mathfrak{i}}=(0)$ is a primary decomposition of the zero ideal in $A$. Since the $\mathfrak{a}_{i}$ are pairwise coprime, so are the $\mathfrak{p}_{\mathfrak{l}}$, and they are therefore isolated prime ideals of $(0)$. Hence all the primary components $a_{i}$ are isolated, and therefore uniquely determined by $A$, by the 2 nd uniqueness theorem (4.11). Hence the rings $A_{i} \cong A / a_{i}$ are uniquely determined by $A$. Example. A ring with only one prime ideal need not be Noetherian (and hence not an Artin ring). Let $A=k\left[x_{1}, x_{2}, \ldots\right]$ be the polynomial ring in a countably infinite set of indeterminates $x_{n}$ over a field $k$, and let $a$ be the ideal $\left(x_{1}, x_{2}^{2}, \ldots\right.$, $x_{n}^{n}, \ldots$ ). The ring $B=A / \mathfrak{a}$ has only one prime ideal (namely the image of $\left.\left(x_{1}, x_{2}, \ldots, x_{n}, \ldots\right)\right)$, hence $B$ is a local ring of dimension 0 . But $B$ is not Noetherian, for it is not difficult to see that its prime ideal is not finitely generated.

If $A$ is a local ring, $\mathfrak{m}$ its maximal ideal, $k=A / \mathfrak{m}$ its residue field, the $A$-module $\mathfrak{m} / \mathfrak{m}^{2}$ is annihilated by $\mathfrak{m}$ and therefore has the structure of a $k$-vector space. If $\mathfrak{m}$ is finitely generated (e.g., if $A$ is Noetherian), the images in $\mathfrak{m} / \mathfrak{m}^{2}$ of a set of generators of $\mathfrak{m}$ will span $\mathfrak{m} / \mathfrak{m}^{2}$ as a vector space, and therefore $\operatorname{dim}_{k}\left(\mathfrak{m} / \mathfrak{m}^{2}\right)$ is finite. (See (2.8).)

Proposition 8.8. Let $A$ be an Artin local ring. Then the following are equivalent:

i) every ideal in $A$ is principal;

ii) the maximal ideal $\mathfrak{m}$ is principal;

iii) $\operatorname{dim}_{k}\left(\mathfrak{m} / \mathfrak{m}^{2}\right) \leqslant 1$.

Proof. i) $\Rightarrow$ ii) $\Rightarrow$ iii) is clear.

iii) $\Rightarrow$ i): If $\operatorname{dim}_{k}\left(\mathfrak{m} / \mathfrak{m}^{2}\right)=0$, then $\mathfrak{m}=\mathfrak{m}^{2}$, hence $\mathfrak{m}=0$ by Nakayama's lemma (2.6), and therefore $\boldsymbol{A}$ is a field and there is nothing to prove.

If $\operatorname{dim}_{k}\left(\mathfrak{m} / \mathfrak{m}^{2}\right)=1$, then $\mathfrak{m}$ is a principal ideal by (2.8) (take $M=\mathfrak{m}$ there), say $\mathfrak{m}=(x)$. Let $\mathfrak{a}$ be an ideal of $A$, other than (0) or (1). We have $\mathfrak{m}=\mathfrak{N}$, hence $\mathfrak{m}$ is nilpotent by (8.4) and therefore there exists an integer $r$ such that $a \subseteq \mathfrak{m}^{r}, \mathfrak{a} \neq \mathfrak{m}^{r+1}$; hence there exists $y \in \mathfrak{a}$ such that $y=a x^{r}$, $y \notin\left(x^{r+1}\right)$; consequently $a \notin(x)$ and $a$ is a unit in $A$. Hence $x^{r} \in \mathfrak{a}$, therefore $\mathfrak{m}^{r}=\left(x^{r}\right)$. $\subseteq \mathfrak{a}$ and hence $\mathfrak{a}=\mathfrak{m}^{r}=\left(x^{r}\right)$. Hence $\mathfrak{a}$ is principal.

Example. The rings $Z /\left(p^{n}\right)$ ( $p$ prime), $k[x] /\left(f^{n}\right)(f$ irreducible) satisfy the conditions of (8.7). On the other hand, the Artin local ring $k\left[x^{2}, x^{3}\right] /\left(x^{4}\right)$ does not: here $\mathfrak{m}$ is generated by $x^{2}$ and $x^{3}\left(\bmod x^{4}\right)$, so that $\mathfrak{m}^{2}=0$ and $\operatorname{dim}\left(\mathfrak{m} / \mathfrak{m}^{2}\right)=2$.

\section{EXERCISES}
\begin{enumerate}
  \item Let $q_{1} \cap \cdots \cap q_{n}=0$ be a minimal primary decomposition of the zero ideal in a Noetherian ring, and let $\mathfrak{q}_{i}$ be $\mathfrak{p}_{i}$-primary. Let $\mathfrak{p}_{i}^{(r)}$ be the $r$ th symbolic power of $p_{i}$ (Chapter 4, Exercise 13). Show that for each $i=1, \ldots, n$ there exists an integer $r_{i}$ such that $\mathfrak{p}_{i}^{\left(r_{i}\right)} \subseteq q_{i}$.
\end{enumerate}

Suppose $\mathfrak{q}_{i}$ is an isolated primary component. Then $A_{\mathfrak{p}_{\mathfrak{i}}}$ is an Artin local ring, hence if $\mathfrak{m}_{t}$ is its maximal ideal we have $\mathfrak{m}_{i}^{r}=0$ for all sufficiently large $r$, hence $\mathfrak{q}_{i}=\mathfrak{p}_{i}^{(r)}$ for all large $r$. If $q_{1}$ is an embedded primary component, then $A_{\psi_{1}}$ is not Artinian, hence the powers $\mathfrak{m}_{i}^{\prime}$ are all distinct, and so the $\mathfrak{p}_{i}^{(r)}$ are all distinct. Hence in the given primary decomposition we can replace $q_{\mathfrak{l}}$ by any of the infinite set of $\mathfrak{p}_{1}$-primary ideals $\mathfrak{p}_{\mathfrak{i}}^{(\prime)}$ where $r \geqslant r_{i}$, and so there are infinitely many minimal primary decompositions of 0 which differ only in the $\mathfrak{p}_{\mathfrak{r}}$-component.

\begin{enumerate}
  \setcounter{enumi}{1}
  \item Let $A$ be a Noetherian ring. Prove that the following are equivalent:
\end{enumerate}

i) $A$ is Artinian;

ii) Spec $(A)$ is discrete and finite;

iii) $\operatorname{Spec}(A)$ is discrete.

\begin{enumerate}
  \setcounter{enumi}{2}
  \item Let $k$ be a field and $A$ a finitely generated $k$-algebra. Prove that the following are equivalent:
\end{enumerate}

i) $A$ is Artinian;

ii) $A$ is a finite $k$-algebra.

[To prove that i) $\Rightarrow$ ii), use (8.7) to reduce to the case where $A$ is an Artin local ring. By the Nullstellensatz, the residue field of $A$ is a finite extension of $k$. Now use the fact that $A$ is of finite length as an $A$-module. To prove ii) $\Rightarrow \mathrm{i}$ ), observe that the ideals of $A$ are $k$-vector subspaces and therefore satisfy d.c.c.]

\begin{enumerate}
  \setcounter{enumi}{3}
  \item Let $f: A \rightarrow B$ be a ring homomorphism of finite type. Consider the following statements:
\end{enumerate}

i) $f$ is finite;

ii) the fibres of $f^{*}$ are discrete subspaces of $\operatorname{Spec}(B)$;

iii) for each prime ideal $\mathfrak{p}$ of $A$, the ring $B \otimes_{A} k(\mathfrak{p})$ is a finite $k(\mathfrak{p})$-algebra $\left(k(\mathfrak{p})\right.$ is the residue field of $\left.A_{\mathfrak{p}}\right)$;

iv) the fibres of $f^{*}$ are finite.

Prove that i) $\Rightarrow$ ii) $\Leftrightarrow$ iii) $\Rightarrow$ iv). [Use Exercises 2 and 3.]

If $f$ is integral and the fibres of $f^{*}$ are finite, is $f$ necessarily finite?

\begin{enumerate}
  \setcounter{enumi}{4}
  \item In Chapter 5, Exercise 16, show that $X$ is a finite covering of $L$ (i.e., the number of points of $X$ lying over a given point of $L$ is finite and bounded).

  \item Let $A$ be a Noetherian ring and $q$ a $p$-primary ideal in $A$. Consider chains of primary ideals from $q$ to $\mathfrak{p}$. Show that all such chains are of finite bounded length, and that all maximal chains have the same length.

\end{enumerate}
\end{document}