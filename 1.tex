\documentclass[class=book, crop=false]{standalone}
\usepackage[subpreambles=true]{standalone}
\usepackage{unicode-math}
\setmathfont{texgyrepagella-math.otf}[math-style=TeX]
\usepackage{fontspec}
\setmainfont{TeX Gyre Pagella}
\usepackage{amsthm}
\newtheorem{theorem}{Theorem}[chapter]
\newtheorem{proposition}[theorem]{Proposition}
\newtheorem{lemma}[theorem]{lemma}
\newtheorem*{example}{Example}
\theoremstyle{definition}
\newtheorem{definition}[theorem]{Definition}
\theoremstyle{remark}
\newtheorem*{remark}{Remark}
\usepackage{enumitem}
\graphicspath{ {./images/} }
\begin{document}
We shall begin by reviewing rapidly the definition and elementary properties of
rings. This will indicate how much we are going to assume of the reader and it
will also serve to fix notation and conventions. After this review we pass on to
a discussion of prime and maximal ideals. The remainder of the chapter is
devoted to explaining the various elementary operations which can be performed
on ideals. The Grothendieck language of schemes is dealt with in the exercises
at the end.

\section{Rings and ring homomorphisms}
A ring $A$ is a set with two binary operations (addition and multiplication)
such that

\begin{enumerate}
  \item $A$ is an abelian group with respect to addition (so that $A$ has a zero
        element, denoted by $0$, and every $x \in A$ has an (additive) inverse,
        $-x$ ).
  \item Multiplication is associative $((x y) z=x(y z))$ and distributive over
        addition $(x(y+z)=x y+x z,(y+z) x=y x+z x)$.
\end{enumerate}
We shall consider only rings which are commutative:
\begin{enumerate}[resume*]
  \item $x y=y x$ for all $x, y \in A$,
\end{enumerate}
and have an identity element (denoted by $1$):
\begin{enumerate}[resume*]
  \item $\exists 1 \in A$ such that $x 1=1 x=x$ for all $x \in A$.
\end{enumerate}

The identity element is then unique.

Throughout this book the word ``ring'' shall mean a commutative ring with an
identity element, that is, a ring satisfying axioms (1) to (4) above.
\begin{remark}
  We do not exclude the possibility in (4) that 1 might be equal to $0$. If so,
  then for any $x \in A$ we have
  \[
    x=x 1=x 0=0
  \]
  and so $A$ has only one element, $0$. In this case $A$ is the zero ring,
  denoted by $0$ (by abuse of notation).
\end{remark}

A ring homomorphism is a mapping $f$ of a ring $A$ into a ring $B$ such that
\begin{enumerate}[i)]
  \item $f(x+y)=f(x)+f(y)$ (so that $f$ is a homomorphism of abelian groups, and
        therefore also $f(x-y)=f(x)-f(y), f(-x)=-f(x), f(0)=0)$,
  \item $f(x y)=f(x) f(y)$,
  \item $f(1)=1$.

\end{enumerate}

In other words, $f$ respects addition, multiplication and the identity element.

A subset $S$ of a ring $A$ is a subring of $A$ if $S$ is closed under addition
and multiplication and contains the identity element of $A$. The identity
mapping of $S$ into $A$ is then a ring homomorphism.

If $f: A\to B, g: B \to C$ are ring homomorphisms then so is their composition
$g \circ f: A \to C$.

\section{Ideals. Quotient rings}
An ideal $\mathfrak{a}$ of a ring $A$ is a subset of $A$ which is an additive
subgroup and is such that $A \mathfrak{a} \subseteq \mathfrak{a}$ (i.e.,
$x \in A$ and $y \in \mathfrak{a}$ imply $x y \in \mathfrak{a}$). The quotient
group $A / \mathfrak{a}$ inherits a uniquely defined multiplication from $A$
which makes it into a ring, called the quotient ring (or residue-class ring)
$A / \mathfrak{a}$. The elements of $A / \mathfrak{a}$ are the cosets of
$\mathfrak{a}$ in $A$, and the mapping $\phi: A \to A / \mathfrak{a}$ which maps
each $x \in A$ to its coset $x+a$ is a surjective ring homomorphism.

We shall frequently use the following fact:
\begin{proposition}
  There is a one-to-one order-preserving. correspondence between the ideals
  $\mathfrak{b}$ of $A$ which contain $\mathfrak{a}$, and the ideals $\bar{b}$
  of $A / \mathfrak{a}$, given by $\mathfrak{b}=\phi^{-1}(\bar{b})$.
\end{proposition}

If $f: A \to B$ is any ring homomorphism, the kernel of
$f\left(=f^{-1}(0)\right)$ is an ideal a of $A$, and the image of $f$ ($=f(A)$)
is a subring $C$ of $B$; and $f$ induces a ring isomorphism
$A / \mathfrak{a} \cong C$.

We shall sometimes use the notation $x \equiv y\pmod{a}$; this means that
$x-y \in \mathfrak{a}$.

\section{Zero-divisors. Nilpotent elements. Units}
A zero-divisor in a ring $A$ is an element $x$ which ``divides 0'', i.e., for
which there exists $y \neq 0$ in $A$ such that $x y=0$. A ring with no
zero-divisors $\neq 0$ (and in which $1 \neq 0$) is called an integral domain.
For example, $\mathbf{Z}$ and $k\left[x_{1}, \ldots, x_{n}\right]$ ( $k$ a
field, $x_{i}$ indeterminates) are integral domains.

An element $x \in A$ is nilpotent if $x^{n}=0$ for some $n>0$. A nilpotent
element is a zero-divisor (unless $A=0$), but not conversely (in general).

A unit in $A$ is an element $x$ which ``divides 1'', i.e., an element $x$ such
that $x y=1$ for some $y \in A$. The element $y$ is then uniquely determined by
$x$, and is written $x^{-1}$. The units in $A$ form a (multiplicative) abelian
group. The multiples $a x$ of an element $x \in A$ form a principal ideal,
denoted by $(x)$ or $A x$. $x$ is a unit $\iff$ $(x)=A=(1)$. The zero ideal
$(0)$ is usually denoted by $0$.

A field is a ring $A$ in which $1 \neq 0$ and every non-zero element is a unit.
Every field is an integral domain (but not conversely: $\mathbf{Z}$ is not a
field).

\begin{proposition}
  Let $A$ be a ring $\neq 0$. Then the following are equivalent:
  \begin{enumerate}[i)]
    \item $A$ is a field;
    \item the only ideals in $A$ are 0 and (1);
    \item every homomorphism of $A$ into a non-zero ring $B$ is injective.
  \end{enumerate}
\end{proposition}
\begin{proof}
  i) $\to$ ii). Let $\mathfrak{a} \neq 0$ be an ideal in $A$. Then $a$ contains
  a non-zero element $x$; $x$ is a unit, hence $\mathfrak{a} \supseteq(x)=(1)$,
  hence $\mathfrak{a}=(1)$.

  ii) $\to$ iii). Let $\phi: A \to B$. be a ring homomorphism. Then
  $\Ker(\phi)$ is an ideal $\neq(1)$ in $A$, hence
  $\Ker(\phi)=0$, hence $\phi$ is injective.

  iii) $\to$ i). Let $x$ be an element of $A$ which is not a unit. Then
  $(x) \neq(1)$, hence $B=A /(x)$ is not the zero ring. Let $\phi: A \to B$ be
  the natural homomorphism of $A$ onto $B$, with kernel $(x)$. By hypothesis,
  $\phi$ is injective, hence $(x)=0$, hence $x=0$.
\end{proof}
\section{Prime ideals and maximal ideals}
An ideal $\mathfrak{p}$ in $A$ is prime if $\mathfrak{p} \neq(1)$ and if
$x y \in \mathfrak{p} \implies x \in \mathfrak{p}$ or $y \in \mathfrak{p}$.

An ideal $\mathfrak{m}$ in $A$ is maximal if $\mathfrak{m} \neq(1)$ and if there
is no ideal $a$ such that $\mathfrak{m} \subset \mathfrak{a} \subset(1)$ (strict
inclusions). Equivalently:

$\mathfrak{p}$ is prime $\iff A / \mathfrak{p}$ is an integral
domain;

$\mathfrak{m}$ is maximal $\iff A / \mathfrak{m}$ is a field (by
(1.1) and (1.2)).

Hence a maximai ideal is prime (but not converseiy, in general). The zero ideal
is prime $\iff A$ is an integral domain.

If $f: A \to B$ is a ring homomorphism and $q$ is a prime ideal in $B$, then
$f^{-1}(q)$ is a prime ideal in $A$, for $A / f^{-1}(q)$ is isomorphic to a
subring of $B / q$ and hence has no zero-divisor $\neq 0$. But if $\mathfrak{n}$
is a maximal ideal of $B$ it is not necessarily true that $f^{-1}(\mathfrak{n})$
is maximal in $A$; all we can say for sure is that it is prime. (Example:
$A=\mathrm{Z}, B=\mathbf{Q}, \mathfrak{n}=0$.)

Prime ideals are fundamental to the whole of commutative algebra. The following
theorem and its corollaries ensure that there is always a sufficient supply of
them.
\begin{theorem}
  Every ring $A \neq 0$ has at least one maximal ideal. \textnormal{(Remember
    that ``ring'' means commutative ring with $1$.)}
\end{theorem}
\begin{proof}
  This is a standard application of Zorn's lemma. \footnote{Let $S$ be a
    non-empty partially ordered set (i.e., we are given a relation $x \leq y$ on
    $S$ which is reflexive and transitive and such that $x \leq y$ and
    $y \leq x$ together imply $x=y$ ). A subset $T$ of $S$ is a chain if either
    $x \leq y$ or $y \leq x$ for every pair of elements $x, y$ in $T$. Then
    Zorn's lemma may be stated as follows: if every chain $T$ of $S$ has an
    upper bound in $S$ (i.e., if there exists $x \in S$ such that $t \leq x$ for
    all $t \in T$ ) then $S$ has at least one maximal element.

    For a proof of the equivalence of Zorn's lemma with the axiom of choice, the
    well-ordering principle, etc., see for example P. R. Halmos, \textit{Na\"ive
      Set Theory}, Van Nostrand (1960).} Let $\Sigma$ be the set of all ideals
  $\ne (1)$ in $A$. Order $\Sigma$ by inclusion. $\Sigma$ is not empty, since
  $0 \in \Sigma$. To apply Zorn's lemma we must show that every chain in
  $\Sigma$ has an upper bound in $\Sigma$; let then
  $\left(\mathfrak{a}_{\alpha}\right)$ be a chain of ideals in $\Sigma$, so that
  for each pair of indices $\alpha, \beta$ we have either
  $\mathfrak{a}_{\alpha} \subseteq \mathfrak{a}_{\beta}$ or
  $\mathfrak{a}_{\beta} \subseteq \mathfrak{a}_{\alpha}$. Let
  $\mathfrak{a}=\bigcup_{\alpha} \mathfrak{a}_{\alpha}$. Then $\mathfrak{a}$ is
  an ideal (verify this) and $1 \notin \mathfrak{a}$ because
  $1 \notin \mathfrak{a}_{\alpha}$ for all $\alpha$. Hence $a \in \Sigma$, and
  $a$ is an upper bound of the chain. Hence by Zorn's lemma $\Sigma$ has a
  maximal element.
\end{proof}
\begin{corollary}
  If $a \neq(1)$ is an ideal of $A$, there exists a maximal ideal of $A$
  containing a.
\end{corollary}
\begin{proof}
  Apply (1.3) to $A / \mathfrak{a}$, bearing in mind (1.1). Alternatively,
  modify the proof of (1.3).
\end{proof}
\begin{corollary}
  Every non-unit of $A$ is contained in a maximal ideal.
\end{corollary}
\begin{remark}
  \begin{enumerate}
    \item If $A$ is Noetherian (Chapter 7) we can avoid the use of Zorn's lemma:
          the set of all ideals $\neq(1)$ has a maximal element.
    \item There exist rings with exactly one maximal ideal, for example fields.
          A ring $A$ with exactly one maximal ideal $m$ is called a local ring.
          The field $k=A / \mathfrak{m}$ is called the residue field of $A$.
  \end{enumerate}
\end{remark}
\begin{proposition}
  \begin{enumerate}[i)]
    \item Let $A$ be a ring and $\mathfrak{m} \neq(1)$ an ideal of $A$ such that
          every $x \in A-\mathfrak{m}$ is a unit in $A$. Then $A$ is a local
          ring and $\mathfrak{m}$ its maximal ideal.
    \item Let $A$ be a ring and $m$ a maximal ideal of $A$, such that every
          element of $1+\mathfrak{m}$ (i.e., every $1+x$, where
          $x \in \mathfrak{m})$ is a unit in $A$. Then $A$ is a local ring.
  \end{enumerate}
\end{proposition}
\begin{proof}
  \begin{enumerate}[i)]
    \item Every ideal $\neq(1)$ consists of non-units, hence is contained in
          $\mathfrak{m}$. Hence $\mathfrak{m}$ is the only maximal ideal of $A$.
    \item Let $x \in A-\mathfrak{m}$. Since $\mathfrak{m}$ is maximal, the ideal
          generated by $x$ and $\mathfrak{m}$ is $(1)$, hence there exist
          $y \in A$ and $t \in \mathfrak{m}$ such that $x y+t=1$; hence
          $x y=1-t$ belongs to $1+\mathfrak{m}$ and therefore is a unit. Now use
          $i$).
  \end{enumerate}
\end{proof}
A ring with only a finite number of maximal ideals is called semi-local.
\begin{example}
  \begin{enumerate}[1)]
    \item $A=k\left[x_{1}, \ldots, x_{n}\right], k$ a field. Let $f \in A$ be an
          irreducible polynomial. By unique factorization, the ideal $(f)$ is
          prime.
    \item $A=Z$. Every ideal in $Z$ is of the form $(m)$ for some $m \geq 0$.
          The ideal $(m)$ is prime $\iff m=0$ or a prime number. All
          the ideals $(p)$, where $p$ is a prime number, are maximal: $Z /(p)$
          is the field of $p$ elements.

          The same holds in Example 1) for $n=1$, but not for $n>1$. The ideal
          $n$ of all polynomials in $A=k\left[x_{1}, \ldots, x_{n}\right]$ with
          zero constant term is maximal (since it is the kernel of the
          homomorphism $A \to k$ which maps $f \in A$ to $f(0)$ ). But if
          $n>1, \mathfrak{m}$ is not a principal ideal: in fact it requires at
          least $n$ generators.
    \item A \textit{principal ideal domain} is an integral domain in which every
          ideal is principal. In such a ring every non-zero prime ideal is
          maximal. For if $(x) \neq 0$ is a prime ideal and $(y) \supset(x)$, we
          have $x \in(y)$, say $x=y z$, so that $y z \in(x)$ and $y \notin(x)$,
          hence $z \in(x)$ : say $z=t x$. Then $x=y z=y t x$, so that $y t=1$
          and therefore $(y)=(1)$.
  \end{enumerate}
\end{example}

\section{Nilradical and Jacobson radical}
\begin{proposition}
  The set $\Re$ of all nilpotent elements in a ring $A$ is an ideal, and
  $A / \Re$ has no nilpotent element $\neq 0$.
\end{proposition}
\begin{proof}
  If $x \in \Re$, clearly $a x \in \Re$ for all $a \in A$. Let $x, y \in \Re$:
  say $x^{m}=0, y^{n}=0$. By the binomial theorem (which is valid in any
  commutative ring), $(x+y)^{m+n-1}$ is a sum of integer multiples of products
  $x^{r} y^{s}$, where $r+s=m+n-1$; we cannot have both $r<m$ and $s<n$, hence
  each of these products vanishes and therefore $(x+y)^{m+n-1}=0$. Hence
  $x+y \in \Re$ and therefore $\Re$ is an ideal.

  Let $\bar{x} \in A / \Re$ be represented by $x \in A$. Then $\bar{x}^{n}$ is
  represented by $x^{n}$, so that
  $\bar{x}^{n}=0 \implies x^{n} \in \Re \implies\left(x^{n}\right)^{k}=0$ for
  some $k>0 \implies x \in \Re \implies \bar{x}=0$.
\end{proof}

The ideal $\Re$ is called the \textit{nilradical} of $A$. The following
proposition gives an alternative definition of $\Re$:

\begin{proposition}
  The nilradical of $A$ is the intersection of all the prime ideals of $A$.
\end{proposition}
\begin{proof}
  Let $\Re'$ denote the intersection of all the prime ideals of $A$. If
  $f \in A$ is nilpotent and if $\mathfrak{p}$ is a prime ideal, then
  $f^{n}=0 \in \mathfrak{p}$ for some $n>0$, hence $f \in \mathfrak{p}$ (because
  $\mathfrak{p}$ is prime). Hence $f \in \Re'$.

  Conversely, suppose that $f$ is not nilpotent. Let $\Sigma$ be the set of
  ideals $a$ with the property
  \[
    n>0 \implies f^{n} \notin \mathfrak{a}.
  \]
  Then $\Sigma$ is not empty because $0 \in \Sigma$. As in (1.3) Zorn's lemma
  can be applied to the set $\Sigma$, ordered by inclusion, and therefore
  $\Sigma$ has a maximal element. Let $\mathfrak{p}$ be a maximal element of
  $\Sigma$. We shall show that $\mathfrak{p}$ is a prime ideal. Let
  $x, y \notin \mathfrak{p}$. Then the ideals $\mathfrak{p}+(x)$,
  $athfrak{p}+(y)$ strictly contain $\mathfrak{p}$ and therefore do not belong
  to $\Sigma$; hence
  \[
    f^{m} \in \mathfrak{p}+(x), \quad f^{n} \in \mathfrak{p}+(y)
  \]
  for some $m, n$. It follows that $f^{m+n} \in \mathfrak{p}+(x y)$, hence the
  ideal $\mathfrak{p}+(x y)$ is not in $\Sigma$ and therefore
  $x y \notin \mathfrak{p}$. Hence we have a prime ideal $\mathfrak{p}$ such
  that $f \notin \mathfrak{p}$, so that $f \not\in\Re'$.
\end{proof}

The \textit{Jacobson radical} $\Re$ of $A$ is defined to be the intersection of
all the maximal ideals of $A$. It can be characterized as follows:
\begin{proposition}
  $x \in \Re \iff 1-x y$ is a unit in $A$ for all $y \in A$.
\end{proposition}
\begin{proof}
  $\implies$: Suppose $1-x y$ is not a unit. By (1.5) it belongs to some maximal
  ideal $\mathfrak{m}$; but $x \in \Re \subseteq \mathfrak{m}$, hence
  $x y \in \mathfrak{m}$ and therefore $1 \in \mathfrak{m}$, which is absurd.

  $\impliedby$: Suppose $x \notin \mathfrak{m}$ for some maximal ideal
  $\mathfrak{m}$. Then $\mathfrak{m}$ and $x$ generate the unit ideal $(1)$, so
  that we have $u+x y=1$ for some $u \in \mathfrak{m}$ and some $y \in A$. Hence
  $1-x y \in \mathfrak{m}$ and is therefore not a unit.
\end{proof}
\section{Operations on ideals}
If $\mathfrak{a}, \mathfrak{b}$ are ideals in a ring $A$, their sum
$\mathfrak{a}+\mathfrak{b}$ is the set of all $x+y$ where $x \in \mathfrak{a}$
and $y \in \mathfrak{b}$. It is the smallest ideal containing $\mathfrak{a}$ and
$\mathfrak{b}$. More generally, we may define the sum
$\sum_{i \in I} \mathfrak{a}_{\mathfrak{i}}$ of any family (possibly infinite)
of ideals $\mathfrak{a}_{i}$ of $A$; its elements are all sums $\sum x_{i}$,
where $x_{i} \in \mathfrak{a}_{i}$ for all $i \in I$ and almost all of the
$x_{i}$ (i.e., all but a finite set) are zero. It is the smallest ideal of $A$
which contains all the ideals $\mathfrak{a}_{i}$.

The \textit{intersection} of any family
$\left(\mathfrak{a}_{i}\right)_{i \in I}$ of ideals is an ideal. Thus the ideals
of $A$ form a complete lattice with respect to inclusion.

The \textit{product} of two ideals $\mathfrak{a}, \mathfrak{b}$ in $A$ is the
ideal $\mathfrak{a} \mathfrak{b}$ \textit{generated by} all products $x y$,
where $x \in \mathfrak{a}$ and $y \in \mathfrak{b}$. It is the set of all finite
sums $\sum x_{i} y_{i}$ where each $x_{i} \in \mathfrak{a}$ and each
$y_{i} \in \mathfrak{b}$. Similarly we define the product of any \textit{finite}
family of ideals. In particular the powers $\mathfrak{a}^{n}$ ($n>0$) of an
ideal a are defined; conventionally, $\mathfrak{a}^{0}=(1)$. Thus
$\mathfrak{a}^{n}$ ($n>0$) is the ideal generated by all products
$x_{1} x_{2} \cdots x_{n}$ in which each factor $x_{i}$ belongs to
$\mathfrak{a}$.

\begin{example}
  \begin{enumerate}[1)]
    \item If $A=\mathbf{Z}, \mathfrak{a}=(m), \mathfrak{b}=(n)$ then
          $\mathfrak{a}+\mathfrak{b}$ is the ideal generated by the
          h.c.f.\marginpar{h.c.f. is also known as g.c.d.} of $m$ and $n$;
          $\mathfrak{a} \cap \mathfrak{b}$ is the ideal generated by their
          l.c.m.; and $\mathfrak{a} \mathfrak{b}=(m n)$. Thus (in this case)
          $\mathfrak{a} \mathfrak{b}=\mathfrak{a} \cap \mathfrak{b} \iff m, n$
          are coprime.
    \item $A=k\left[x_{1}, \ldots, x_{n}\right]$,
          $a=\left(x_{1}, \ldots, x_{n}\right)=$ ideal generated by
          $x_{1}, \ldots, x_{n}$. Then $a^{m}$ is the set of all polynomials
          with no terms of degree $<m$.
  \end{enumerate}
\end{example}

The three operations so far defined (sum, intersection, product) are all
commutative and associative. Also there is the \textit{distributive law}
\[
  \mathfrak{a}(\mathfrak{b}+\mathfrak{c})=\mathfrak{a} \mathfrak{b}+\mathfrak{a} \mathfrak{c} .
\]

In the ring $Z$, $\cap$ and $+$ are distributive over each other. This is not
the case in general, and the best we have in this direction is the
\textit{modular law}
\[
  \mathfrak{a} \cap(\mathfrak{b}+\mathfrak{c})=\mathfrak{a} \cap \mathfrak{b}+\mathfrak{a} \cap \mathfrak{c} \text{
    if
  } \mathfrak{a} \supseteq \mathfrak{b} \text { or } \mathfrak{a} \supseteq \mathfrak{c}.
\]

Again, in $\mathbf{Z}$, we have
$(\mathfrak{a}+\mathfrak{b})(\mathfrak{a} \cap \mathfrak{b})=\mathfrak{a} \mathfrak{b}$;
but in general we have only
$(\mathfrak{a}+\mathfrak{b})(\mathfrak{a} \cap \mathfrak{b}) \subseteq \mathfrak{a} \mathfrak{b}$
(since
$(\mathfrak{a}+\mathfrak{b})(\mathfrak{a} \cap \mathfrak{b})=\mathfrak{a}(\mathfrak{a} \cap \mathfrak{b})+\mathfrak{b}(\mathfrak{a} \cap \mathfrak{b}) \subseteq \mathfrak{a} \mathfrak{b}$).
Clearly $\mathfrak{a} \mathfrak{b} \subseteq \mathfrak{a} \cap \mathfrak{b}$,
hence
\[
  \mathfrak{a} \cap \mathfrak{b}=\mathfrak{a} \mathfrak{b} \text{ provided
  } \mathfrak{a}+\mathfrak{b}=(1).
\]

Two ideals $\mathfrak{a}, \mathfrak{b}$ are said to be \textit{coprime} (or
comaximal) if $\mathfrak{a}+\mathfrak{b}=(1)$. Thus for coprime ideals we have
$\mathfrak{a} \cap \mathfrak{b}=\mathfrak{a} \mathfrak{b}$. Clearly two ideals
$\mathfrak{a}, \mathfrak{b}$ are coprime if and only if there exist
$x \in \mathfrak{a}$ and $y \in \mathfrak{b}$ such that $x+y=1$.

Let $A_{i}, \ldots, A_{n}$ be rings. Their \textit{direct product}
\[
  A=\prod_{i=1}^{n} A_{i}
\]
is the set of all sequences $x=\left(x_{1}, \ldots, x_{n}\right)$ with
$x_{i} \in A_{i}(1 \leq i \leq n)$ and componentwise addition and
multiplication. $A$ is a commutative ring with identity element
$(1,1, \ldots, 1)$. We have projections $p_{i}: A \to A_{i}$ defined by
$p_{i}(x)=x_{i}$; they are ring homomorphisms.

Let $A$ be a ring and $\mathfrak{a}_{1}, \ldots, \mathfrak{a}_{n}$ ideals of
$A$. Define a homomorphism
\[
  \phi: A \to \prod_{i=1}^{n}\left(A / \mathfrak{a}_{i}\right)
\]
by the rule
$\phi(x)=\left(x+\mathfrak{a}_{1}, \ldots, x+\mathfrak{a}_{n}\right)$.
\begin{proposition}
  \begin{enumerate}[i)]
    \item If $\mathfrak{a}_{i}, \mathfrak{a}_{j}$ are coprime whenever
          $i \neq j$, then $\prod a_{i}=\bigcap a_{i}$.
    \item $\phi$ is surjective $\iff \mathfrak{a}_{i}, \mathfrak{a}_{j}$ are
          coprime whenever $i \neq j$.
    \item $\phi$ is injective $\iff\bigcap a_{i}=(0)$.
  \end{enumerate}
\end{proposition}
\begin{proof}
  \begin{enumerate}[i)]
    \item by induction on $n$. The case $n=2$ is dealt with above. Suppose $n>2$
          and the result true for
          $\mathfrak{a}_{1}, \ldots, \mathfrak{a}_{n-1}$, and let
          $\mathfrak{b}=\prod_{i=1}^{n-1} \mathfrak{a}_{i}=\bigcap_{\mathfrak{i}=1}^{n-1} \mathfrak{a}_{i}$.
          Since $a_{i}+a_{n}=(1)$ ($1 \le i \le n-1$) we have equations
          $x_{i}+y_{i}=1$ ($x_{i} \in a_{i}$, $y_{i} \in a_{n}$) and therefore
          \[
          \prod_{1}^{n-1} x_{i}=\prod_{i=1}^{n-1}(1-y_{i}) \equiv 1\pmod {a_{n}}.
          \]
          Hence $\mathfrak{a}_{n}+\mathfrak{b}=(1)$ and so
          \[
          \prod_{i=1}^{n} \mathfrak{a}_{i}=\mathfrak{b} \mathfrak{a}_{n}=\mathfrak{b} \cap \mathfrak{a}_{n}=\bigcap_{i=1}^{n} \mathfrak{a}_{i}.
          \]
    \item $\implies$: Let us show for example that
          $\mathfrak{a}_{1}, \mathfrak{a}_{2}$ are coprime. There exists
          $x \in A$ such that $\phi(x)=(1,0, \ldots, 0)$; hence
          $x \equiv 1\pmod {a_{1}}$ and $x \equiv 0\pmod {a_{2}}$, so that
          \[
          1=(1-x)+x \in \mathfrak{a}_{1}+\mathfrak{a}_{2} .
          \]
          $\impliedby$: It is enough to show, for example, that there is an
          element $x \in A$ such that $\phi(x)=(1,0, \ldots, 0)$. Since
          $a_{1}+a_{i}=(1)$ ($i>1$) we have equations $u_{i}+v_{i}=1$
          ($\u_{i} \in a_{1}, v_{i} \in a_{i}$). Take $x=\prod_{i=2}^{n} v_{i}$,
          then $x=\prod\left(1-u_{i}\right) \equiv 1\pmod{a_{1}}$, and
          $x \equiv 0\pmod{a_{i}}$, $i>1$. Hence $\phi(x)=(1,0, \ldots, 0)$ as
          required.
    \item Clear, since $\cap a_{i}$ is the kernel of $\phi$.
  \end{enumerate}
\end{proof}
The \textit{union} $\mathfrak{a} \cup \mathfrak{b}$ of ideals is not in general
an ideal.
\begin{proposition}
  \begin{enumerate}[i)]
    \item Let $\mathfrak{p}_{1}, \ldots, \mathfrak{p}_{n}$ be prime ideals and
          let $\mathfrak{a}$ be an ideal contained in
          $\bigcup_{i=1}^{n} \mathfrak{p}_{i}$. Then
          $\mathfrak{a} \subseteq \mathfrak{p}_{i}$ for some $i$.
    \item Let $\mathfrak{a}_{1}, \ldots, \mathfrak{a}_{n}$ be ideals and let
          $\mathfrak{p}$ be a prime ideal containing
          $\bigcap_{i=1}^{n} \mathfrak{a}_{i}$. Then
          $\mathfrak{p} \supseteq \mathfrak{a}_{i}$ for some $i$. If
          $\mathfrak{p}=\bigcap \mathfrak{a}_{i}$, then
          $\mathfrak{p}=\mathfrak{a}_{i}$ for some $i$.
  \end{enumerate}
\end{proposition}
\begin{proof}
  \begin{enumerate}[i)]
    \item is proved by induction on $n$ in the form

\[
          \mathfrak{a} \nsubseteq \mathfrak{p}_{i}(1 \le i \le n) \implies \mathfrak{a} \nsubseteq \bigcup_{i=1}^{n} \mathfrak{p}_{i} .
          \]

          It is certainly true for $n=1$. If $n>1$ and the result is true for
          $n-1$, then for each $i$ there exists $x_{i} \in \mathfrak{a}$ such
          that $x_{i} \notin \mathfrak{p}_{j}$ whenever $j \neq i$. If for some
          $i$ we have $x_{i} \notin \mathfrak{p}_{i}$, we are through. If not,
          then $x_{i} \in \mathfrak{p}_{i}$ for all $i$. Consider the element
          \[
          y=\sum_{i=1}^{n} x_{1} x_{2} \cdots x_{i-1} x_{i+1} x_{i+2} \cdots x_{n};
          \]
          we have $y \in \mathfrak{a}$ and $y \notin \mathfrak{p}_{i}$
          ($1 \leq i \leq n$). Hence
          $a \nsubseteq \bigcup_{i=1}^{n} \mathfrak{p}_{i}$.
    \item Suppose $\mathfrak{p} \neq \mathfrak{a}_{i}$ for all $i$. Then there
          exist $x_{i} \in \mathfrak{a}_{i}, x_{i} \notin \mathfrak{p}$
          ($1 \leq i \leq n$), and therefore
          $\prod x_{i} \in \prod \mathfrak{a}_{i} \subseteq \bigcap \mathfrak{a}_{i}$;
          but $\prod x_{i} \notin \mathfrak{p}$ (since $\mathfrak{p}$ is prime).
          Hence $\mathfrak{p} \notin \cap \mathfrak{a}_{i}$. Finally, if
          $\mathfrak{p}=\bigcap \mathfrak{a}_{i}$, then
          $\mathfrak{p} \subseteq \mathfrak{a}_{i}$ and hence
          $\mathfrak{p}=\mathfrak{a}_{i}$ for some $i$.
  \end{enumerate}
\end{proof}

If $\symfrak{a}, \symfrak{b}$ are ideals in a ring $A$, their \textit{ideal
  quotient} is
\[
  (\mathfrak{a}: \mathfrak{b})=\{x \in A: x \mathfrak{b} \subseteq \mathfrak{a}\}
\]

which is an ideal. In particular, $(0: \mathfrak{b})$ is called the
\textit{annihilator} of $\mathfrak{b}$ and is also denoted by
$\operatorname{Ann}(\mathfrak{b})$ is the set of all $x \in A$ such that
$x \mathfrak{b}=0$. In this notation the set of all zero-divisors in $A$ is
\[
  D=\bigcup_{x \neq 0} \operatorname{Ann}(x).
\]
If $\mathfrak{b}$ is a principal ideal $(x)$, we shall write $(\mathfrak{a}: x)$
in place of $(\mathfrak{a}:(x))$.
\begin{example}
  If $A=\mathbf{Z}, \mathfrak{a}=(m), \mathfrak{b}=(n)$, where say
  $m=\prod_{p} p^{\mu_{p}}, n=\Pi_{p} p^{\nu_{p}}$, then
  $(\mathfrak{a}: \mathfrak{b})=(q)$ where $q=\prod_{p} p^{\gamma_{p}}$ and

\[
  \gamma_{p}=\max \left(\mu_{p}-\nu_{p}, 0\right)=\mu_{p}-\min \left(\mu_{p}, \nu_{p}\right).
\]

Hence $q=m /(m, n)$, where $(m, n)$ is the h.c.f. of $m$ and $n$.
\end{example}
\begin{exercise}
  \begin{enumerate}[i)]
    \item $a \subseteq(a: b)$
    \item $(\mathfrak{a}: \mathfrak{b}) \mathfrak{b} \subseteq \mathfrak{a}$
    \item
          $((\mathfrak{a}: \mathfrak{b}): \mathfrak{c})=(\mathfrak{a}: \mathfrak{b c})=((\mathfrak{a}: \mathfrak{c}): \mathfrak{b})$
    \item
          $\left(\bigcap_{i} \mathfrak{a}_{i}: \mathfrak{b}\right)=\bigcap_{i}\left(\mathfrak{a}_{i}: \mathfrak{b}\right)$
    \item
          $\left(\mathfrak{a}: \sum_{i} \mathfrak{b}_{i}\right)=\bigcap_{i}\left(\mathfrak{a}: \mathfrak{b}_{i}\right)$.
  \end{enumerate}
\end{exercise}

If $\mathfrak{a}$ is any ideal of $A$, the \textit{radical} of $\mathfrak{a}$ is
\[
  r(\mathfrak{a})=\left\{x \in A: x^{n} \in \mathfrak{a} \text { for some } n>0\right\}.
\]
If $\phi: A \to A / \mathfrak{a}$ is the standard homomorphism, then
$r(\mathfrak{a})=\phi^{-1}\left(\mathfrak{R}_{A / \mathfrak{a}}\right)$ and
hence $r(\mathfrak{a})$ is an ideal by (1.7).
\begin{exercise}
  \begin{enumerate}[i)]
    \item $r(\mathfrak{a}) \supseteq \mathfrak{a}$
    \item $r(r(\mathfrak{a}))=r(\mathfrak{a})$
    \item
          $r(\mathfrak{a} \mathfrak{b})=r(\mathfrak{a} \cap \mathfrak{b})=r(\mathfrak{a}) \cap r(\mathfrak{b})$
    \item $r(a)=(1) \iff \mathfrak{a}=(1)$
    \item $r(\mathfrak{a}+\mathfrak{b})=r(r(\mathfrak{a})+r(\mathfrak{b}))$
    \item if $\mathfrak{p}$ is prime,
          $r\left(\mathfrak{p}^{n}\right)=\mathfrak{p}$ for all $n>0$.
  \end{enumerate}
\end{exercise}
\begin{proposition}
  The radical of an ideal $\mathfrak{a}$ is the intersection of the prime ideals
  which contain $\mathfrak{a}$.
\end{proposition}
\begin{proof}
  Apply (1.8) to $A / \mathfrak{a}$.
\end{proof}

More generally, we may define the radical $r(E)$ of any subset $E$ of $A$ in the
same way. It is \textit{not} an ideal in general. We have
$r\left(\bigcup_{\alpha} E_{\alpha}\right)=\bigcup r\left(E_{\alpha}\right)$,
for any family of subsets $E_{\alpha}$ of $A$.
\begin{proposition}
  $D=$ set of zero-divisors of $A=\bigcup_{x \neq 0} r(\operatorname{Ann}(x))$.
\end{proposition}
\begin{proof}
  $D=r(D)=r\left(\bigcup_{x \neq 0}\operatorname{Ann}(x)\right)=\bigcup_{x \neq 0} r(\operatorname{Ann}(x))$.
\end{proof}
\begin{example}
  If $A=\mathbf{Z}, \mathfrak{a}=(m)$, let $p_{i}$ ($1 \leq i \leq r$) be the
  distinct prime divisors of $m$. Then
  $r(\mathfrak{a})=\left(p_{1} \cdots p_{r}\right)=\bigcap_{i=1}^{r}\left(p_{i}\right)$.
\end{example}
\begin{proposition}
  Let $\mathfrak{a}, \mathfrak{b}$ be ideals in a ring $A$ such that
  $r(\mathfrak{a}), r(\mathfrak{b})$ are coprime. Then
  $\mathfrak{a}, \mathfrak{b}$ are coprime.
\end{proposition}
\begin{proof}
  $r(\mathfrak{a}+\mathfrak{b})=r(r(\mathfrak{a})+r(\mathfrak{b}))=r(1)=(1)$,
  hence $\mathfrak{a}+\mathfrak{b}=(1)$ by (1.13).
\end{proof}
\section{Extension and contraction}
Let $f: A \to B$ be a ring homomorphism. If $\mathfrak{a}$ is an ideal in $A$,
the set $f(\mathfrak{a})$ is not necessarily an ideal in $B$ (e.g., let $f$ be
the embedding of $\mathbf{Z}$ in $\mathbf{Q}$, the field of rationals, and take
$\mathfrak{a}$ to be any non-zero ideal in $Z$.) We define the extension
$\mathfrak{a}^{e}$ of $\mathfrak{a}$ to be the ideal $B f(\mathfrak{a})$
generated by $f(\mathfrak{a})$ in $B$: explicitly, $\mathfrak{a}^{e}$ is the set
of all sums $\sum y_{i} f\left(x_{i}\right)$ where
$x_{i} \in \mathfrak{a}, y_{i} \in B$.

If $\mathfrak{b}$ is an ideal of $B$, then $f^{-1}(\mathfrak{b})$ is always an
ideal of $A$, called the \textit{contraction} $\mathfrak{b}^{c}$ of
$\mathfrak{b}$. If $\mathfrak{b}$ is prime, then $\mathfrak{b}^{c}$ is prime. If
$\mathfrak{a}$ is prime, $\mathfrak{a}^{e}$ need not be prime (for example,
$f: \mathbf{Z} \to \mathbf{Q}, \mathfrak{a} \neq 0$; then
$\mathfrak{a}^{e}=\mathbf{Q}$, which is not a prime ideal).

We can factorize $f$ as follows:

\[
  A \stackrel{p}{\to} f(A) \stackrel{j}{\to} B
\]
where $p$ is surjective and $j$ is injective. For $p$ the situation is very
simple (1.1): there is a one-to-one correspondence between ideals of $f(A)$ and
ideals of $A$ which contain $\Ker(f)$, and prime ideals correspond
to prime ideals. For $j$, on the other hand, the general situation is very
complicated. The classical example is from algebraic number theory.
\begin{example}
  Consider $\mathbf{Z} \to \mathbf{Z}[i]$, where $i=\sqrt{-1}$. A prime ideal
  $(p)$ of $\mathbf{Z}$ may or may not stay prime when extended to $Z[i]$. In
  fact $Z[i]$ is a principal ideal domain (because it has a Euclidean algorithm)
  and the situation is as follows:
  \begin{enumerate}[i)]
    \item $(2)^{e}=\left((1+i)^{2}\right)$, the \textit{square} of a prime ideal
          in $Z[i]$;
    \item If $p \equiv 1\pmod{4}$ then $(p)^{e}$ is the product of two distinct
          prime ideals (for example, $\left.(5)^{e}=(2+i)(2-i)\right)$;
    \item If $p \equiv 3\pmod{4}$ then $(p)^{e}$ is prime in $Z[i]$.
  \end{enumerate}
  Of these, ii) is not a trivial result. It is effectively equivalent to a
  theorem of Fermat which says that a prime $p \equiv 1\pmod{4}$ can be
  expressed, essentially uniquely, as a sum of two integer squares (thus
  $5=2^{2}+1^{2}, 97=9^{2}+4^{2}$, etc.).

  In fact the behavior of prime ideals under extensions of this sort is one of
  the central problems of algebraic number theory.
\end{example}

Let $f: A \to B$, $\mathfrak{a}$ and $\mathfrak{b}$ be as before. Then
\begin{proposition}
  \begin{enumerate}[i)]
    \item
          $\mathfrak{a} \subseteq \mathfrak{a}^{e c}, \mathfrak{b} \supseteq \mathfrak{b}^{c e}$;
    \item
          $\mathfrak{b}^{c}=\mathfrak{b}^{\text {cec }}, \mathfrak{a}^{e}=\mathfrak{a}^{e c e}$;
    \item If $C$ is the set of contracted ideals in $A$ and if $E$ is the set of
          extended ideals in $B$, then
          $C=\left\{\mathfrak{a} \mid \mathfrak{a}^{e c}=\mathfrak{a}\right\}, E=\left\{\mathfrak{b} \mid \mathfrak{b}^{c e}=\mathfrak{b}\right\}$,
          and $\mathfrak{a} \mapsto \mathfrak{a}^{e}$ is a bijective map of $C$
          onto $E$, whose inverse is $\mathfrak{b} \mapsto \mathfrak{b}^{c}$.
  \end{enumerate}
\end{proposition}
\begin{proof}
  \begin{enumerate}[i)]
    \item is trivial, and ii) follows from i).
    \item If $\mathfrak{a} \in C$, then
          $\mathfrak{a}=\mathfrak{b}^{c} .=\mathfrak{b}^{\text {cec }}=\mathfrak{a}^{e c}$;
          conversely if $\mathfrak{a}=\mathfrak{a}^{e c}$ then $\mathfrak{a}$ is
          the contraction of $\mathfrak{a}^{e}$. Similarly for $E$.
  \end{enumerate}
\end{proof}
\begin{exercise}
  If $\mathfrak{a}_{1}$, $\mathfrak{a}_{2}$ are ideals of $A$ and if
  $\mathfrak{b}_{1}$, $\mathfrak{b}_{2}$ are ideals of $B$, then
  \[
    \begin{array}{ll}
      \left(\mathfrak{a}_{1}+\mathfrak{a}_{2}\right)^{e}=\mathfrak{a}_{1}^{e}+\mathfrak{a}_{2}^{e}, & \left(\mathfrak{b}_{1}+\mathfrak{b}_{2}\right)^{c} \supseteq \mathfrak{b}_{1}^{c}+\mathfrak{b}_{2}^{c}, \\
      \left(\mathfrak{a}_{1} \cap \mathfrak{a}_{2}\right)^{e} \subseteq \mathfrak{a}_{1}^{e} \cap \mathfrak{a}_{2}^{e}, & \left(\mathfrak{b}_{1} \cap \mathfrak{b}_{2}\right)^{c}=\mathfrak{b}_{1}^{c} \cap \mathfrak{b}_{2}^{c}, \\
      \left(\mathfrak{a}_{1} \mathfrak{a}_{2}\right)^{e}=\mathfrak{a}_{1}^{e} \mathfrak{a}_{2}^{e}, & \left(\mathfrak{b}_{1} \mathfrak{b}_{2}\right)^{c} \supseteq \mathfrak{b}_{1}^{c} \mathfrak{b}_{2}^{c}, \\
      \left(\mathfrak{a}_{1}: \mathfrak{a}_{2}\right)^{e} \subseteq\left(\mathfrak{a}_{1}^{e}: \mathfrak{a}_{2}^{e}\right), & \left(\mathfrak{b}_{1}: \mathfrak{b}_{2}\right)^{c} \subseteq\left(\mathfrak{b}_{1}^{c}: \mathfrak{b}_{2}^{c}\right), \\
      r(\mathfrak{a})^{e} \subseteq r\left(\mathfrak{a}^{e}\right), & r(\mathfrak{b})^{c}=r\left(\mathfrak{b}^{c}\right).
    \end{array}
  \]
  The set of ideals $E$ is closed under sum and product, and $C$ is closed under
  the other three operations.
\end{exercise}
\section{Exercises}
\begin{enumerate}[series=exc1]
  \item Let $x$ be a nilpotent element of a ring $A$. Show that $1+x$ is a unit
        of $A$. Deduce that the sum of a nilpotent element and a unit is a unit.

  \item Let $A$ be a ring and let $A[x]$ be the ring of polynomials in an
        indeterminate $x$, with coefficients in $A$. Let
        $f=a_{0}+a_{1} x+\cdots+a_{n} x^{n} \in A[x]$. Prove that
        \begin{enumerate}
          \item $f$ is a unit in $A[x] \iff a_{0}$ is a unit in $A$
                and $a_{1}, \ldots, a_{n}$ are nilpotent. [If
                $b_{0}+b_{1} x+\cdots+b_{m} x^{m}$ is the inverse of $f$, prove
                by induction on $r$ that $a_{n}^{r+1} b_{m-r}=0$. Hence show
                that $a_{n}$ is nilpotent, and then use Ex. 1.]
          \item $f$ is nilpotent $\iff a_{0}, a_{1}, \ldots, a_{n}$
                are nilpotent.
          \item $f$ is a zero-divisor $\iff$ there exists $a \neq 0$
                in $A$ such that $a f=0$. [Choose a polynomial
                $g=b_{0}+b_{1} x+\cdots+b_{m} x^{m}$ of least degree $m$ such
                that $f g=0$. Then $a_{n} b_{m}=0$, hence $a_{n} g=0$ (because
                $a_{n} g$ annihilates $f$ and has degree $<m$). Now show by
                induction that $a_{n - r} g=0$ ($0 \leq r \leq n$).]
          \item $f$ is said to be primitive if
                $\left(a_{0}, a_{1}, \ldots, a_{n}\right)=(1)$. Prove that if
                $f, g \in A[x]$, then $f g$ is primitive $\iff f$ and
                $g$ are primitive.
        \end{enumerate}

  \item Generalize the results of Exercise 2 to a polynomial ring
        $A[x_{1}, \ldots, x_{r}]$ in several indeterminates.

  \item In the ring $A[x]$, the Jacobson radical is equal to the nilradical.

  \item Let $A$ be a ring and let $A\lBrack x\rBrack $ be the ring of formal power series
  $f=\sum_{n=0}^{\infty} a_{n} x^{n}$ with coefficients in $A$. Show that
  \begin{enumerate}
    \item $f$ is a unit in $A\lBrack x\rBrack  \iff a_{0}$ is a unit in $A$.
    \item If $f$ is nilpotent, then $a_{n}$ is nilpotent for all $n \geq 0$. Is the
converse true? (See Chapter 7, Exercise 2.)
    \item $f$ belongs to the Jacobson radical of $A\lBrack x\rBrack  \iff a_{0}$
belongs to the Jacobson radical of $A$.
    \item The contraction of a maximal ideal $\mathfrak{m}$ of $A\lBrack x\rBrack $ is a maximal
ideal of $A$, and $m$ is generated by $\mathfrak{m}^{c}$ and $x$.
    \item Every prime ideal of $A$ is the contraction of a prime ideal of $A\lBrack x\rBrack $.
  \end{enumerate}

  \item A ring $A$ is such that every ideal not contained in the nilradical
        contains a nonzero idempotent (that is, an element $e$ such that
        $e^{2}=e \neq 0$ ). Prove that the nilradical and Jacobson radical of
        $\boldsymbol{A}$ are equal.

  \item Let $A$ be a ring in which every element $x$ satisfies $x^{n}=x$ for
        some $n>1$ (depending on $x$ ). Show that every prime ideal in $A$ is
        maximal.

  \item Let $A$ be a ring $\neq 0$. Show that the set of prime ideals of $A$ has
        minimal elements with respect to inclusion.

  \item Let $\mathfrak{a}$ be an ideal $\neq (1)$ in a ring $A$. Show that
        $\mathfrak{a}=r(\mathfrak{a}) \iff \mathfrak{a}$ is an
        intersection of prime ideals.

  \item Let $A$ be a ring, $\Re$ its nilradical. Show that the following are
        equivalent:
\begin{enumerate}[i)]
  \item $A$ has exactly one prime ideal;
  \item every element of $\boldsymbol{A}$ is either a unit or nilpotent;
  \item $A / \Re$ is a field.
\end{enumerate}

  \item A ring $A$ is Boolean if $x^{2}=x$ for all $x \in A$. In a Boolean ring
  $A$, show that
  \begin{enumerate}[i)]
    \item $2 x=0$ for all $x \in A$;
    \item every prime ideal $p$ is maximal, and $A / \mathfrak{p}$ is a field with two
elements;
    \item every finitely generated ideal in $A$ is principal.
  \end{enumerate}
  \item A local ring contains no idempotent $\neq 0,1$.
\end{enumerate}

\subsection*{Construction of an algebraic closure of a field (E. Artin).}

\begin{enumerate}[resume*=exc1]
  \item Let $K$ be a field and let $\Sigma$ be the set of all irreducible monic
        polynomials $f$ in one indeterminate with coefficients in $K$. Let $A$
        be the polynomial ring over $K$ generated by indeterminates $x_{f}$, one
        for each $f \in \Sigma$. Let $a$ be the ideal of $A$ generated by the
        polynomials $f\left(x_{f}\right)$ for all $f \in \Sigma$. Show that
        $a \neq(1)$.

Let $\mathfrak{m}$ be a maximal ideal of $A$ containing $\mathfrak{a}$, and let
$K_{1}=A / \mathfrak{m}$. Then $K_{1}$ is an extension field of $K$ in which
each $f \in \Sigma$ has a root. Repeat the construction with $K_{1}$ in place of
$K$, obtaining a field $K_{2}$, and so on. Let $L=\bigcup_{n=1}^{\infty} K_{n}$.
Then $L$ is a field in which each $f \in \Sigma$ splits completely into linear
factors. Let $\bar{K}$ be the set of all elements of $L$ which are algebraic over $K$.
Then $\bar{K}$ is an algebraic closure of $K$.

  \item In a ring $A$, let $\Sigma$ be the set of all ideals in which every
        element is a zero-divisor. Show that the set $\Sigma$ has maximal
        elements and that every maximal element of $\Sigma$ is a prime ideal.
        Hence the set of zero-divisors in $A$ is a union of prime ideals.
\end{enumerate}

\subsection*{The prime spectrum of a ring}
\begin{enumerate}[resume*=exc1]
  \item Let $A$ be a ring and let $X$ be the set of all prime ideals of $A$. For
        each subset $E$ of $A$, let $V(E)$ denote the set of all prime ideals of
        $A$ which contain $E$. Prove that
        \begin{enumerate}[i)]
          \item if $\mathfrak{a}$ is the ideal generated by $E$, then
$V(E)=V(\mathfrak{a})=V(r(\mathfrak{a}))$.
          \item $V(0)=X, V(1)=\varnothing$.
          \item if $\left(E_{i}\right)_{i \in I}$ is any family of subsets of $A$, then
\[
  V\left(\bigcup_{i \in I} E_{i}\right)=\bigcap_{i \in I} V\left(E_{i}\right).
\]
          \item $V(\mathfrak{a} \cap \mathfrak{b})=V(\mathfrak{a} \mathfrak{b})=V(\mathfrak{a}) \cup V(\mathfrak{b})$
for any ideals $\mathfrak{a}$, $\mathfrak{b}$ of $A$.
        \end{enumerate}

These results show that the sets $V(E)$ satisfy the axioms for closed sets in a
topological space. The resulting topology is called the \textit{Zariski topology}. The
topological space $X$ is called the \textit{prime spectrum} of $A$, and is written
$\Spec(A)$.

  \item Draw pictures of $\Spec(\mathbf{Z})$, $\Spec(\mathbf{R})$, $\Spec(\mathbf{C}[x])$, $\Spec(\mathbf{R}[x])$, $\Spec(\mathbf{Z}[x])$.

  \item For each $f \in A$, let $X_{f}$ denote the complement of $V(f)$ in
        $X=\Spec(A)$. The sets $X_{f}$ are open. Show that they
        form a basis of open sets for the Zariski topology, and that
\begin{enumerate}[i)]
  \item $X_{f} \cap X_{g}=X_{f g}$;
  \item $X_{f}=\varnothing \iff f$ is nilpotent;
  \item $X_{f}=X \iff f$ is a unit;
  \item $X_{f}=X_{g} \iff r((f))=r((g))$;
  \item $X$ is quasi-compact (that is, every open covering of $X$ has a finite
subcovering).
  \item More generally, each $X_{f}$ is quasi-compact.
  \item An open subset of $X$ is quasi-compact if and only if it is a finite union
of sets $X_{f}$.

The sets $X_{f}$ are called basic open sets of $X=\Spec(A)$.
\end{enumerate}
[To prove (v), remark that it is enough to consider a covering of $X$ by basic
open sets $X_{f_{i}}$ ($i \in I$). Show that the $f_{i}$ generate the unit ideal
and hence that there is an equation of the form
\[
  1=\sum_{i \in J} g_{i} f_{i} \quad\left(g_{i} \in A\right)
\]
where $J$ is some \textit{finite} subset of $I$. Then the $X_{f_{i}}$ ($i \in J$) cover
$X$.]

\item For psychological reasons it is sometimes convenient to denote a prime
ideal of $A$ by a letter such as $x$ or $y$ when thinking of it as a point of
$X=\Spec(A)$. When thinking of $x$ as a prime ideal of $A$, we
denote it by $\mathfrak{p}_{x}$ (logically, of course, it is the same thing).
Show that
\begin{enumerate}[i)]
  \item the set $\{x\}$ is closed (we say that $x$ is a "closed point") in
$\Spec(A) \iff \mathfrak{p}_{x}$ is maximal;
  \item $\overline{\{x\}}=V\left(\mathfrak{p}_{x}\right)$;
  \item $y \in \overline{\{x\}} \iff \mathfrak{p}_{x} \subseteq \mathfrak{p}_{y}$;
  \item $X$ is a $T_{0}$-space (this means that if $x, y$ are distinct points of
$X$, then either there is a neighborhood of $x$ which does not contain $y$, or
else there is a neighborhood of $y$ which does not contain $x$ ).
\end{enumerate}
  \item A topological space $X$ is said to be \textit{irreducible} if
        $X \neq \varnothing$ and if every pair of non-empty open sets in $X$
        intersect, or equivalently if every non-empty open set is dense in $X$.
        Show that $\Spec(A)$ is irreducible if and only if the
        nilradical of $A$ is a prime ideal.

  \item Let $X$ be a topological space.
\begin{enumerate}
  \item If $Y$ is an irreducible (Exercise 19) subspace of $X$, then the closure
$\overline{Y}$ of $Y$ in $X$ is irreducible.
  \item Every irreducible subspace of $X$ is contained in a maximal irreducible
subspace.
  \item The maximal irreducible subspaces of $X$ are closed and cover $X$. They are
called the \textit{irreducible components} of $X$. What are the irreducible components of
a Hausdorff space?
  \item If $A$ is a ring and $X=\Spec(A)$, then the irreducible
components of $X$ are the closed sets $V(\mathfrak{p})$, where $\mathfrak{p}$ is
a minimal prime ideal of $A$ (Exercise 8).
\end{enumerate}

  \item Let $\phi: A \to B$ be a ring homomorphism. Let
        $X=\Spec(A)$ and $Y=\Spec(B)$. If
        $\mathfrak{q} \in Y$, then $\phi^{-1}(\mathfrak{q})$ is a prime ideal of $A$,
        i.e., a point of $X$. Hence $\phi$ induces a mapping
        $\phi^{*}: Y \to X$. Show that
\begin{enumerate}[i)]
  \item If $f \in A$ then $\phi^{*-1}\left(X_{f}\right)=Y_{\phi(f)}$, and hence that
$\phi^{*}$ is continuous.
  \item If $\mathfrak{a}$ is an ideal of $A$, then
$\phi^{*-1}(V(\mathfrak{a}))=V\left(\mathfrak{a}^{e}\right)$.
  \item If $\mathfrak{b}$ is an ideal of $B$, then
$\overline{\phi^{*}(V(\mathfrak{b}))}=V\left(\mathfrak{b}^{c}\right)$.
  \item If $\phi$ is surjective, then $\phi^{*}$ is a homeomorphism of $Y$ onto the
closed subset $V(\Ker(\phi))$ of $X$. (In particular,
$\Spec(A)$ and $\Spec(A / \Re)$ (where $\Re$ is the
nilradical of $A$ ) are naturally homeomorphic.)
  \item If $\phi$ is injective, then $\phi^{*}(Y)$ is dense in $X$. More precisely,
$\phi^{*}(Y)$ is dense in
$X \iff \Ker(\phi) \subseteq \Re$.
  \item Let $\psi: B \to C$ be another ring homomorphism. Then
$(\psi \circ \phi)^{*}=\phi^{*} \circ \psi^{*}$.
  \item Let $A$ be an integral domain with just one non-zero prime ideal
$\mathfrak{p}$, and let $K$ be the field of fractions of $A$. Let
$B=(A / \mathfrak{p}) \times K$. Define $\phi: A \to B$ by
$\phi(x)=(\bar{x}, x)$, where $\bar{x}$ is the image of $x$ in
$A / \mathfrak{p}$. Show that $\phi^{*}$ is bijective but not a homeomorphism.
\end{enumerate}

  \item Let $A=\prod_{i=1}^{n} A_{i}$ be the direct product of rings $A_{i}$.
        Show that $\Spec(A)$ is the disjoint union of open (and
        closed) subspaces $X_{i}$, where $X_{i}$ is canonically homeomorphic
        with $\Spec\left(A_{i}\right)$.

        Conversely, let $A$ be any
        ring. Show that the following statements are equivalent:
        \begin{enumerate}[i)]
          \item $X=\Spec(A)$ is disconnected.
          \item $A \cong A_{1} \times A_{2}$ where neither of the rings $A_{1}, A_{2}$ is
the zero ring.
          \item $A$ contains an idempotent $\neq 0,1$.
        \end{enumerate}
In particular, the spectrum of a local ring is always connected (Exercise 12).

\item Let $A$ be a Boolean ring (Exercise 11), and let $X=\Spec(A)$.
\begin{enumerate}[i)]
  \item For each $f \in A$, the set $X_{f}$ (Exercise 17) is both open and closed in
$X$.
  \item Let $f_{1}, \ldots, f_{n} \in A$. Show that
$X_{f_{1}} \cup \ldots \cup X_{f_{n}}=X_{f}$ for some $f \in A$.
  \item The sets $X_{f}$ are the only subsets of $X$ which are both open and closed.
[Let $Y \subseteq X$ be both open and closed. Since $Y$ is open, it is a union
of basic open sets $X_{f}$. Since $Y$ is closed and $X$ is quasi-compact
(Exercise 17), $Y$ is quasi-compact. Hence $Y$ is a finite union of basic open
sets; now use (ii) above.]
  \item $X$ is a compact Hausdorff space.
\end{enumerate}

  \item Let $L$ be a lattice, in which the sup and inf of two elements $a$, $b$
        are denoted by $a \vee b$ and $a \wedge b$ respectively. $L$ is a
        \textit{Boolean lattice} (or \textit{Boolean algebra}) if
\begin{enumerate}[i)]
  \item $L$ has a least element and a greatest element (denoted by $0$,$1$ respectively).
  \item Each of $\vee$, $\wedge$ is distributive over the other.
  \item Each $a \in L$ has a unique ``complement'' $a' \in L$ such that
$a \vee a'=1$ and $a \wedge a'=0$.
\end{enumerate}
(For example, the set of all subsets of a set, ordered by inclusion, is a
Boolean lattice.)

Let $L$ be a Boolean lattice. Define addition and multiplication in $L$ by the
rules

\[
  a+b=\left(a \wedge b'\right) \vee\left(a' \wedge b\right), \quad a b=a \wedge b .
\]

Verify that in this way $L$ becomes a Boolean ring, say $A(L)$.

Conversely, starting from a Boolean ring $A$, define an ordering on $A$ as
follows: $a \leq b$ means that $a=a b$. Show that, with respect to this
ordering, $A$ is a Boolean lattice. [The sup and inf are given by
$a \vee b=a+b+a b$ and $a \wedge b=a b$, and the complement by
$a'=1-a$.] In this way we obtain a one-to-one correspondence between
(isomorphism classes of) Boolean rings and (isomorphism classes of) Boolean
lattices.

  \item From the last two exercises deduce Stone's theorem, that every Boolean
        lattice is isomorphic to the lattice of open-and-closed subsets of some
        compact Hausdorff topological space.

  \item Let $A$ be a ring. The subspace of $\Spec(A)$ consisting
        of the \textit{maximal} ideals of $A$, with the induced topology, is called the
        \textit{maximal spectrum} of $A$ and is denoted by $\Max(A)$. For
        arbitrary commutative rings it does not have the nice functorial
        properties of $\Spec(A)$ (see Exercise 21), because the inverse image of
        a maximal ideal under a ring homomorphism need not be maximal.

Let $X$ be a compact Hausdorff space and let $C(X)$ denote the ring of all
real-valued continuous functions on $X$ (add and multiply functions by adding
and multiplying their values). For each $x \in X$, let $\mathfrak{m}_{x}$ be the
set of all $f \in C(X)$ such that $f(x)=0$. The ideal $\mathfrak{m}_{x}$ is
maximal, because it is the kernel of the (surjective) homomorphism
$C(X) \to \mathrm{R}$ which takes $f$ to $f(x)$. If $\tilde{X}$ denotes
$\Max(C(X))$, we have therefore defined a mapping
$\mu: X \to \tilde{X}$, namely $x \to \mathfrak{m}_{x}$. We shall show that $\mu$
is a homeomorphism of $X$ onto $\tilde{X}$.
\begin{enumerate}[i)]
  \item Let $\mathfrak{m}$ be any maximal ideal of $C(X)$, and let $V=V(\mathfrak{m})$ be the set of common zeros of the functions in $\mathfrak{m}$ : that is,
\[
  V=\{x \in X: f(x)=0 \text { for all } f \in m\}.
\]
Suppose that $V$ is empty. Then for each $x \in X$ there exists
$f_{x} \in \mathfrak{m}$ such that $f_{x}(x) \neq 0$. Since $f_{x}$ is
continuous, there is an open neighborhood $U_{x}$ of $x$ in $X$ on which $f_{x}$
does not vanish. By compactness a finite number of the neighborhoods, say
$U_{x_{1}}, \ldots, U_{x_{n}}$, cover $X$. Let
\[
  f=f_{x_{1}}^{2}+\cdots+f_{x_{n}}^{2}.
\]
Then $f$ does not vanish at any point of $X$, hence is a unit in $C(X)$. But
this contradicts $f \in \mathfrak{m}$, hence $V$ is not empty.

Let $x$ be a point of $V$. Then $\mathfrak{m} \subseteq \mathfrak{m}_{x}$, hence
$\mathfrak{m}=\mathfrak{m}_{x}$ because $\mathfrak{m}$ is maximal. Hence $\mu$
is surjective.
  \item By Urysohn's lemma (this is the only non-trivial fact required in the
argument) the continuous functions separate the points of $X$. Hence
$x \neq y \implies \mathfrak{m}_{x} \neq \mathfrak{m}_{y}$, and therefore $\mu$ is
injective.
  \item Let $f \in \bar{C}(\bar{X})$; let
\[
  U_{f}=\{x \in X: f(x) \neq 0\}
\]
and let
\[
  \tilde{U}_{f}=\{\mathfrak{m} \in \tilde{X}: f \notin \mathfrak{m}\}
\]
Show that $\mu\left(U_{f}\right)=\tilde{U}_{f}$. The open sets $U_{f}$ (resp.
$\tilde{U}_{f}$ ) form a basis of the topology of $X$ (resp. $\tilde{X}$ ) and
therefore $\mu$ is a homeomorphism.

Thus $X$ can be reconstructed from the ring of functions $C(X)$.
\end{enumerate}
\end{enumerate}

\subsection*{Affine algebraic varieties}
\begin{enumerate}[resume*=exc1]
  \item Let $k$ be an algebraically closed field and let
  \[
  f_{\alpha}\left(t_{1}, \ldots, t_{n}\right)=0
\]
be a set of polynomial equations in $n$ variables with coefficients in $k$. The
set $X$ of all points $x=\left(x_{1}, \ldots, x_{n}\right) \in k^{n}$ which
satisfy these equations is an \textit{affine algebraic variety}.

Consider the set of all polynomials $g \in k\left[t_{1}, \ldots, t_{n}\right]$
with the property that $g(x)=0$ for all $x \in X$. This set is an ideal $I(X)$
in the polynomial ring, and is called the \textit{ideal of the variety} $X$. The quotient ring
\[
  P(X)=k\left[t_{1}, \ldots, t_{n}\right] / I(X)
\]
is the ring of polynomial functions on $X$, because two polynomials $g, h$
define the same polynomial function on $X$ if and only if $g-h$ vanishes at
every point of $X$, that is, if and only if $g-h \in I(X)$.

Let $\xi_{l}$ be the
image of $t_{i}$ in $P(X)$. The $\xi_{t}$ ($1 \leq i \leq n$) are the \textit{coordinate
functions} on $X$: if $x \in X$, then $\xi_{i}(x)$ is the $i$ th coordinate of
$x$. $P(X)$ is generated as a $k$-algebra by the coordinate functions, and is
called the \textit{coordinate ring} (or affine algebra) of $X$.

As in Exercise 26, for each $x \in X$ let $\mathfrak{m}_{x}$ be the ideal of all
$f \in P(X)$ such that $f(x)=0$; it is a maximal ideal of $P(X)$. Hence, if
$\tilde{X}=\Max(P(X))$, we have defined a mapping
$\mu: X \to \tilde{X}$, namely $x \mapsto \mathfrak{m}_{x}$.

It is easy to show that $\mu$ is injective: if $x \neq y$, we must have
$x_{i} \neq y_{i}$ for for some $i$ ($1 \leq i \leq n$), and hence $\xi_{i}-x_{i}$
is in $\mathfrak{m}_{x}$ but not in $\mathfrak{m}_{y}$, so that
$\mathfrak{m}_{x} \neq \mathfrak{m}_{y}$. What is less obvious (but still true)
is that $\mu$ is surjective. This is one form of Hilbert's Nullstellensatz (see
Chapter 7).

\item Let $f_{1}, \ldots, f_{m}$ be elements of
        $k\left[t_{1}, \ldots, t_{n}\right]$. They determine a \textit{polynomial
        mapping} $\phi: k^{n} \to k^{m}$: if $x \in k^{n}$, the coordinates of
        $\phi(x)$ are $f_{1}(x), \ldots, f_{m}(x)$.

Let $X, Y$ be affine algebraic varieties in $k^{n}, k^{m}$ respectively. A
mapping $\phi: X \to Y$ is said to be \textit{regular} if $\phi$ is the restriction to
$X$ of a polynomial mapping from $k^{n}$ to $k^{m}$.

If $\eta$ is a polynomial function on $Y$, then $\eta \circ \phi$ is a
polynomial function on $X$. Hence $\phi$ induces a $k$-algebra homomorphism
$P(Y) \to P(X)$, namely $\eta \mapsto \eta \circ \phi$. Show that in this way we
obtain a one-to-one correspondence between the regular mappings $X \to Y$ and
the $k$-algebra homomorphisms $P(Y) \to P(X)$.
\end{enumerate}
\end{document}
