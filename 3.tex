\documentclass{standalone}
\usepackage{unicode-math}
\setmathfont{texgyrepagella-math.otf}[math-style=TeX]
\usepackage{fontspec}
\setmainfont{TeX Gyre Pagella}
\usepackage{amsthm}
\newtheorem{theorem}{Theorem}[chapter]
\newtheorem{proposition}[theorem]{Proposition}
\newtheorem{lemma}[theorem]{lemma}
\newtheorem*{example}{Example}
\theoremstyle{definition}
\newtheorem{definition}[theorem]{Definition}
\theoremstyle{remark}
\newtheorem*{remark}{Remark}
\usepackage[export]{adjustbox}\graphicspath{ {./images/} }
\begin{document}
\section{Rings and Modules of Fractions}
The formation of rings of fractions and the associated process of localization are perhaps the most important technical tools in commutative algebra. They correspond in the algebro-geometric picture to concentrating attention on an open set or near a point, and the importance of these notions should be selfevident. This chapter gives the definitions and simple properties of the formation of fractions.

The procedure by which one constructs the rational field $\mathbf{Q}$ from the ring of integers $\mathbf{Z}$ (and embeds $\mathbf{Z}$ in $\mathbf{Q}$ ) extends easily to any integral domain $A$ and produces the field of fractions of $A$. The construction consists in taking all ordered pairs $(a, s)$ where $a, s \in A$ and $s \neq 0$, and setting up an equivalence relation between such pairs:

\[
(a, s) \equiv(b, t) \Leftrightarrow a t-b s=0 .
\]

This works only if $A$ is an integral domain, because the verification that the relation is transitive involves canceling, i.e. the fact that $A$ has no zero-divisor $\neq 0$. However, it can be generalized as follows:

Let $A$ be any ring. A multiplicatively closed subset of $A$ is a subset $S$ of $A$ such that $1 \in S$ and $S$ is closed under multiplication: in other words $S$ is a subsemigroup of the multiplicative semigroup of $A$. Define a relation $\equiv$ on $A \times S$ as follows:

\[
(a, s) \equiv(b, t) \Leftrightarrow(a t-b s) u=0 \text { for some } u \in S .
\]

Clearly this relation is reflexive and symmetric. To show that it is transitive, suppose $(a, s) \equiv(b, t)$ and $(b, t) \equiv(c, u)$. Then there exist $v, w$ in $S$ such that $(a t-b s) v=0$ and $(b u-c t) w=0$. Eliminate $b$ from these two equations and we have $(a u-c s) t v w=0$. Since $S$ is closed under multiplication, we have $t v w \in S$, hence $(a, s) \equiv(c, u)$. Thus we have an equivalence relation. Let $a / s$ denote the equivalence class of $(a, s)$, and let $S^{-1} A$ denote the set of equivalence classes. We put a ring structure on $S^{-1} A$ by defining addition and multiplication of these "fractions" $a / s$ in the same way as in elementary algebra: that is,

\[
\begin{aligned}
(a / s)+(b / t) & =(a t+b s) / s t \\
(a / s)(b / t) & =a b / s t .
\end{aligned}
\]

Exercise. Verify that these definitions are independent of the choices of representatives $(a, s)$ and $(b, t)$, and that $S^{-1} A$ satisfies the axioms of a commutative ring with identity.

We also have a ring homomorphism $f: A \rightarrow S^{-1} A$ defined by $f(x)=x / 1$. This is not in general injective.

Remark. If $A$ is an integral domain and $S=A-\{0\}$, then $S^{-1} A$ is the field of fractions of $A$.

The ring $S^{-1} A$ is called the ring of fractions of $A$ with respect to $S$. It has a universal property:

Proposition 3.1. Let $g: A \rightarrow B$ be a ring homomorphism such that $g(s)$ is $a$ unit in $B$ for all $s \in S$. Then there exists a unique ring homomorphism $h: S^{-1} A \rightarrow B$ such that $g=h \circ f$.

Proof. i) Uniqueness. If $h$ satisfies the conditions, then $h(a / 1)=h f(a)=g(a)$ for all $a \in A$; hence, if $s \in S$,

\[
h(1 / s)=h\left((s / 1)^{-1}\right)=h(s / 1)^{-1}=g(s)^{-1}
\]

and therefore $h(a / s)=h(a / 1) \cdot h(1 / s)=g(a) g(s)^{-1}$, so that $h$ is uniquely determined by $g$.

ii) Existence. Let $h(a / s)=g(a) g(s)^{-1}$. Then $h$ will clearly be a ring homomorphism provided that it is well-defined. Suppose then that $a / s=a^{\prime} / s^{\prime}$; then there exists $t \in S$ such that $\left(a s^{\prime}-a^{\prime} s\right) t=0$, hence

\[
\left(g(a) g\left(s^{\prime}\right)-g\left(a^{\prime}\right) g(s)\right) g(t)=0 ;
\]

now $g(t)$ is a unit in $B$, hence $g(a) g(s)^{-1}=g\left(a^{\prime}\right) g\left(s^{\prime}\right)^{-1}$.

The ring $S^{-1} A$ and the homomorphism $f: A \rightarrow S^{-1} A$ have the following properties:

\begin{enumerate}
  \item $s \in S \Rightarrow f(s)$ is a unit in $S^{-1} A$;

  \item $f(a)=0 \Rightarrow a s=0$ for some $s \in S$;

  \item Every element of $S^{-1} A$ is of the form $f(a) f(s)^{-1}$ for some $a \in A$ and some $s \in S$.

\end{enumerate}

Conversely, these three conditions determine the ring $S^{-1} A$ up to isomorphism. Precisely:

Corollary 3.2. If $g: A \rightarrow B$ is a ring homomorphism such that

i) $s \in S \Rightarrow g(s)$ is a unit in $B$;

ii) $g(a)=0 \Rightarrow a s=0$ for some $s \in S$; iii) Every element of $B$ is of the form $g(a) g(s)^{-1}$; then there is a unique isomorphism $h: S^{-1} A \rightarrow B$ such that $g=h \circ f$.

Proof. By (3.1) we have to show that $h: S^{-1} A \rightarrow B$, defined by

\[
h(a / s)=g(a) g(s)^{-1}
\]

(this definition uses i)) is an isomorphism. By iii), $h$ is surjective. To show $h$ is injective, look at the kernel of $h:$ if $h(a / s)=0$, then $g(a)=0$, hence by ii) we have $a t=0$ for some $t \in S$, hence $(a, s) \equiv(0,1)$, i.e., $a / s=0$ in $S^{-1} A$.

Examples. 1) Let $\mathfrak{p}$ be a prime ideal of $A$. Then $S=A-\mathfrak{p}$ is multiplicatively closed (in fact $A-\mathfrak{p}$ is multiplicatively closed $\Leftrightarrow \mathfrak{p}$ is prime). We write $A_{\mathfrak{p}}$ for $S^{-1} A$ in this case. The elements $a / s$ with $a \in \mathfrak{p}$ form an ideal $\mathfrak{m}$ in $A_{\mathfrak{p}}$. If $b / t \notin \mathrm{m}$, then $b \notin \mathfrak{p}$, hence $b \in S$ and therefore $b / t$ is a unit in $A_{\mathfrak{p}}$. It follows that if $a$ is an ideal in $A_{\mathfrak{b}}$ and $\mathfrak{a} \neq \mathrm{m}$, then $\mathfrak{a}$ contains a unit and is therefore the whole ring. Hence $m$ is the only maximal ideal in $A_{\mathfrak{p}}$; in other words, $A_{\mathfrak{p}}$ is a local ring.

The process of passing from $A$ to $A_{\mathfrak{p}}$ is called localization at $\mathfrak{p}$.

\begin{enumerate}
  \setcounter{enumi}{1}
  \item $S^{-1} A$ is the zero ring $\Leftrightarrow 0 \in S$.

  \item Let $f \in A$ and let $S=\left\{f^{n}\right\}_{n>0}$. We write $A_{f}$ for $S^{-1} A$ in this case.

  \item Let $a$ be any ideal in $A$, and let $S=1+\mathfrak{a}=$ set of all $1+x$ where $x \in a$. Clearly $S$ is multiplicatively closed.

  \item Special cases of 1) and 3):

\end{enumerate}

i) $A=\mathbf{Z}, \mathfrak{p}=(p), p$ a prime number; $A_{\mathfrak{p}}=$ set of all rational numbers $m / n$ where $n$ is prime to $p$; if $f \in \mathbf{Z}$ and $f \neq 0$, then $A_{f}$ is the set of all rational numbers whose denominator is a power of $f$.

ii) $A=k\left[t_{1}, \ldots, t_{n}\right]$, where $k$ is a field and the $t_{t}$ are independent indeterminates, $b$ a prime ideal in $A$. Then $A_{y}$ is the ring of all rational functions $f / g$, where $g \notin \mathfrak{p}$. If $V$ is the variety defined by the ideal $\mathfrak{p}$, that is to say the set of all $x=\left(x_{1}, \ldots, x_{n}\right) \in k^{n}$ such that $f(x)=0$ whenever $f \in \mathfrak{p}$, then (provided $k$ is infinite) $A_{\mathfrak{b}}$ can be identified with the ring of all rational functions on $k^{n}$ which are defined at almost all points of $V$; it is the local ring of $k^{n}$ along the variety $V$. This is the prototype of the local rings which arise in algebraic geometry.

The construction of $S^{-1} A$ can be carried through with an $A$-module $M$ in place of the ring $A$. Define a relation $\equiv$ on $M \times S$ as follows:

\[
(m, s) \equiv\left(m^{\prime}, s^{\prime}\right) \Leftrightarrow \exists t \in S \text { such that } t\left(s m^{\prime}-s^{\prime} m\right)=0 .
\]

As before, this is an equivalence relation. Let $\mathrm{m} / \mathrm{s}$ denote the equivalence class of the pair $(m, s)$, let $S^{-1} M$ denote the set of such fractions, and make $S^{-1} M$ into an $S^{-1} A$-module with the obvious definitions of addition and scalar multiplication. As in Examples 1) and 3) above, we write $M_{\mathfrak{p}}$ instead of $S^{-1} M$ when $S=A-\mathfrak{p}\left(p\right.$ prime) and $M_{f}$ when $S=\left\{f^{n}\right\}_{n}>0$.

Let $u: M \rightarrow N$ be an $A$-module homomorphism. Then it gives rise to an $S^{-1} A$-module homomorphism $S^{-1} u: S^{-1} M \rightarrow S^{-1} N$, namely $S^{-1} u$ maps $m / s$ to $u(m) / s$. We have $S^{-1}(v \circ u)=\left(S^{-1} v\right) \circ\left(S^{-1} u\right)$. Proposition 3.3. The operation $S^{-1}$ is exact, i.e., if $M^{\prime} \stackrel{f}{\rightarrow} M \stackrel{0}{\rightarrow} M^{\prime \prime}$ is exact at $M$, then $S^{-1} M^{\prime} \stackrel{S^{-1} f}{\longrightarrow} S^{-1} M \stackrel{S^{-1} g}{\longrightarrow} S^{-1} M^{\prime \prime}$ is exact at $S^{-1} M$.

Proof. We have $g \circ f=0$, hence $S^{-1} g \circ S^{-1} f=S^{-1}(0)=0$, hence $\operatorname{Im}\left(S^{-1} f\right)$ $\subseteq \operatorname{Ker}\left(S^{-1} g\right)$. To prove the reverse inclusion, let $\mathrm{m} / \mathrm{s} \in \operatorname{Ker}\left(S^{-1} g\right)$, then $g(m) / s=0$ in $S^{-1} M^{\prime \prime}$, hence there exists $t \in S$ such that $\operatorname{tg}(m)=0$ in $M^{\prime \prime}$. But $\operatorname{tg}(m)=g(t m)$ since $g$ is an $A$-module homomorphism, hence $t m \in \operatorname{Ker}(g)=$ $\operatorname{Im}(f)$ and therefore $t m=f\left(m^{\prime}\right)$ for some $m^{\prime} \in M^{\prime}$. Hence in $S^{-1} M$ we have $m / s=f\left(m^{\prime}\right) / s t=\left(S^{-1} f\right)\left(m^{\prime} / s t\right) \in \operatorname{Im}\left(S^{-1} f\right)$. Hence $\operatorname{Ker}\left(S^{-1} g\right) \subseteq \operatorname{Im}\left(S^{-1} f\right)$.

In particular, it follows from (3.3) that if $M^{\prime}$ is a submodule of $M$, the mapping $S^{-1} M^{\prime} \rightarrow S^{-1} M$ is injective and therefore $S^{-1} M^{\prime}$ can be regarded as a submodule of $S^{-1} M$. With this convention,

Corollary 3.4. Formation of fractions commutes with formation of finite sums, finite intersections and quotients. Precisely, if $N, P$ are submodules of an A-module $M$, then

i) $S^{-1}(N+P)=S^{-1}(N)+S^{-1}(P)$

ii) $S^{-1}(N \cap P)=S^{-1}(N) \cap S^{-1}(P)$

iii) the $S^{-1} A$-modules $S^{-1}(M / N)$ and $\left(S^{-1} M\right) /\left(S^{-1} N\right)$ are isomorphic.

Proof. i) follows readily from the definitions and ii) is easy to verify: if $y / s=z / t(y \in N, z \in P, s, t \in S)$ then $u(t y-s z)=0$ for some $u \in S$, hence $w=u t y=u s z \in N \cap P$ and therefore $y / s=w / s t u \in S^{-1}(N \cap P)$. Consequently $S^{-1} N \cap S^{-1} P \subseteq S^{-1}(N \cap P)$, and the reverse inclusion is obvious.

iii) Apply $S^{-1}$ to the exact sequence $0 \rightarrow N \rightarrow M \rightarrow M / N \rightarrow 0$.

Proposition 3.5. Let $M$ be an A-module. Then the $S^{-1} A$ modules $S^{-1} M$ and $S^{-1} A \otimes_{A} M$ are isomorphic; more precisely, there exists $a$ unique isomorphism $f: S^{-1} A \otimes_{1} M \rightarrow S^{-1} M$ for which

\[
f((a / s) \otimes m)=a m / s \text { for all } a \in A, m \in M, s \in S \text {. }
\]

Proof. The mapping $S^{-1} A \times M \rightarrow S^{-1} M$ defined by

\[
(a / s, m) \mapsto a m / s
\]

is $A$-bilinear, and therefore by the universal property (2.12) of the tensor product induces an $A$-homomorphism

\[
f: S^{-1} A \otimes_{1} M \rightarrow S^{-1} M
\]

satisfying (1). Clearly $f$ is surjective, and is uniquely defined by (1).

Let $\sum_{t}\left(a_{t} / s_{t}\right) \otimes m_{1}$ be any element of $S^{-1} A \otimes M$. If $s=\Pi_{l} s_{t} \in S$, $t_{i}=\Pi_{j \neq 1} s_{j}$, we have

\[
\sum_{i} \frac{a_{i}}{s_{i}} \otimes m_{i}=\sum_{i} \frac{a_{i} t_{i}}{s} \otimes m=\sum_{i} \frac{1}{s} \otimes a_{i} t_{i} m=\frac{1}{s} \otimes \sum_{i} a_{i} t_{i} m,
\]

so that every element of $S^{-1} A \otimes M$ is of the form $(1 / s) \otimes m$. Suppose that $f((1 / s) \otimes m)=0$. Then $m / s=0$, hence $t m=0$ for some $t \in S$, and therefore

\[
\frac{1}{s} \otimes m=\frac{t}{s t} \otimes m=\frac{1}{s t} \otimes t m=\frac{1}{s t} \otimes 0=0 .
\]

Hence $f$ is injective and therefore an isomorphism.

Corollary 3.6. $S^{-1} A$ is a flat A-module.

Proof. (3.3), (3.5).

Proposition 3.7. If $M, N$ are A-modules, there is a unique isomorphism of $S^{-1} A$-modules $f: S^{-1} M \otimes_{S^{-1}} S^{-1} N \rightarrow S^{-1}\left(M \otimes_{A} N\right)$ such that

\[
f((m / s) \otimes(n / t))=(m \otimes n) / s t .
\]

In particular, if $\mathfrak{p}$ is any prime ideal, then

\[
M_{\mathfrak{p}} \otimes_{A_{\mathfrak{p}}} N_{\mathfrak{p}} \cong\left(M \otimes_{A} N\right)_{\mathfrak{p}}
\]

as $A_{\mathfrak{p}}$-modules.

Proof. Use (3.5) and the canonical isomorphisms of Chapter 2.

\section{LOCAL PROPERTIES}
A property $P$ of a ring $A$ (or of an $A$-module $M$ ) is said to be a local property if the following is true:

$A$ (or $M$ ) has $P \Leftrightarrow A_{\mathfrak{p}}$ (or $M_{\mathfrak{p}}$ ) has $P$, for each prime ideal $\mathfrak{p}$ of $A$. The following propositions give examples of local properties:

Proposition 3.8. Let $M$ be an A-module. Then the following are equivalent:

i) $M=0$;

ii) $M_{\mathfrak{p}}=0$ for all prime ideals $\mathfrak{p}$ of $A$;

iii) $M_{\mathfrak{m}}=0$ for all maximal ideals $\mathfrak{m}$ of $A$.

Proof. Clearly i) $\Rightarrow$ ii) $\Rightarrow$ iii). Suppose iii) satisfied and $M \neq 0$. Let $x$ be a non-zero element of $M$, and let $\mathfrak{a}=\operatorname{Ann}(x) ; \mathfrak{a}$ is an ideal $\neq(1)$, hence is contained in a maximal ideal $\mathfrak{m}$ by (1.4). Consider $x / l \in M_{\mathfrak{m}}$. Since $M_{\mathfrak{m}}=0$ we have $x / 1=0$, hence $x$ is killed by some element of $A-\mathfrak{m}$; but this is impossible since Ann $(x) \subseteq \mathfrak{m}$.

Proposition 3.9. Let $\phi: M \rightarrow N$ be an A-module homomorphism. Then the following are equivalent:

i) $\phi$ is injective;

ii) $\phi_{\mathfrak{p}}: M_{\mathfrak{p}} \rightarrow N_{\mathfrak{p}}$ is injective for each prime ideal $\mathfrak{p}$;

iii) $\phi_{\mathfrak{m}}: M_{\mathfrak{m}} \rightarrow N_{\mathfrak{n}}$ is injective for each maximal ideal $\mathfrak{m}$.

Similarly with "injective" replaced by "surjective" throughout. Proof. i) $\Rightarrow$ ii). $0 \rightarrow M \rightarrow N$ is exact, hence $0 \rightarrow M_{\mathfrak{p}} \rightarrow N_{\mathfrak{b}}$ is exact, i.e., $\phi_{\mathfrak{p}}$ is injective.

ii) $\Rightarrow$ iii) because a maximal ideal is prime.

iii) $\Rightarrow$ i). Let $M^{\prime}=\operatorname{Ker}(\phi)$, then the sequence $0 \rightarrow M^{\prime} \rightarrow M \rightarrow N$ is exact, hence $0 \rightarrow M_{m}^{\prime} \rightarrow M_{m} \rightarrow N_{m}$ is exact by (3.3) and therefore $M_{m}^{\prime} \cong$ $\operatorname{Ker}\left(\phi_{m}\right)=0$ since $\phi_{m}$ is injective. Hence $M^{\prime}=0$ by (3.8), hence $\phi$ is injective. For the other part of the proposition, just reverse all the arrows.

Flatness is a local property:

Proposition 3.10. For any A-module $M$, the following statements are equivalent:

i) $M$ is a flat A-module:

ii) $M_{\mathfrak{p}}$ is a flat $A_{\mathfrak{p}}$-module for each prime ideal $\mathfrak{p}$;

iii) $M_{\mathfrak{m}}$ is a flat $A_{\mathfrak{m}}$-module for each maximal ideal $\mathfrak{m}$.

Proof. i) $\Rightarrow$ ii) by (3.5) and (2.20).

ii) $\Rightarrow$ iii) O.K.

iii) $\Rightarrow$ i). If $N \rightarrow P$ is a homomorphism of $A$-modules, and $\mathfrak{m}$ is any maximal ideal of $A$, then

\[
\begin{aligned}
N \rightarrow P \text { injective } & \Rightarrow N_{\mathfrak{m}} \rightarrow P_{\mathfrak{m}} \text { injective, by (3.9) } \\
& \Rightarrow N_{\mathfrak{m}} \otimes_{A_{\mathfrak{m}}} M_{\mathfrak{m}} \rightarrow P_{\mathfrak{m}} \otimes_{A_{\mathfrak{m}}} M_{\mathfrak{m}} \text { injective, by (2.19) } \\
& \Rightarrow\left(N \otimes_{A} M\right)_{\mathfrak{m}} \rightarrow\left(P \otimes_{A} M\right)_{\mathfrak{m}} \text { injective, by (3.7) } \\
& \Rightarrow N \otimes_{A} M \rightarrow P \otimes_{A} M \text { injective, by (3.9). }
\end{aligned}
\]

Hence $M$ is flat by (2.19).

\section{EXTENDED AND CONTRACTED IDEALS IN RINGS OF FRACTIONS}
Let $A$ be a ring, $S$ a multiplicatively closed subset of $A$ and $f: A \rightarrow S^{-1} A$ the natural homomorphism, defined by $f(a)=a / 1$. Let $C$ be the set of contracted ideals in $A$, and let $E$ be the set of extended ideals in $S^{-1} A$ (cf. (1.17)). If $\mathfrak{a}$ is an ideal in $A$, its extension $a^{e}$ in $S^{-1} A$ is $S^{-1} \mathfrak{a}$ (for any $y \in a^{e}$ is of the form $\sum a_{\mathfrak{l}} / s_{\text {, }}$, where $a_{\mathfrak{t}} \in \mathfrak{a}$ and $s_{\mathfrak{i}} \in S$; bring this fraction to a common denominator).

Proposition 3.11. i) Every ideal in $S^{-1} A$ is an extended ideal.

ii) If $\mathfrak{a}$ is an ideal in $A$, then $\mathfrak{a}^{e c}=\bigcup_{s \in S}(\mathfrak{a}: s)$. Hence $\mathfrak{a}^{e}=(1)$ if and only if a meets $S$.

iii) $a \in C \Rightarrow$ no element of $S$ is a zero-divisor in $A / a$.

iv) The prime ideals of $S^{-1} A$ are in one-to-one correspondence (p $\leftrightarrow S^{-1} \mathfrak{p}$ ) with the prime ideals of $A$ which don't meet $S$. v) The operation $S^{-1}$ commutes with formation of finite sums, products, intersections and radicals.

Proof. i) Let $\mathfrak{b}$ be an ideal in $S^{-1} A$, and let $x / s \in \mathfrak{b}$. Then $x / 1 \in \mathfrak{b}$, hence $x \in \mathfrak{b}^{\mathfrak{c}}$ and therefore $x / s \in \mathfrak{b}^{c \varepsilon}$. Since $\mathfrak{b} \supseteq \mathfrak{b}^{c e}$ in any case (1.17), it follows that $\mathfrak{b}=\mathfrak{b}^{c e}$.

ii) $x \in \mathfrak{a}^{e c}=\left(S^{-1} \mathfrak{a}\right)^{c} \Leftrightarrow x / 1=a / s$ for some $a \in \mathfrak{a}, s \in S \Leftrightarrow(x s-a) t=0$ for some $t \in S \Leftrightarrow x s t \in \mathfrak{a} \Leftrightarrow x \in \bigcup_{s \in S}(\mathfrak{a}: s)$.

iii) $a \in C \Leftrightarrow \mathfrak{a}^{e c} \subseteq \mathfrak{a} \Leftrightarrow(s x \in \mathfrak{a}$ for some $s \in S \Rightarrow x \in \mathfrak{a}) \Leftrightarrow$ no $s \in S$ is a zero-divisor in $A / a$.

iv) If $q$ is a prime ideal in $S^{-1} A$, then $q^{c}$ is a prime ideal in $A$ (this much is true for any ring homomorphism). Conversely, if $\mathfrak{p}$ is a prime ideal in $A$, then $A / \mathfrak{p}$ is an integral domain; if $\bar{S}$ is the image of $S$ in $A / \mathfrak{p}$, we have $S^{-1} A / S^{-1} \mathfrak{p} \cong$ $S^{-1}(A / \mathfrak{p})$ which is either 0 or else is contained in the field of fractions of $A / \mathfrak{p}$ and is therefore an integral domain, and therefore $S^{-1} \mathfrak{p}$ is either prime or is the unit ideal; by i) the latter possibility occurs if and only if $\mathfrak{p}$ meets $S$.

v) For sums and products, this follows from (1.18); for intersections, from (3.4). As to radicals, we have $S^{-1} r(\mathfrak{a}) \subseteq r\left(S^{-1} \mathfrak{a}\right)$ from (1.18), and the proof of the reverse inclusion is a routine verification which we leave to the reader.

Remarks. 1) If $a, \mathfrak{b}$ are ideals of $A$, the formula

\[
S^{-1}(\mathfrak{a}: \mathfrak{b})=\left(S^{-1} \mathfrak{a}: S^{-1} \mathfrak{b}\right)
\]

is true provided the ideal $\mathfrak{b}$ is finitely generated: see (3.15).

\begin{enumerate}
  \setcounter{enumi}{1}
  \item The proof in (1.8) that if $f \in A$ is not nilpotent there is a prime ideal of $A$ which does not contain $f$ can be expressed more concisely in the language of rings of fractions. Since the set $S=\left(f^{n}\right)_{n>0}$ does not contain 0 , the ring $S^{-1} A=A_{f}$ is not the zero ring and therefore by (1.3) has a maximal ideal, whose contraction in $A$ is a prime ideal $\mathfrak{p}$ which does not meet $S$ by (3.11); hence $f \notin \mathfrak{p}$.
\end{enumerate}
\begin{corollary}
If $\Re$ is the inlradical of $A$, the nilradical of $S^{-1}A$ is $S^{-1}\Re$.
\end{corollary}

Corollary 3.13. If $\mathfrak{p}$ is a prime ideal of $A$, the prime ideals of the local ring $A_{\mathfrak{p}}$ are in one-to-one correspondence with the prime ideals of $A$ contained in $\mathfrak{p}$.

Proof. Take $S=A-\mathfrak{p}$ in (3.11) (iv).

Remark. Thus the passage from $A$ to $A_{\mathfrak{p}}$ cuts out all prime ideals except those contained in $\mathfrak{p}$. In the other direction, the passage from $A$ to $A / \mathfrak{p}$ cuts out all prime ideals except those containing $\mathfrak{p}$. Hence if $\mathfrak{p}, \mathfrak{q}$ are prime ideals such that $\mathfrak{p} \supseteq \mathfrak{q}$, then by localizing with respect to $\mathfrak{p}$ and taking the quotient mod $q$ (in either order: these two operations commute, by (3.4)), we restrict our attention to those prime ideals which lie between $\mathfrak{p}$ and $q$. In particular, if $\mathfrak{p}=\mathfrak{q}$ we end up with a field, called the residue field at $\mathfrak{p}$, which can be obtained either as the field of fractions of the integral domain $A / p$ or as the residue field of the local ring $A_{\mathfrak{p}}$.

Proposition 3.14. Let $M$ be a finitely generated A-module, $S$ a multiplicatively closed subset of $A$. Then $S^{-1}(\operatorname{Ann}(M))=\operatorname{Ann}\left(S^{-1} M\right)$.

Proof. If this is true for two $A$-modules, $M, N$, it is true for $M+N$ :

\[
\begin{aligned}
S^{-1}(\operatorname{Ann}(M+N)) & =S^{-1}(\operatorname{Ann}(M) \cap \operatorname{Ann}(N)) \text { by }(2.2) \\
& =S^{-1}(\operatorname{Ann}(M)) \cap S^{-1}(\operatorname{Ann}(N)) \text { by (3.4) } \\
& =\operatorname{Ann}\left(S^{-1} M\right) \cap \operatorname{Ann}\left(S^{-1} N\right) \text { by hypothesis } \\
& =\operatorname{Ann}\left(S^{-1} M+S^{-1} N\right)=\operatorname{Ann}\left(S^{-1}(M+N)\right) .
\end{aligned}
\]

Hence it is enough to prove (3.14) for $M$ generated by a single element: then $M \cong A / \mathfrak{a}$ (as $A$-module), where $\mathfrak{a}=\operatorname{Ann}(M) ; S^{-1} M \cong\left(S^{-1} A\right) /\left(S^{-1} \mathfrak{a}\right)$ by (3.4), so that Ann $\left(S^{-1} M\right)=S^{-1} \mathfrak{a}=S^{-1}(\operatorname{Ann}(M))$.

Corollary 3.15. If $N, P$ are submodules of an A-module $M$ and if $P$ is finitely generated, then $S^{-1}(N: P)=\left(S^{-1} N: S^{-1} P\right)$.

Proof. $(N: P)=$ Ann $((N+P) / N)$ by (2.2); now apply (3.14).

Proposition 3.16. Let $A \rightarrow B$ be a ring homomorphism and let $\mathfrak{p}$ be a prime ideal of $A$. Then $\mathfrak{p}$ is the contraction of a prime ideal of $B$ if and only if $\mathfrak{p}^{\text {ec }}=\mathfrak{p}$.

Proof. If $\mathfrak{p}=\mathfrak{q}^{c}$ then $\mathfrak{p}^{e c}=\mathfrak{p}$ by (1.17). Conversely, if $\mathfrak{p}^{e c}=\mathfrak{p}$, let $S$ be the image of $A-\mathfrak{p}$ in $B$. Then $\mathfrak{p}^{e}$ does not meet $S$, therefore by (3.11) its extension in $S^{-1} B$ is a proper ideal and hence is contained in a maximal ideal $\mathrm{m}$ of $S^{-1} B$. If $\mathfrak{q}$ is the contraction of $\mathfrak{m}$ in $B$, then $\mathfrak{q}$ is prime, $q \supseteq \mathfrak{p}^{*}$ and $q \cap S=\varnothing$. Hence $\mathfrak{q}^{c}=\mathfrak{p}$.

\section{EXERCISES}
\begin{enumerate}
  \item Let $S$ be a multiplicatively closed subset of a ring $A$, and let $M$ be a finitely generated $A$-module. Prove that $S^{-1} M=0$ if and only if there exists $s \in S$ such that $s M=0$.

  \item Let $a$ be an ideal of a ring $A$, and let $S=1+\mathfrak{a}$. Show that $S^{-1} \mathfrak{a}$ is contained in the Jacobson radical of $S^{-1} A$.

\end{enumerate}

Use this result and Nakayama's lemma to give a proof of (2.5) which does not depend on determinants. [If $M=a M$, then $S^{-1} M=\left(S^{-1} a\right)\left(S^{-1} M\right)$, hence by Nakayama we have $S^{-1} M=0$. Now use Exercise 1.]

\begin{enumerate}
  \setcounter{enumi}{2}
  \item Let $A$ be a ring, let $S$ and $T$ be two multiplicatively closed subsets of $A$, and let $U$ be the image of $T$ in $S^{-1} A$. Show that the rings $(S T)^{-1} A$ and $U^{-1}\left(S^{-1} A\right)$ are isomorphic. 4. Let $f: A \rightarrow B$ be a homomorphism of rings and let $S$ be a multiplicatively closed subset of $A$. Let $T=f(S)$. Show that $S^{-1} B$ and $T^{-1} B$ are isomorphic as $S^{-1} A$-modules.

  \item Let $A$ be a ring. Suppose that, for each prime ideal $\mathfrak{p}$, the local ring $A_{\mathfrak{p}}$ has no nilpotent element $\neq 0$. Show that $A$ has no nilpotent element $\neq 0$. If each $A_{\mathfrak{p}}$ is an integral domain, is $A$ necessarily an integral domain?

  \item Let $A$ be a ring $\neq 0$ and let $\Sigma$ be the set of all multiplicatively closed subsets $S$ of $A$ such that $0 \notin S$. Show that $\Sigma$ has maximal elements, and that $S \in \Sigma$ is maximal if and only if $A-S$ is a minimal prime ideal of $A$.

  \item A multiplicatively closed subset $S$ of a ring $A$ is said to be saturated if

\end{enumerate}

Prove that

\[
x y \in S \Leftrightarrow x \in S \text { and } y \in S \text {. }
\]

i) $S$ is saturated $\Leftrightarrow A-S$ is a union of prime ideals.

ii) If $S$ is any multiplicatively closed subset of $A$, there is a unique smallest saturated multiplicatively closed subset $\bar{S}$ containing $S$, and that $S$ is the complement in $A$ of the union of the prime ideals which do not meet $S$. ( $S$ is called the saturation of $S$.)

If $S=1+a$, where $a$ is an ideal of $A$, find $S$.

\begin{enumerate}
  \setcounter{enumi}{7}
  \item Let $S, T$ be multiplicatively closed subsets of $A$, such that $S \subseteq T$. Let $\phi: S^{-1} A$ $\rightarrow T^{-1} A$ be the homomorphism which maps each $a / s \in S^{-1} A$ to $a / s$ considered as an element of $T^{-1} A$. Show that the following statements are equivalent:
\end{enumerate}

i) $\phi$ is bijective.

ii) For each $t \in T, t / 1$ is a unit in $S^{-1} A$.

iii) For each $t \in T$ there exists $x \in A$ such that $x t \in S$.

iv) $T$ is contained in the saturation of $S$ (Exercise 7).

v) Every prime ideal which meets $T$ also meets $S$.

\begin{enumerate}
  \setcounter{enumi}{8}
  \item The set $S_{0}$ of all non-zero-divisors in $A$ is a saturated multiplicatively closed subset of $A$. Hence the set $D$ of zero-divisors in $A$ is a union of prime ideals (see Chapter 1, Exercise 14). Show that every minimal prime ideal of $A$ is contained in $D$. [Use Exercise 6.]
\end{enumerate}

The ring $S_{0}^{-1} A$ is called the total ring of fractions of $A$. Prove that

i) $S_{0}$ is the largest multiplicatively closed subset of $A$ for which the homomorphism $A \rightarrow S_{0}^{-1} A$ is injective.

ii) Every element in $S_{0}^{-1} A$ is either a zero-divisor or a unit.

iii) Every ring in which every non-unit is a zero-divisor is equal to its total ring of fractions (i.e., $A \rightarrow S_{0}^{-1} A$ is bijective).

\begin{enumerate}
  \setcounter{enumi}{9}
  \item Let $A$ be a ring.
\end{enumerate}

i) If $A$ is absolutely flat (Chapter 2, Exercise 27) and $S$ is any multiplicatively closed subset of $A$, then $S^{-1} A$ is absolutely flat.

ii) $A$ is absolutely flat $\Leftrightarrow A_{\mathfrak{m}}$ is a field for each maximal ideal $\mathrm{m}$.

\begin{enumerate}
  \setcounter{enumi}{10}
  \item Let $A$ be a ring. Prove that the following are equivalent:
\end{enumerate}

i) $A / \mathfrak{N}$ is absolutely flat ( $\mathfrak{R}$ being the nilradical of $A$ ).

ii) Every prime ideal of $A$ is maximal. iii) Spec $(A)$ is a $T_{1}$-space (i.e., every subset consisting of a s'ngle point is closed).

iv) $\operatorname{Spec}(A)$ is Hausdorff.

If these conditions are satisfied, show that $\operatorname{Spec}(.4)$ is compact and totally disconnected (i.e. the only connected subsets of Spec $(A)$ are those consisting of a single point).

\begin{enumerate}
  \setcounter{enumi}{11}
  \item Let $A$ be an integral domain and $M$ an $A$-module. An element $x \in M$ is a torsion element of $M$ if Ann (x) $\neq 0$, that is if $x$ is killed by some non-zero element of $A$. Show that the torsion elements of $M$ form a submodule of $M$. This submodule is called the torsion submodule of $M$ and is denoted by $T(M)$. If $T(M)=0$, the module $M$ is said to be torsion-free. Show that
\end{enumerate}

i) If $M$ is any $A$-module, then $M / T(M)$ is torsion-free.

ii) If $f: M \rightarrow N$ is a module homomorphism, then $f(T(M)) \subseteq T(N)$.

iii) If $0 \rightarrow M^{\prime} \rightarrow M \rightarrow M^{\prime \prime}$ is an exact sequence, then the sequence $0 \rightarrow T\left(M^{\prime}\right)$ $\rightarrow T(M) \rightarrow T\left(M^{\prime \prime}\right)$ is exact.

iv) If $M$ is any $A$-module, then $T(M)$ is the kernel of the mapping $x \mapsto 1 \otimes x$ of $M$ into $K \otimes_{A} M$, where $K$ is the field of fractions of $A$.

[For iv), show that $K$ may be regarded as the direct limit of its submodules $A \xi(\xi \in K)$; using Chapter 1 , Exercise 15 and Exercise 20, show that if $1 \otimes x=0$ in $K \otimes M$ then $1 \otimes x=0$ in $A \xi \otimes M$ for some $\xi \neq 0$. Deduce that $\xi^{-1} x=0$.]

\begin{enumerate}
  \setcounter{enumi}{12}
  \item Let $S$ be a multiplicatively closed subset of an integral domain $A$. In the notation of Exercise 12, show that $T\left(S^{-1} M\right)=S^{-1}(T M)$. Deduce that the following are equivalent:
\end{enumerate}

i) $M$ is torsion-free.

ii) $M_{\mathfrak{p}}$ is torsion-free for all prime ideals $\mathfrak{p}$.

iii) $M_{\mathfrak{m}}$ is torsion-free for all maximal ideals $\mathrm{m}$.

\begin{enumerate}
  \setcounter{enumi}{13}
  \item Let $M$ be an $A$-module and $a$ an ideal of $A$. Suppose that $M_{\mathfrak{m}}=0$ for all maximal ideals $m \supseteq \mathfrak{a}$. Prove that $M=\mathfrak{a} M$. [Pass to the $A / \mathfrak{a}$-module $M / \mathfrak{a} M$ and use (3.8).]

  \item Let $A$ be a ring, and let $F$ be the $A$-module $A^{n}$. Show that every set of $n$ generators of $F$ is a basis of $F$. [Let $x_{1}, \ldots, x_{n}$ be a set of generators and $e_{1}, \ldots, e_{n}$ the canonical basis of $F$. Define $\phi: F \rightarrow F$ by $\phi\left(e_{i}\right)=x_{i}$. Then $\phi$ is surjective and we have to prove that it is an isomorphism. By (3.9) we may assume that $A$ is a local ring. Let $N$ be the kernel of $\phi$ and let $k=A / \mathrm{m}$ be the residue field of $A$. Since $F$ is a flat $A$-module, the exact sequence $0 \rightarrow N \rightarrow F \rightarrow F \rightarrow 0$ gives an exact sequence $0 \longrightarrow k \otimes N \longrightarrow k \otimes F \stackrel{1 \otimes \phi}{\longrightarrow} k \otimes F \longrightarrow 0$. Now $k \otimes F=k^{n}$ is an $n$-dimensional vector space over $k ; 1 \otimes \phi$ is surjective, hence bijective, hence $k \otimes N=0$.

\end{enumerate}

Also $N$ is finitely generated, by Chapter 2 , Exercise 12 , hence $N=0$ by Nakayama's lemma. Hence $\phi$ is an isomorphism.]

Deduce that every set of generators of $F$ has at least $n$ elements.

\begin{enumerate}
  \setcounter{enumi}{15}
  \item Let $B$ be a flat $A$-algebra. Then the following conditions are equivalent:
\end{enumerate}

i) $\mathfrak{a}^{e c}=a$ for all ideals $\mathfrak{a}$ of $A$.

ii) $\operatorname{Spec}(B) \rightarrow \operatorname{Spec}(A)$ is surjective.

iii) For every maximal ideal $m$ of $A$ we have $\mathfrak{m}^{e} \neq(1)$. iv) If $M$ is any non-zero $A$-module, then $M_{B} \neq 0$.

v) For every $A$-module $M$, the mapping $x \mapsto 1 \otimes x$ of $M$ into $M_{B}$ is injective. (For i) $\Rightarrow$ ii), use (3.16). ii) $\Rightarrow$ iii) is clear.

iii) $\Rightarrow$ iv): Let $x$ be a non-zero element of $M$ and let $M^{\prime}=A x$. Since $B$ is flat over $A$ it is enough to show that $M_{B}^{\prime} \neq 0$. We have $M^{\prime} \cong A / a$ for some ideal $a \neq(1)$, hence $M_{B}^{\prime} \cong B / \alpha^{*}$. Now $a \subseteq \mathfrak{m}$ for some maximal ideal $m$, hence $Q^{e} \subseteq \pi^{e} \neq(1)$. Hence $M_{B}^{\prime} \neq 0$.

iv) $\Rightarrow v$ ): Let $M^{\prime}$ be the kernel of $M \rightarrow M_{B}$. Since $B$ is flat over $A$, the sequence $0 \rightarrow M_{B}^{\prime} \rightarrow M_{B} \rightarrow\left(M_{B}\right)_{B}$ is exact. But (Chapter 2, Exercise 13, with $\bar{N}=M_{B}$ ) the mapping $M_{B} \rightarrow\left(M_{B}\right)_{B}$ is injective, hence $M_{B}^{\prime}=0$ and therefore $M^{\prime}=0$.

v) $\Rightarrow$ i): Take $M=A /$ a.]

$B$ is said to be faithfully flat over $A$.

\begin{enumerate}
  \setcounter{enumi}{16}
  \item Let $A \stackrel{f}{\rightarrow} B \stackrel{g}{\rightarrow} C$ be ring homomorphisms. If $g \circ f$ is flat and $g$ is faithfully fiat, then $f$ is fiat.

  \item Let $f: A \rightarrow B$ be a flat homomorphism of rings, let $q$ be a prime ideal of $B$ and let $\mathfrak{p}=q^{\mathfrak{c}}$. Then $f^{*}: \operatorname{Spec}\left(B_{q}\right) \rightarrow \operatorname{Spec}\left(A_{\mathfrak{p}}\right)$ is surjective. [For $B_{\mathfrak{p}}$ is fiat over $A_{\mathfrak{p}}$ by (3.10), and $B_{q}$ is a local ring of $B_{\mathfrak{p}}$, hence is flat over $B_{\mathfrak{p}}$. Hence $B_{q}$ is flat over $A_{\mathfrak{p}}$ and satisfies condition (3) of Exercise 16.]

  \item Let $A$ be a ring, $M$ an $A$-module. The support of $M$ is defined to be the set Supp $(M)$ of prime ideals $\mathfrak{p}$ of $A$ such that $M_{\mathfrak{p}} \neq 0$. Prove the following results:

\end{enumerate}

i) $M \neq 0 \leftrightarrow \operatorname{Supp}(M) \neq \varnothing$.

ii) $V(\mathfrak{a})=\operatorname{Supp}(A / \mathfrak{a})$.

iii) If $0 \rightarrow M^{\prime} \rightarrow M \rightarrow M^{\prime \prime} \rightarrow 0$ is an exact sequence, then Supp $(M)=$ Supp $\left(M^{\prime}\right) \cup \operatorname{Supp}\left(M^{\prime}\right)$.

iv) If $M=\sum M_{i}$, then $\operatorname{Supp}(M)=\bigcup \operatorname{Supp}\left(M_{0}\right)$.

v) If $M$ is finitely generated, then Supp $(M)=V(\operatorname{Ann}(M)$ ) (and is therefore a closed subset of Spec $(A)$ ).

vi) If $M, N$ are finitely generated, then $\operatorname{Supp}\left(M \otimes_{A} N\right)=\operatorname{Supp}(M) \cap$ Supp ( $N$ ), - [Use Chapter 2, Exercise 3.]

vii) If $M$ is finitely generated and $\mathfrak{a}$ is an ideal of $A$, then Supp $(M / a M)=$ $V(\mathfrak{a}+\operatorname{Ann}(M))$.

viii) If $f: A \rightarrow B$ is a ring homomorphism and $M$ is a finitely generated $A$ module, then Supp $\left(B \otimes_{A} M\right)=f^{\star-1}(\operatorname{Supp}(M))$.

\begin{enumerate}
  \setcounter{enumi}{19}
  \item Let $f: A \rightarrow B$ be a ring homomorphism, $f^{\star}: \operatorname{Spec}(B) \rightarrow \operatorname{Spec}(A)$ the associated mapping. Show that
\end{enumerate}

i) Every prime ideal of $A$ is a contracted ideal $\Leftrightarrow f^{*}$ is surjective.

ii) Every prime ideal of $B$ is an extended ideal $\Rightarrow f^{*}$ is injective.

Is the converse of ii) true?

\begin{enumerate}
  \setcounter{enumi}{20}
  \item i) Let $A$ be a ring, $S$ a multiplicatively closed subset of $A$, and $\phi: A \rightarrow S^{-1} A$ the canonical homomorphism. Show that $\phi^{*}: \operatorname{Spec}\left(S^{-1} A\right) \rightarrow \operatorname{Spec}(A)$ is a homeomorphism of Spec $\left(S^{-1} A\right)$ onto its image in $X=\operatorname{Spec}(A)$. Let this image be denoted by $S^{-1} X$.
\end{enumerate}

In particular, if $f \in A$, the image of Spec $\left(A_{f}\right)$ in $X$ is the basic open set $X_{f}$ (Chapter 1, Exercise 17). ii) Let $f: A \rightarrow B$ be a ring homomorphism. Let $X=\operatorname{Spec}(A)$ and $Y=$ Spec $(B)$, and let $f^{*}: Y \rightarrow X$ be the mapping associated with $f$. Identifying Spec $\left(S^{-1} A\right)$ with its canonical image $S^{-1} X$ in $X$, and $\operatorname{Spec}\left(S^{-1} B\right)$ $\left(=\operatorname{Spec}\left(f(S)^{-1} B\right)\right.$ ) with its canonical image $S^{-1} Y$ in $Y$, show that $S^{-1} f^{*}$ : $\operatorname{Spec}\left(S^{-1} B\right) \rightarrow \operatorname{Spec}\left(S^{-1} A\right)$ is the restriction of $f^{*}$ to $S^{-1} Y$, and that $S^{-1} Y=f^{*-1}\left(S^{-1} X\right)$

iii) Let $a$ be an ideal of $A$ and let $b=a^{a}$ be its extension in $B$. Let $f: A / a \rightarrow B / b$ be the homomorphism induced by $f$. If $\operatorname{Spec}(A / \mathfrak{a})$ is identified with its canonical image $V(\mathfrak{a})$ in $X$, and $\operatorname{Spec}(B / \mathfrak{b})$ with its image $V(\mathfrak{b})$ in $Y$, show that $f^{*}$ is the restriction of $f^{*}$ to $V(b)$.

iv) Let $\mathfrak{p}$ be a prime ideal of $A$. Take $S=A-\mathfrak{p}$ in ii) and then reduce $\bmod S^{-1} \mathfrak{p}$ as in iii). Deduce that the subspace $f^{*-1}(\mathfrak{p})$ of $Y$ is naturally homeomorphic to $\operatorname{Spec}\left(B_{\mathfrak{p}} / \mathfrak{p} B_{\mathfrak{p}}\right)=\operatorname{Spec}\left(k(\mathfrak{p}) \otimes_{A} B\right)$, where $k(\mathfrak{p})$ is the residue field of the local ring $A_{\mathfrak{p}}$.

Spec $\left(k(\mathfrak{p}) \otimes_{1} B\right)$ is called the fiber of $f^{*}$ over $\mathfrak{p}$.

\begin{enumerate}
  \setcounter{enumi}{21}
  \item Let $A$ be a ring and $p$ a prime ideal of $A$. Then the canonical image of $\operatorname{Spec}\left(A_{\mathfrak{p}}\right)$ in Spec $(A)$ is equal to the intersection of all the open neighborhoods of $\mathfrak{p}$ in Spec $(A)$.

  \item Let $A$ be a ring, let $X=\operatorname{Spec}(A)$ and let $U$ be a basic open set in $X$ (i.e., $U=X$, for some $f \in A$ : Chapter 1, Exercise 17).

\end{enumerate}

i) If $U=X_{f}$, show that the ring $A(U)=A_{f}$ depends only on $U$ and not on $f$.

ii) Let $U^{\prime}=X_{g}$ be another basic open set such that $U^{\prime} \subseteq U$. Show that there is an equation of the form $g^{n}=u f$ for some integer $n>0$ and some $u \in A$, and use this to define a homomorphism $\rho: A(U) \rightarrow A\left(U^{\prime}\right)$ (i.e., $A_{f} \rightarrow A_{g}$ ) by mapping $a / f^{m}$ to $a u^{m} / g^{m n}$. Show that $\rho$ depends only on $U$ and $U^{\prime}$. This homomorphism is called the restriction homomorphism.

iii) If $U=U^{\prime}$, then $\rho$ is the identity map.

iv) If $U \supseteq U^{\prime} \supseteq U^{\prime}$ are basic open sets in $X$, show that the diagram

\begin{center}
  \begin{tikzcd}
A(U) \arrow[rd] \arrow[rr] &                  & A(U'') \\
                           & A(U') \arrow[ru] &
\end{tikzcd}
\end{center}

(in which the arrows are restriction homomorphisms) is commutative.

v) Let $x(=\mathfrak{p})$ be a point of $X$. Show that

\[
\varinjlim_{U\ni x} A(U) \cong A_{\mathfrak{p}}.
\]

The assignment of the ring $A(U)$ to each basic open set $U$ of $X$, and the restriction homomorphisms $\rho$, satisfying the conditions iii) and iv) above, constitutes a \textit{presheaf of rings} on the basis of open sets $\left(X_{f}\right)_{f\in A}$. v) says that the stalk of this presheaf at $x \in X$ is the corresponding local ring $A_{\mathfrak{p}}$.

\begin{enumerate}
  \setcounter{enumi}{23}
  \item Show that the presheaf of Exercise 23 has the following property. Let $\left(U_{t}\right)_{\text {ler }}$ be a covering of $X$ by basic open sets. For each $i \in I$ let $s_{i} \in A\left(U_{i}\right)$ be such that, for each pair of indices $i, j$, the images of $s_{i}$ and $s_{j}$ in $A\left(U_{i} \cap U_{j}\right)$ are equal. Then there exists a unique $s \in A(=A(X))$ whose image in $A\left(U_{i}\right)$ is $s_{i}$, for all $i \in I$. (This essentially implies that the presheaf is a sheaf.) 25. Let $f: A \rightarrow B, g: A \rightarrow C$ be ring homomorphisms and let $h: A \rightarrow B \otimes_{A} C$ be defined by $h(x)=f(x) \otimes g(x)$. Let $X, Y, Z, T$ be the prime spectra of $A, B, C$, $B \otimes_{A} C$ respectively. Then $h^{*}(T)=f^{*} Y \cap g^{*}(Z)$.
\end{enumerate}

[Let $\mathfrak{p} \in X$, and let $k=k(\mathfrak{p})$ be the residue field at $\mathfrak{p}$. By Exercise 21, the fiber $h^{*-1}(\mathfrak{p})$ is the spectrum of $\left(B \otimes_{A} C\right) \otimes_{A} k \cong\left(B \otimes_{A} k\right) \otimes_{k}\left(C \otimes_{A} k\right)$. Hence $\mathfrak{p} \in h^{*}(T) \Leftrightarrow\left(B \otimes_{A} k\right) \otimes_{k}\left(C \otimes_{A} k\right) \neq 0 \Leftrightarrow B \otimes_{A} k \neq 0$ and $C \otimes_{A} k \neq 0 \Leftrightarrow$ $\left.\mathfrak{p} \in f^{*}(Y) \cap g^{*}(Z).\right]$

2ó. Lêt $\left(B_{\alpha}, g_{\alpha \beta}\right)$ be a direct system of rings and $B$ the direct limit. For each $\alpha$, let $f_{\alpha}: A \rightarrow B_{\alpha}$ be a ring homomorphism such that $g_{\alpha \beta} \circ f_{\alpha}=f_{\beta}$ whenever $\alpha \leqslant \beta$ (i.e. the $B_{\alpha}$ form a direct system of $A$-algebras). The $f_{\alpha}$ induce $f: A \rightarrow B$. Show that

\[
f^{*}(\operatorname{Spec}(B))=\bigcap_{\alpha} f_{\alpha}^{*}\left(\operatorname{Spec}\left(B_{\alpha}\right)\right)
\]

[Let $\mathfrak{p} \in \operatorname{Spec}(A)$. Then $f^{*-1}(\mathfrak{p})$ is the spectrum of

\[
B \otimes_{A} k(\mathfrak{p}) \cong \lim _{\longrightarrow}\left(B_{\alpha} \otimes_{A} k(\mathfrak{p})\right)
\]

(since tensor products commute with direct limits: Chapter 2, Exercise 20). By Exercise 21 of Chapter 2 it follows that $f^{*-1}(p)=\varnothing$ if and only if $B_{\alpha} \otimes_{A} k(p)=0$ for some $\alpha$, i.e., if and only if $f_{\alpha}^{*-1}(p)=\varnothing$.]

\begin{enumerate}
  \setcounter{enumi}{26}
  \item i) Let $f_{\alpha}: A \rightarrow B_{\alpha}$ be any family of $A$-algebras and let $f: A \rightarrow B$ be their tensor product over $A$ (Chapter 2, Exercise 23). Then
\end{enumerate}

\[
f^{*}(\operatorname{Spec}(B))=\bigcap_{\alpha} f_{\alpha}^{*}\left(\operatorname{Spec}\left(B_{\alpha}\right)\right)
\]

[Use Examples 25 and 26.]

ii) Let $f_{\alpha}: A \rightarrow B_{\alpha}$ be any finite family of $A$-algebras and let $B=\prod_{\alpha} B_{\alpha}$. Define $f: A \rightarrow B$ by $f(x)=\left(f_{\alpha}(x)\right)$. Then $f^{*}(\operatorname{Spec}(B))=\bigcup_{\alpha} f_{\alpha}^{*}\left(\operatorname{Spec}\left(B_{\alpha}\right)\right)$.

iii) Hence the subsets of $X=\operatorname{Spec}(A)$ of the form $f^{*}(\operatorname{Spec}(B))$, where $f: A \rightarrow B$ is a ring homomorphism, satisfy the axioms for closed sets in a topological space. The associated topology is the constructible topology on $X$. It is finer than the Zariski topology (i.e., there are more open sets, or equivalently more closed sets).

iv) Let $X_{C}$ denote the set $X$ endowed with the constructible topology. Show that $X_{C}$ is quasi-compact.

\begin{enumerate}
  \setcounter{enumi}{27}
  \item (Continuation of Exercise 27.)
\end{enumerate}

i) For each $g \in A$, the set $X_{g}$ (Chapter 1, Exercise 17) is both open and closed in the constructible topology.

ii) Let $C^{\prime}$ denote the smallest topology on $X$ for which the sets $X_{g}$ are both open and closed, and let $X_{C}$, denote the set $X$ endowed with this topology. Show that $X_{C^{\prime}}$ is Hausdorff.

iii) Deduce that the identity mapping $X_{C} \rightarrow X_{C}$, is a homeomorphism. Hence a subset $E$ of $X$ is of the form $f^{*}(\operatorname{Spec}(B))$ for some $f: A \rightarrow B$ if and only if it is closed in the topology $C^{\prime}$.

iv) The topological space $X_{C}$ is compact, Hausdorff and totally disconnected. 29. Let $f: A \rightarrow B$ be a ring homomorphism. Show that $f^{*}: \operatorname{Spec}(B) \rightarrow \operatorname{Spec}(A)$ is a continuous closed mapping (i.e., maps closed sets to closed sets) for the constructible topology.

\begin{enumerate}
  \setcounter{enumi}{29}
  \item Show that the Zariski topology and the constructible topology on $\operatorname{Spec}(A)$ are the same if and only if $A / \mathfrak{R}$ is absolutely flat (where $\mathfrak{N}$ is the nilradical of $A$ ). [Use Exercise 11.]
\end{enumerate}
\end{document}
